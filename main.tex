\documentclass[journal]{IEEEtran}

\usepackage{adjustbox}
\usepackage{algorithm}
\usepackage{algpseudocode}
\usepackage{amsfonts}
\usepackage{amsmath}
\usepackage{amssymb}
\usepackage{amsthm}
\usepackage{array}
% \usepackage{caption}
\usepackage{cite}
\usepackage{environ}
\usepackage{grffile}
\usepackage{hyperref}
\usepackage{import}
\usepackage{mathtools}
\usepackage{microtype}
\usepackage{pgfplots}
\usepackage{siunitx}
\usepackage{stfloats}
\usepackage{url}
\usepackage{xcolor}
\usepackage[RPvoltages]{circuitikz}
\usepackage[T1]{fontenc}
\usepackage[caption=false,font=footnotesize]{subfig}
% \usepackage[cmintegrals]{newtxmath}
\usepackage[short]{optidef}


\interdisplaylinepenalty=2500
\pgfplotsset{compat=newest}
\usetikzlibrary{plotmarks}
\usetikzlibrary{arrows.meta}
\usepgfplotslibrary{patchplots}
\newtheorem{proposition}{Proposition}
\DeclareSIUnit{\belm}{Bm}
\DeclareSIUnit{\dBm}{\deci\belm}
\DeclareSIUnit{\beli}{Bi}
\DeclareSIUnit{\dBi}{\deci\beli}

\makeatletter
\newcommand{\forcealgorithm}{\let\@latex@error\@gobble}
\makeatother


\begin{document}
	\title{IRS-aided SWIPT: Joint waveform, active and passive beamforming design}
	\author{
		\IEEEauthorblockN{
			Yang~Zhao,~\IEEEmembership{Student~Member,~IEEE,}
			Bruno~Clerckx,~\IEEEmembership{Fellow,~IEEE,}
			and~Zhenyuan~Feng,~\IEEEmembership{Student~Member,~IEEE}
		}
		\thanks{
			The authors are with the Department of Electrical and Electronic Engineering, Imperial College London, London SW7 2AZ, U.K. (e-mail: yang.zhao18@imperial.ac.uk; b.clerckx@imperial.ac.uk).
		}
	}
	\maketitle


	\begin{abstract}
		The performance of Simultaneous Wireless Information and Power Transfer (SWIPT) is mainly restricted by the strength of the received Radio-Frequency (RF) signal. To tackle this problem, we introduce a low-power Intelligent Reflecting Surface (IRS) that compensates the propagation loss and boosts the energy efficiency with a passive beamforming gain. This paper investigates an IRS-aided Orthogonal Frequency Division Multiplexing (OFDM) SWIPT system based on a practical nonlinear energy harvester model, where a multi-antenna Access Point (AP) transmits information and energy simultaneously to a single-antenna user under the assist of IRS. We aim to maximize the Rate-Energy (R-E) region via jointly optimizing the transmit waveform at the AP, the reflection coefficients at the IRS, and the power splitting ratio at the user. The performance of the proposed waveform and beamforming design is compared with those of no IRS, fixed IRS and ideal frequency-selective (FS) IRS, and we confirm that due to rectifier nonlinearity, a dedicated power signal can be beneficial to energy harvesting (EH) while the optimal transceiving strategy depends on the system configuration. Simulation results also demonstrate that the proposed adaptive IRS design brings significant R-E enhancement over benchmark schemes for broadband transmission, and the optimal IRS can be approximated in closed form with negligible performance loss for SISO narrowband transmission.
	\end{abstract}


	\begin{IEEEkeywords}
		Wireless information and power transfer, intelligent reflecting surface, waveform design, active and passive beamforming.
	\end{IEEEkeywords}


	\begin{section}{Introduction}
		\begin{subsection}{Simultaneous Wireless Information and Power Transfer}
			\IEEEPARstart{W}{ith} the great advance in communication performance, the main challenge of wireless network has come to energy supply. Most existing mobile devices are powered by batteries that require frequent charging or replacement, which leads to high maintenance cost thus restricts the scale of networks. Although solar energy and inductive coupling has become popular alternatives, the former depends on the environment while the latter has a very short operation range. Simultaneous Wireless Information and Power Transfer (SWIPT) is a promising solution to connect and power mobile devices via electromagnetic (EM) waves in the Radio-Frequency (RF) band. It provides low power in \si{\uW} level but broad coverage up to hundreds of meters in a sustainable and controllable manner \cite{Ng2018}. The decreasing trend of electronic power consumption also boosts the paradigm shift from dedicated power source to Wireless Power Transfer (WPT) and SWIPT.

			The concept of SWIPT were first cast in \cite{Varshney2008}, where the authors investigated the Rate-Energy (R-E) tradeoff for a flat Gaussian channel and typical discrete channels. Two practical receiver structures were then proposed in \cite{Zhou2013}, namely Time Switching (TS) that switches between Energy Harvesting (EH) and Information Decoding (ID) modes, and Power Splitting (PS) that splits the received signal into individual components. On top of this, \cite{Zhang2013} characterized the R-E region for a Multiple-Input Multiple-Output (MIMO) broadcast system under TS and PS setup. Information and power beamforming was then considered in multiuser Multi-Input Single-Output (MISO) systems to maximize the Weighted Sum-Power (WSP) subjective to Signal-to-Interference-plus-Noise Ratio (SINR) constraints \cite{Xu2014}. Motivated by this, \cite{Krikidis2014} investigated fundamental transceiver modules, scheduling schemes, and interference management for SWIPT systems. However, \cite{Boshkovska2015} pointed out that the Radio Frequency-to-Direct Current (RF-to-DC) conversion efficiency depends on the harvester input power level. The authors also suggested a curve-fitting based parametric harvester model and proposed an iterative resource allocation algorithm. From another perspective, \cite{Trotter2009,Boaventura2011} demonstrated that multisine waveform is more suitable for WPT as it outperforms single tone in both operation range and RF-to-DC efficiency. \cite{Clerckx2016a} derived a tractable nonlinear harvester model based on the Taylor expansion of diode I-V characteristics, then implemented waveform design for WPT. Simulation and experiments demonstrated the benefit of modelling rectifier nonlinearity in system design \cite{Kim2019,Kim2019a}. The work was extended to SWIPT in \cite{Clerckx2018} where a superposition of modulated information waveform and multisine power waveform was employed to enlarge the R-E region. In contrast, \cite{Park2018} suggested an adaptive dual-mode SWIPT, which alternates between single-tone transmission that exploits conventional modulation for high-rate applications and multisine transmission that encodes the information in the Peak-to-Average Ratio (PAPR) for power-demanding applications. Modulation and coding schemes for SWIPT also receive much attention. By assuming On-Off-Keying (OOK) where bit \num{1} carries energy, \cite{Hu2019} compared unary and Run-Length-Limited (RLL) code in terms of rate versus battery overflow/underflow probability. It suggested that WPT should be activated only at constellation points with a large offset. Also, a learning approach \cite{Varasteh2019c} demonstrated that the offset of the power symbol is positively correlated to the harvester energy constraint, while the optimal information symbols are symmetrically distributed around the origin. It confirmed that the superposed waveform is effective to enlarge R-E region when considering rectifier nonlinearity. As for the network design, \cite{Liu2016} proposed a cooperative SWIPT Non-Orthogonal Multiple Access (NOMA) protocol with three user selection schemes such that the strong user assists the EH of the weak user. SWIPT based on Rate Splitting (RS) technique was also explored in \cite{Mao2019}.
		\end{subsection}


		\begin{subsection}{Intelligent Reflecting Surface}
			Intelligent Reflecting Surface (IRS) adapts the wireless channel to increase spectrum and energy efficiency. In practice, an IRS consists of multiple individual reflecting elements that adjust the amplitude and phase of the incident signal through passive beamforming. Different from relay and backscatter, IRS assists the primary transmission using fully passive components, thus consumes less power with no additional thermal noise but is limited to frequency-flat (FF) reflection. Although Frequency-Selective Surface (FSS) has received much attention for wideband communications, it is different from IRS as active FSS requires RF-chains \cite{Kim2006,Xu2014} while passive FSS has fixed physical characteristics thus not adjustable \cite{Anwar2018}.

			Inspired by the advance of real-time reconfigurable metamaterials in \cite{Cui2014}, \cite{Liaskos2018} introduced a programmable metasurface that steers or polarizes the EM wave at specific frequency to mitigate signal attenuation. At the same time, \cite{Tan2018} constructed an adjustable reflect array that ensures reliable millimeter-wave (mmWave) communication based on a beam-searching algorithm to reduce indoor signal blockage. Motivated by this, \cite{Wu2018,Wu2019} introduced an IRS-assisted MISO system and proposed a beamforming algorithm that jointly optimizes the precoder at the Access Point (AP) and the phase shifts at the IRS to maximize Signal-to-Noise Ratio (SNR). The active and passive beamforming problem was extended to the discrete phase shift case \cite{Wu2019a} and the multiuser case \cite{Guo2019a}. The application of IRS in interference cancellation and secure communication are also investigated in \cite{Zhou2020,Cui2019}. In \cite{Nadeem2019}, channel estimation for Time-Division Duplex (TDD) systems was carried through a two-stage Minimum Mean Squared Error (MMSE)-based protocol that sequentially estimates the cascaded channel of each IRS element with the others switched off. Starting from the impedance equation, \cite{Abeywickrama2019} investigated the influence of phase shift on the reflection amplitude and proposed a parametric IRS model via curve fitting. Recent research also explored the opportunity of integrating IRS with Orthogonal Frequency-Division Multiplexing (OFDM) systems. \cite{Yang2019} exploited spatial correlation to reduce estimation overhead and design complexity by assuming adjacent elements share a common reflection coefficient. On top of this, group-based OFDM channel estimation was investigated in \cite{Zheng2019}. By adjusting IRS over time slots, \cite{Yang2020} introduced artificial diversity within coherence time and investigated resource allocation and IRS configuration per Resource Block (RB). Real-time high-definition video transmission was performed over a prototype constructed with Positive Intrinsic-Negative (PIN) diodes, which demonstrated the feasibility and benefit of IRS at GHz and mmWave frequency \cite{Dai2020}.
		\end{subsection}


		\begin{subsection}{IRS-aided SWIPT}
			The effective channel enhancement and low power consumption of IRS are expected to bring more opportunities to SWIPT. Supposing energy interference and linear harvester model, \cite{Wu2019b} proved that at most one energy beam is required to maximize the WSP subject to SINR constraints. The fairness issue was then considered in \cite{Tang2019}, which maximize the minimum output power with the assumption of perfect energy interference cancellation. \cite{Wu2019c} proposed a novel penalty-based algorithm, whose inner layer employs Block Coordinate Descent (BCD) method to update transmit precoders and IRS phase shifts while the outer layer updates the penalty coefficients. It demonstrated that Line-of-Sight (LoS) links can boost the harvested power, as the rank-deficient channels are highly correlated such that a single energy stream can satisfy the energy constraints of all ERs. In \cite{Pan2019a}, the authors proposed low-complexity BCD algorithms to maximize the Weighted Sum-Rate (WSR) of Information Receivers (IRs) subject to the power constraint at Energy Receivers (ERs). However, most existing IRS-assisted SWIPT papers focused on narrow-band transmission for separated IRs and ERs based on an oversimplified linear harvester model.
		\end{subsection}


		\begin{subsection}{Objective and Methodology}
			In this paper, we study an IRS-aided downlink MISO SWIPT system where the IRS assists the information and energy transmission of a single user. A multicarrier unmodulated power waveform (deterministic multisine) is superposed to a multicarrier modulated information waveform (e.g. OFDM) to boost the energy transfer efficiency without creating additional interference. The transmit waveform, IRS phase shift and receive splitting ratio are jointly optimized to maximize the R-E tradeoff. Different from previous research, this paper focus on multicarrier IRS-SWIPT and investigates the fundamental impact of harvester nonlinearity on passive beamforming design. The R-E region characterization problem is transformed into multiple current maximization problems subject to different rate constraints. To reduce the design complexity, we propose an Alternating Optimization (AO) algorithm that updates the channel and transceiver iteratively based on Semidefinite Relaxation (SDR) and Geometric Programming (GP) techniques. Numerical results showed that SDR is tight and the proposed algorithm can find a stationary point for all tested channel realizations. We demonstrate that dedicated power waveform can boost the energy transmission efficiency such that TS and PS are preferred at low SNR and high SNR, respectively. Also, IRS brings a significant channel amplification and R-E enhancement especially when deployed next to the transmitter or the receiver. Finally, the proposed adaptive IRS design outperformed the benchmark schemes for broadband transmission, and the optimal IRS can be approximated in closed form for narrowband SISO transmission.

			\textit{Organization:} The rest of this paper is organized as follows. Section~\ref{se:system_model} introduces the signal, channel, decoder, harvester, and R-E tradeoff models of the IRS-aided SWIPT system. Section~\ref{se:problem_formulation} tackles the waveform, active and passive beamforming optimization. Section~\ref{se:performance_evaluation} presents simulation results to evaluate the proposed design. Section~[TODO] concludes the paper.

			\textit{Notations:} Scalars are denoted by italic letters, vectors are denoted by bold lower-case letters, and matrices are denoted by bold upper-case letters. $\mathbb{C}^{x \times y}$ denotes the subspace spanned by complex $x \times y$ matrices. $\Re\{\cdot\}$ and $\Im\{\cdot\}$ stand for the real and imaginary part of a complex number or variable, respectively. $(\cdot)^*$, $(\cdot)^T$ and $(\cdot)^H$ represent the conjugate, transpose, and conjugate transpose operators, respectively. $\mathcal{A}\{\cdot\}$ extracts the DC component of a signal, and $\mathcal{E}_X\{\cdot\}$ takes the expectation over the distribution of the random variable $X$ ($X$ may be omitted for simplicity). For a scalar $x$, $\lvert{x}\rvert$ denotes its absolute value. For a vector $\boldsymbol{x}$, $\lVert{\boldsymbol{x}}\rVert$ refers to its Euclidean norm, $\arg(\boldsymbol{x})$ refers to its argument vector, and $\mathrm{diag}(\boldsymbol{x})$ refers to a square diagonal matrix with the elements of $\boldsymbol{x}$ on the main diagonal. For a general matrix $\boldsymbol{M}$, $\mathrm{rank}(\boldsymbol{M})$ denotes it rank. For a square matrix $S$, $\mathrm{Tr}(\boldsymbol{S})$ denotes its trace, and $\boldsymbol{S} \succeq 0$ means that $\boldsymbol{S}$ is positive semi-definite. The distribution of a Circularly Symmetric Complex Gaussian (CSCG) random vector with mean $\boldsymbol{\mu}$ and covariance matrix $\boldsymbol{\Sigma}$ is denoted by $\mathcal{CN}(\boldsymbol{\mu},\boldsymbol{\Sigma})$ where $\sim$ stands for "distributed as". We also denote $(\cdot)^{\star}$ and $(\cdot)^{(i)}$ as stationary solution and variable value at iteration $i$, respectively.
		\end{subsection}
	\end{section}


	\begin{section}{System Model}\label{se:system_model}
		\begin{figure}[!t]
			\centering
			\def\svgwidth{\columnwidth}
			\import{assets/}{system.pdf_tex}
			\caption{An IRS-aided OFDM SWIPT system}
			\label{fi:system}
		\end{figure}

		As shown in Fig.~\ref{fi:system}, we consider an IRS-aided SWIPT system where a $M$-antenna AP delivers information and power simultaneously, through a $L$-reflector IRS, to a single-antenna user over $N$ orthogonal evenly-spaced subbands with center frequency $f_n$ ($n=1,\dots,N$). Perfect Channel State Information (CSI) with negligible training overhead is assumed to explore the upper bound of the proposed design. A quasi-static block fading channel model is considered for all links, and we focus on one particular block where the channels are approximately unchanged. Two practical co-located receiver architectures are compared in terms of R-E region. Specifically, TS divides each time slot into orthogonal data and energy slots and performs a time sharing between WPT and Wireless Information Transfer (WIT). In comparison, PS splits the received signal into individual ID and EH streams such that the splitting ratio $\rho$ is coupled with waveform and IRS design. Perfect synchronization is assumed among the three parties in both scenarios, and signals reflected by IRS for two and more times are omitted. We also assume the noise is too small to harvest.


		\begin{subsection}{Transmit Signal}
			Denote $\tilde{x}_{I,n}(t)$ as the information symbol transmitted over subband $n$, which follows a capacity-achieving i.i.d. CSCG distribution with zero mean and unit variance, namely $\tilde{x}_{I,n}\sim\mathcal{CN}(0,1)$. The superposed transmit signal on antenna $m$ ($m=1,\dots,M$) at time $t$ is
			\begin{equation}\label{eq:x_m}
				x_m(t)=\Re\left\{\sum_{n=1}^N\left({w_{I,n,m}\tilde{x}_{I,n}(t)}+w_{P,n,m}\right){e^{j2{\pi}{f_n}{t}}}\right\}
			\end{equation}
			where $w_{I/P,n,m}$ denotes the weight on the information and power signal transmitted by antenna $m$ at subband $n$. Define $\boldsymbol{w}_{I/P,n}=[w_{I/P,n,1},\dots,w_{I/P,n,M}]^T \in \mathbb{C}^{M \times 1}$ by stacking up weights across all antennas. Therefore, the transmit information and power signals write as
			\begin{align}
				\boldsymbol{x}_{I}(t) &= \Re{\left\{\sum_{n=1}^N\boldsymbol{w}_{I,n}\tilde{x}_{I,n}(t){e^{j2{\pi}{f_n}{t}}}\right\}}\label{eq:x_I}\\
				\boldsymbol{x}_{P}(t) &= \Re{\left\{\sum_{n=1}^N\boldsymbol{w}_{P,n}{e^{j2{\pi}{f_n}{t}}}\right\}}\label{eq:x_P}
			\end{align}
		\end{subsection}


		\begin{subsection}{Composite Channel}
			At subband $n$, denote the AP-user direct channel as $\boldsymbol{h}_{D,n}^H \in \mathbb{C}^{1 \times M}$, AP-IRS incident channel as $\boldsymbol{H}_{I,n} \in \mathbb{C}^{L \times M}$, and IRS-user reflective channel as $\boldsymbol{h}_{R,n}^H \in \mathbb{C}^{1 \times L}$. At the IRS, element $l$ ($l=1,\dots,L$) redistributes the incoming signal by adjusting the reflection amplitude $\gamma_l \in [0,1]$ and phase shift $\theta_l \in [0,2\pi)$ \footnote{To investigate the performance upper bound of IRS, we suppose the reflection coefficient is maximized $\gamma_l=1 \ \forall l$ while the phase shift is a continuous variable over $[0,2\pi)$.}. On top of this, the IRS matrix collects the reflection coefficients onto the main diagonal entries as $\boldsymbol{\Theta} = \mathrm{diag}(\gamma_1 e^{j \theta_1}, \dots, \gamma_L e^{j \theta_L}) \in \mathbb{C}^{L \times L}$. The extra link introduced by IRS can be modeled as a concatenation of the AP-IRS incident channel, IRS reflection, and IRS-user reflective channel. On top of this, the total composite channel is obtained by superposing the IRS-aided extra channel to the AP-user direct channel as
			\begin{equation}\label{eq:h_n}
				\boldsymbol{h}_{n}^H = \boldsymbol{h}_{D,n}^H + \boldsymbol{h}_{R,n}^H \boldsymbol{\Theta} \boldsymbol{H}_{I,n} = \boldsymbol{h}_{D,n}^H + \boldsymbol{\phi}^H \boldsymbol{V}_{n}
			\end{equation}
			where $\boldsymbol{\phi}=[\gamma_1 e^{j \theta_1}, \dots, \gamma_L e^{j \theta_L}]^H \in \mathbb{C}^{L \times 1}$ and $\boldsymbol{V}_{n}=\mathrm{diag}(\boldsymbol{h}_{R,n}^H)\boldsymbol{H}_{I,n} \in \mathbb{C}^{L \times M}$. Note the conjugate transpose in the notation of $\boldsymbol{\phi}$ makes its entries the complex conjugate of the diagonal entries of $\boldsymbol{\Theta}$.
		\end{subsection}


		\begin{subsection}{Receive Signal}
			At the single-antenna receiver, the total received signal $y(t)=y_I(t)+y_P(t)$ captures the contribution of information and power components over $N$ subbands, where
			\begin{align}\label{eq:y_IP}
				y_{I}(t) & = \Re\left\{\sum_{n=1}^N{\boldsymbol{h}_{n}^H}{\boldsymbol{w}_{I,n}\tilde{x}_{I,n}(t)}{e^{j2{\pi}{f_n}{t}}}\right\}\\
				y_{P}(t) & = \Re\left\{\sum_{n=1}^N{\boldsymbol{h}_{n}^H}\boldsymbol{w}_{P,n}{e^{j2{\pi}{f_n}{t}}}\right\}
			\end{align}
		\end{subsection}


		\begin{subsection}{Information Decoder}
			A major benefit of the superposed waveform is that the determined power waveform creates no interference to the information waveform. Therefore, the achievable rate writes as
			\begin{equation}\label{eq:R}
				R(\boldsymbol{\phi},\boldsymbol{w}_I,\rho) = \sum_{n=1}^N{\log_2\left(1+\frac{(1-\rho)\lvert \boldsymbol{h}_{n}^H\boldsymbol{w}_{I,n} \rvert^2}{\sigma_n^2}\right)}
			\end{equation}
			where $\rho$ is the power splitting ratio for the energy harvester, $\sigma_n^2$ is the variance of the total noise (RF-band and RF-to-baseband conversion) on tone $n$. Rate \ref{eq:R} is achievable with either waveform cancellation or translated demodulation \cite{Clerckx2018b}.
		\end{subsection}


		\begin{subsection}{Energy Harvester}
			\begin{figure}[!t]
				\centering
				\noindent
				\begin{minipage}[b]{0.5\linewidth}
					\centering
					\subfloat[Antenna equivalent circuit\label{ci:antenna_equivalent_circuit}]{
						\resizebox{0.9\linewidth}{!}{
							\begin{circuitikz}[transform shape]
							\draw (0,0) to [sV=$v_s$] (0,2);
							\draw (0,2) to [short] (1,2);
							\draw (1,2) to [R=$R_\text{ant}$] (3,2);
							\draw (3,2) to [short] (4,2);
							\draw (4,2) to [R=$R_\text{in}$, v<=$v_{\text{in}}$] (4,0);
							\draw (4,0) to [short] (2,0) node[ground](GND){};
							\draw (2,0) to [short] (0,0);
						\end{circuitikz}
						}
					}
				\end{minipage}%
				\begin{minipage}[b]{0.5\linewidth}
					\centering
					\subfloat[A single diode rectifier\label{ci:single_diode_rectifier}]{
						\resizebox{0.9\linewidth}{!}{
							\begin{circuitikz}[transform shape]
							\draw (0,0) to [sV=$v_\text{in}$] (0,2);
							\draw (0,2) to [short] (0.25,2);
							\draw (0.25,2) to [D, v<=$v_d$, i=$i_d$] (2.25,2);
							\draw (2.25,2) to [short, -*] (2.5,2);
							\draw (2.5,2) to [C, l_=$C$, -*] (2.5,0);
							\draw (2.5,2) to [short] (4,2);
							\draw (4,2) to [R=$R_L$, v<=$v_{\text{out}}$, i=$i_{\text{out}}$] (4,0);
							\draw (4,0) to [short] (2,0) node[ground](GND){};
							\draw (2,0) to [short] (0,0);
						\end{circuitikz}
						}
					}
				\end{minipage}
				\caption{Rectenna circuits}
			\end{figure}

			In this section, we briefly revisit a tractable nonlinear rectenna model that relates the harvester output DC current to the received waveform \cite{Clerckx2016a,Clerckx2018b}. Fig.~\ref{ci:antenna_equivalent_circuit} illustrates the equivalent circuit of a lossless antenna, where the incoming signal creates an voltage source $v_s(t)$ and the antenna has an impedance $R_{\text{ant}}$. Let $R_{\text{in}}$ be the total input impedance of the rectifier and matching network, and we assume the voltage across matching network is negligible. When perfectly matched ($R_{\text{in}}=R_{\text{ant}}$), the rectifier input voltage is $v_{\text{in}}(t)=y(t)\sqrt{\rho R_{\text{ant}}}$.

			Rectifiers consist of nonlinear components as diode and capacitor to produce DC output and store energy \cite{Hagerty2004,Pinuela2013}. Consider a simplified rectifier in Fig.~\ref{ci:single_diode_rectifier} where a single series diode is followed by a low-pass filter with a parallel load. Denote $i_s$ as the reverse bias saturation current, $n'$ as the diode ideality factor, $v_t$ as the thermal voltage, $v_d(t)=v_{\text{in}}(t)-v_{\text{out}}(t)$ as the voltage across the diode where $v_{\text{out}}(t)$ is the output voltage across the load. A Taylor expansion of the diode characteristic equation $i_d(t)=i_s(e^{v_d(t)/n' v_t}-1)$ around a quiescent operating point $a$ writes as $i_d(t)=\sum_{i=0}^{\infty}k_i'(v_d(t)-a)^i$, where $k_0'=i_s(e^{a/n' v_t}-1)$ and $k_i'=i_se^{a/n'v_t}/i!(n'v_t)^i$ for $i=1,\dots,\infty$. Note that this small-signal expansion model is only valid for the non-linear operation region, and the I-V relationship would be linear if the diode behavior is dominated by the load \cite{Clerckx2016a}. Also, an ideal low-pass filter with steady-state response can provide a constant $v_{\text{out}}$ that depends on the peak of $v_{\text{in}}(t)$ \cite{Curty2005}. Therefore, a proper choice of the operating voltage drop is $a=\mathcal{E}\{v_d(t)\}=-v_{\text{out}}$ such that
			\begin{equation}\label{eq:i_d}
				i_d(t)=\sum_{i=0}^{\infty}k_i'\rho^{i/2}R_{\text{ant}}^{i/2}y(t)^i
			\end{equation}

			By discarding the non-DC components, taking expectation over symbol distribution, and truncating \ref{eq:i_d} to the $n_0$-th order, we approximate the average output DC current for a given channel as
			\begin{equation}\label{eq:i_out}
				i_{\text{out}}(t)=\mathcal{A}\{i_d(t)\}\approx\sum_{i=0}^{\infty}{k_i'}{\rho^{i/2}}{R_{\text{ant}}^{i/2}}\mathcal{E}\left\{{\mathcal{A}\left\{y(t)^i\right\}}\right\}
			\end{equation}

			With the assumption of evenly spaced frequencies, it holds that $\mathcal{A}\left\{y(t)^i\right\}=0$ for odd $i$ thus the related terms has no contribution to DC output. However, $k_i'$ is still a function of $i_{\text{out}}$, and \cite{Clerckx2016a} proved that maximizing a truncated $i_{\text{out}}$ is equivalent to maximizing a monotonic function
			\begin{equation}\label{eq:z}
				z(\boldsymbol{\phi},\boldsymbol{w}_I,\boldsymbol{w}_P,\rho)=\sum_{i\,\text{even},i\ge2}^{n_0}{k_i}{\rho^{i/2}}{R_{\text{ant}}^{i/2}}{\mathcal{E}\left\{\mathcal{A}\left\{y(t)^i\right\}\right\}}
			\end{equation}
			where $k_i=i_s/i!(nv_t)^i$. It can be observed that the traditional linear harvester model, where the output DC power equals the sum of the power harvested on each frequency, is a special case of \ref{eq:z} with $n_0=2$. However, due to the coupling among different frequencies, some high-order terms also cancel out non-DC components thus contribute to the output DC power. In other words, even terms with $i \ge 4$ account for the nonlinear behavior of the diode. For simplicity, we let $\beta_2={k_2}{R_{\text{ant}}}$, $\beta_4={k_4}{R_{\text{ant}}^2}$ and choose $n_0=4$ to investigate fundamental nonlinearity. Note that $\mathcal{E}\left\{\lvert\tilde{x}_{I,n}\rvert^2\right\}=1$ but $\mathcal{E}\left\{\lvert\tilde{x}_{I,n}\rvert^4\right\}=2$, which can be interpreted as a modulation gain on the nonlinear terms of the output DC current.

			Inspired by \cite{Huang2017}, we stack up channel and waveform vectors over all subbands as $\boldsymbol{h}=[\boldsymbol{h}_1^H,\dots,\boldsymbol{h}_N^H]^H \in \mathbb{C}^{MN \times 1}$, $\boldsymbol{w}_{I/P}=[\boldsymbol{w}_{I/P,1}^H,\dots,\boldsymbol{w}_{I/P,N}^H]^H \in \mathbb{C}^{MN \times 1}$. Moreover, let $\boldsymbol{W}_{I/P,n}$ keep the $n$-th ($n=-N+1,\dots,N-1$) block diagonal of $\boldsymbol{W}_{I/P}=\boldsymbol{w}_{I/P}\boldsymbol{w}_{I/P}^H$ and null the remaining entries, where the blocks are of size $M \times M$. On top of this, $z$ is reduced to \ref{eq:z_expand} and the corresponding DC terms are expressed in \ref{eq:y_I2} -- \ref{eq:y_P4}.
			\begin{figure*}[b]
				\hrule
				\begin{align}
					z(\boldsymbol{\phi},\boldsymbol{w}_I,\boldsymbol{w}_P,\rho)
					& = \beta_2\rho\Bigl(\mathcal{E}\left\{\mathcal{A}\left\{y_{I}^2(t)\right\}\right\}+\mathcal{A}\left\{y_{P}^2(t)\right\}\Bigr)+\beta_4\rho^2\Bigl(\mathcal{E}\left\{\mathcal{A}\left\{y_{I}^4(t)\right\}\right\}+\mathcal{A}\left\{y_{P}^4(t)\right\}+6\mathcal{E}\left\{\mathcal{A}\left\{y_{I}^2(t)\right\}\right\}\mathcal{A}\left\{y_{P}^2(t)\right\}\Bigr)\label{eq:z_expand}\\
					\mathcal{E}\left\{\mathcal{A}\left\{y_{I}^2(t)\right\}\right\}
					& = \frac{1}{2}\sum_{n=1}^N{(\boldsymbol{h}_{n}^H\boldsymbol{w}_{I,n})(\boldsymbol{h}_{n}^H\boldsymbol{w}_{I,n})^H} = \frac{1}{2}\boldsymbol{h}^H\boldsymbol{W}_{I,0}\boldsymbol{h}\label{eq:y_I2}\\
					\mathcal{E}\left\{\mathcal{A}\left\{y_{I}^4(t)\right\}\right\}
					& = \frac{3}{4}\left(\sum_{n=1}^N{(\boldsymbol{h}_{n}^H\boldsymbol{w}_{I,n})(\boldsymbol{h}_{n}^H\boldsymbol{w}_{I,n})^H}\right)^2 = \frac{3}{4}(\boldsymbol{h}^H\boldsymbol{W}_{I,0}\boldsymbol{h})^2\label{eq:y_I4}
				\end{align}
				\begin{align}
					\mathcal{A}\left\{y_{P}^2(t)\right\}
					& = \frac{1}{2}\sum_{n=1}^N{(\boldsymbol{h}_{n}^H\boldsymbol{w}_{P,n})(\boldsymbol{h}_{n}^H\boldsymbol{w}_{P,n})^H} = \frac{1}{2}\boldsymbol{h}^H\boldsymbol{W}_{P,0}\boldsymbol{h}\label{eq:y_P2}\\
					\mathcal{A}\left\{y_{P}^4(t)\right\}
					& = \frac{3}{8}\sum_{\substack{{n_1},{n_2},{n_3},{n_4}\\{n_1}+{n_2}={n_3}+{n_4}}}{(\boldsymbol{h}_{{n_1}}^H\boldsymbol{w}_{P,{n_1}})(\boldsymbol{h}_{{n_2}}^H\boldsymbol{w}_{P,{n_2}})(\boldsymbol{h}_{{n_3}}^H\boldsymbol{w}_{P,{n_3}})^H(\boldsymbol{h}_{{n_4}}^H\boldsymbol{w}_{P,{n_4}})^H} = \frac{3}{8}\sum_{n=-N+1}^{N-1}(\boldsymbol{h}^H\boldsymbol{W}_{P,n}\boldsymbol{h})(\boldsymbol{h}^H\boldsymbol{W}_{P,n}\boldsymbol{h})^H\label{eq:y_P4}
				\end{align}
			\end{figure*}
		\end{subsection}

		\begin{subsection}{Rate-Energy Region}
			Define the achievable R-E region as
			\begin{align}
				C_{R_{\text{ID}}-I_{\text{EH}}}(P)
				&\triangleq \biggl\{(R_{\text{ID}}, I_{\text{EH}}): R_{\text{ID}} \le R, I_{\text{EH}} \le z,\nonumber\\
				&\quad \frac{1}{2}\left(\lVert{\boldsymbol{w}_I}\rVert^2+\lVert{\boldsymbol{w}_P}\rVert^2\right) \le P\biggr\}
			\end{align}
			where $P$ is the average transmit power budget and the coefficient \num{1/2} converts the peak power of sine waves to the average power.
		\end{subsection}
	\end{section}


	\begin{section}{Problem Formulation}\label{se:problem_formulation}
		We characterize the R-E region through multiple current maximization problems subject to transmit power, IRS magnitude, and different rate constraints
		\begin{maxi!}
			{\boldsymbol{\phi},\boldsymbol{w}_I,\boldsymbol{w}_P,\rho}{z(\boldsymbol{\phi},\boldsymbol{w}_I,\boldsymbol{w}_P,\rho)}{\label{op:original}}{\label{ob:original}}
			\addConstraint{\frac{1}{2}\left(\lVert{\boldsymbol{w}_I}\rVert^2+\lVert{\boldsymbol{w}_P}\rVert^2\right)\le{P}}\label{co:original_power}
			\addConstraint{R(\boldsymbol{\phi},\boldsymbol{w}_I,\rho) \ge \bar{R}}\label{co:original_rate}
			\addConstraint{\lvert{\phi_l}\rvert=1, \quad l=1,\dots,L}\label{co:original_modulus}
			\addConstraint{0 \le \rho \le 1}
		\end{maxi!}
		Problem~\ref{op:original} is intricate due to coupled variables involved in non-convex objective function \ref{ob:original} and rate constraint \ref{co:original_rate}. To reduce the design complexity, we propose an suboptimal AO algorithm that iteratively updates the IRS phase shift and transmit waveform plus the receive splitting ratio until convergence.


		\begin{subsection}{IRS Phase Shift}
			In this section, the IRS phase shift $\boldsymbol{\phi}$ is optimized for any given waveform $\boldsymbol{w}_{I/P}$ and splitting ratio $\rho$. We observe that
			\begin{align}
				\lvert \boldsymbol{h}_{n}^H\boldsymbol{w}_{I,n} \rvert^2
				& = \boldsymbol{w}_{I,n}^H\boldsymbol{h}_n\boldsymbol{h}_n^H\boldsymbol{w}_{I,n}\nonumber\\
				& = \boldsymbol{w}_{I,n}^H(\boldsymbol{h}_{D,n}+\boldsymbol{V}_n^H\boldsymbol{\phi})(\boldsymbol{h}_{D,n}^H+\boldsymbol{\phi}^H\boldsymbol{V}_n)\boldsymbol{w}_{I,n}\nonumber\\
				& = \boldsymbol{w}_{I,n}^H\boldsymbol{M}_n^H\boldsymbol{\Phi}\boldsymbol{M}_n\boldsymbol{w}_{I,n}\nonumber\\
				& = \mathrm{Tr}(\boldsymbol{M}_n\boldsymbol{w}_{I,n}\boldsymbol{w}_{I,n}^H\boldsymbol{M}_n^H\boldsymbol{\Phi})\nonumber\\
				& = \mathrm{Tr}(\boldsymbol{C}_n\boldsymbol{\Phi})
			\end{align}
			where $t$ is an auxiliary variable with unit modulus, $\boldsymbol{M}_n=[\boldsymbol{V}_n^H, \boldsymbol{h}_{D,n}]^H \in \mathbb{C}^{(L+1) \times M}$, $\bar{\boldsymbol{\phi}}=[\boldsymbol{\phi}^H, t]^H \in \mathbb{C}^{(L+1) \times 1}$, $\boldsymbol{\Phi}=\bar{\boldsymbol{\phi}}\bar{\boldsymbol{\phi}}^H \in \mathbb{C}^{(L+1) \times (L+1)}$, $\boldsymbol{C}_n = \boldsymbol{M}_n\boldsymbol{w}_{I,n}\boldsymbol{w}_{I,n}^H\boldsymbol{M}_n^H \in \mathbb{C}^{(L+1)\times(L+1)}$. Also, define $t_{I/P,n}$ ($n=-N+1,\dots,N-1$) as
			\begin{align}
				t_{I/P,n}
				& = \boldsymbol{h}^H\boldsymbol{W}_{I/P,n}\boldsymbol{h}\nonumber\\
				& = \mathrm{Tr}(\boldsymbol{h}\boldsymbol{h}^H\boldsymbol{W}_{I/P,n})\nonumber\\
				& = \mathrm{Tr}\left((\boldsymbol{h}_{D}+\boldsymbol{V}^H\boldsymbol{\phi})(\boldsymbol{h}_{D}^H+\boldsymbol{\phi}^H\boldsymbol{V})\boldsymbol{W}_{I/P,n}\right)\nonumber\\
				& = \mathrm{Tr}(\boldsymbol{M}^H\boldsymbol{\Phi}\boldsymbol{M}\boldsymbol{W}_{I/P,n})\nonumber\\
				& = \mathrm{Tr}(\boldsymbol{M}\boldsymbol{W}_{I/P,n}\boldsymbol{M}^H\boldsymbol{\Phi})\nonumber\\
				& = \mathrm{Tr}(\boldsymbol{C}_{I/P,n}\boldsymbol{\Phi})
			\end{align}
			where $\boldsymbol{V}=[\boldsymbol{V}_1,\dots,\boldsymbol{V}_N] \in \mathbb{C}^{L \times MN}$, $\boldsymbol{M}=[\boldsymbol{V}^H, \boldsymbol{h}_{D}]^H \in \mathbb{C}^{(L+1) \times MN}$, $\boldsymbol{C}_{I/P,n}=\boldsymbol{M}\boldsymbol{W}_{I/P,n}\boldsymbol{M}^H \in \mathbb{C}^{(L+1)\times(L+1)}$. Therefore, the rate and objective expressions rewrite as
			\begin{align}
				R(\boldsymbol{\Phi})
				& = \sum_{n}{\log_2\left(1+\frac{(1-\rho)\mathrm{Tr}(\boldsymbol{C}_n\boldsymbol{\Phi})}{\sigma_n^2}\right)}\label{eq:R_irs}\\
				z(\boldsymbol{\Phi})
				& = \frac{1}{2}{\beta_2}{\rho}(t_{I,0}+t_{P,0})\nonumber\\
				& \quad + \frac{3}{8}{\beta_4}{\rho^2} \left(2t_{I,0}^2 + \sum_{n=-N+1}^{N-1}{t_{P,n}t_{P,n}^*}\right)\nonumber\\
				& \quad + \frac{3}{2}{\beta_4}{\rho^2}t_{I,0}t_{P,0}\label{eq:z_irs}
			\end{align}

			To maximize non-concave expression \ref{eq:z_irs}, we propose a Successive Convex Approximation (SCA) algorithm that approximate the second-order terms by first-order Taylor expansion. Based on the variables optimized at iteration $i - 1$, the local approximation at iteration $i$ suggests \cite{Adali2010}
			\begin{align}
				(t_{I,0}^{(i)})^2
				& \ge 2 t_{I,0}^{(i)}t_{I,0}^{(i-1)} - (t_{I,0}^{(i-1)})^2\label{eq:taylor_1}\\
				t_{P,n}^{(i)} (t_{P,n}^{(i)})^*
				& \ge 2 \Re\left\{t_{P,n}^{(i)} (t_{P,n}^{(i-1)})^*\right\} - t_{P,n}^{(i-1)} (t_{P,n}^{(i-1)})^*\label{eq:taylor_2}\\
				t_{I,0}^{(i)} t_{P,0}^{(i)}
				& = \frac{1}{4}(t_{I,0}^{(i)} + t_{P,0}^{(i)})^2 - \frac{1}{4}(t_{I,0}^{(i)} - t_{P,0}^{(i)})^2\nonumber\\
				& \ge \frac{1}{2}(t_{I,0}^{(i)} + t_{P,0}^{(i)})(t_{I,0}^{(i-1)} + t_{P,0}^{(i-1)})\nonumber\\
				& \quad - \frac{1}{4}(t_{I,0}^{(i-1)} + t_{P,0}^{(i-1)})^2 - \frac{1}{4}(t_{I,0}^{(i)} - t_{P,0}^{(i)})^2\label{eq:taylor_3}
			\end{align}
			which provide lower bounds to the corresponding terms in \ref{eq:z_irs}. At iteration $i$, the approximated objection function $\tilde{z}(\boldsymbol{\Phi}^{(i)})$ is detailed in \ref{eq:z_irs_approx}.
			\begin{figure*}[b]
				\hrule
				\begin{equation}\label{eq:z_irs_approx}
					\begin{split}
						\tilde{z}(\boldsymbol{\Phi}^{(i)})
						& = \frac{1}{2}{\beta_2}{\rho}(t_{I,0}^{(i)}+t_{P,0}^{(i)})\\
						& \quad + \frac{3}{8}{\beta_4}{\rho^2} \left(4 (t_{I,0}^{(i)})(t_{I,0}^{(i-1)}) - 2 (t_{I,0}^{(i-1)})^2 + \sum_{n=-N+1}^{N-1}{2 \Re\left\{t_{P,n}^{(i)} (t_{P,n}^{(i-1)})^*\right\} - t_{P,n}^{(i-1)} (t_{P,n}^{(i-1)})^*}\right)\\
						& \quad + \frac{3}{2}{\beta_4}{\rho^2} \left(\frac{1}{2}(t_{I,0}^{(i)} + t_{P,0}^{(i)})(t_{I,0}^{(i-1)} + t_{P,0}^{(i-1)}) - \frac{1}{4}(t_{I,0}^{(i-1)} + t_{P,0}^{(i-1)})^2 - \frac{1}{4}(t_{I,0}^{(i)} - t_{P,0}^{(i)})^2\right)
					\end{split}
				\end{equation}
			\end{figure*}
			Hence, problem~\ref{op:original} is transformed to
			\begin{maxi!}
				{\boldsymbol{\Phi}}{\tilde{z}(\boldsymbol{\Phi})}{\label{op:irs}}{\label{ob:irs}}
				\addConstraint{R(\boldsymbol{\Phi}) \ge \bar{R}}\label{co:irs_rate}
				\addConstraint{\boldsymbol{\Phi}_{l,l}=1, \quad l=1,\dots,L+1}\label{co:irs_modulus}
				\addConstraint{\boldsymbol{\Phi}\succeq{0}}
				\addConstraint{\mathrm{rank}(\boldsymbol{\Phi})=1\label{co:irs_rank}}
			\end{maxi!}

			Problem~\ref{op:irs} is not a standard Semidefinite Programming (SDP) due to the rate constraint \ref{co:irs_rate}. If we relax the rank constraint \ref{co:irs_rank} to formulate a convex problem, there is no guarantee that the optimal rank-\num{1} solution $\bar{\boldsymbol{\phi}}^{\star}$ extracted from $\boldsymbol{\Phi}^{\star}$ is a stationary point of the original problem~\ref{op:original}. In Section~\ref{se:performance_evaluation}, we numerically show that $\boldsymbol{\Phi}^{\star}$ is rank-\num{1} for all tested channel realizations so that the performance loss is insignificant. A related version of problem~\ref{op:irs} can be solved using existing optimization tools such as CVX \cite{Grant2013}.

			When $\boldsymbol{\Phi}^{\star}$ is rank-\num{1}, the optimal phase shift vector $\bar{\boldsymbol{\phi}}^\star$ can be obtained by Eigenvalue Decomposition (EVD). Otherwise, a suboptimal solution can be extracted via Gaussian randomization method \cite{Huang2010}. Specifically, we perform EVD $\boldsymbol{\Phi}^{\star}=\boldsymbol{U}\boldsymbol{\Sigma}\boldsymbol{U}^H$, generate $Q$ CSCG random vectors $\boldsymbol{r}_q \sim \mathcal{CN}(\boldsymbol{0},\boldsymbol{I}_{L+1}),\ q=1,\dots,Q$, construct the corresponding candidates $\bar{\boldsymbol{\phi}}_q=e^{j\arg\left(\boldsymbol{U}\boldsymbol{\Sigma}^{1/2}\boldsymbol{r}_q\right)}$, and choose the one that maximizes the objective function \ref{ob:irs}. Finally, the phase shift is retrieved by $\theta_l=\arg(\phi_l^\star/\phi_{L+1}^\star), \ l=1,\dots,L$. The algorithm for phase shift optimization is summarized in Algorithm~\ref{al:irs}.
			\begin{algorithm}[!t]
				\caption{SCA: IRS Phase Shift}
				\label{al:irs}
				\begin{algorithmic}[1]
					\State \textbf{input} $\beta_2,\beta_4,\boldsymbol{h}_{D,n},\boldsymbol{H}_{I,n},\boldsymbol{h}_{R,n},\boldsymbol{w}_I,\boldsymbol{w}_P,\rho,\sigma_n,\bar{R},Q,\epsilon$
					\State Construct $\boldsymbol{M},\boldsymbol{M}_n,\boldsymbol{C}_{n}$ for $n=1,\dots,N$, $\boldsymbol{C}_{I/P,n}$ for $n=-N+1,\dots,N-1$
					\State \textbf{initialize} $i \gets 0,\boldsymbol{\Phi}^{(0)},t_{I/P,n}^{(0)}$ for $n=-N+1,\dots,N-1$
					\Repeat
						\State $i \gets i + 1$
						\State Obtain $\boldsymbol{\Phi}^{(i)}, t_{I/P,n}^{(i)}$ by solving problem~\ref{op:irs}
						\State Compute $z^{(i)}$ by \ref{eq:z_irs}
					\Until $\lvert z^{(i)}-z^{(i-1)} \rvert \le \epsilon$
					\State Set $\boldsymbol{\Phi}^{\star}=\boldsymbol{\Phi}^{(i)}$
					\If{$\mathrm{rank}(\boldsymbol{\Phi}^{\star})=1$}
						\State Obtain $\bar{\boldsymbol{\phi}}^\star$ by EVD, $\boldsymbol{\Phi}^{\star}=\bar{\boldsymbol{\phi}}^\star(\bar{\boldsymbol{\phi}}^\star)^H$
					\Else
						\State Obtain $\boldsymbol{U},\boldsymbol{\Sigma}$ by EVD, $\boldsymbol{\Phi}^{\star}=\boldsymbol{U}\boldsymbol{\Sigma}\boldsymbol{U}^H$
						\State Generate $\boldsymbol{r}_q \sim \mathcal{CN}(\boldsymbol{0},\boldsymbol{I}_{L+1})$, $q=1,\dots,Q$
						\State Construct $\bar{\boldsymbol{\phi}}_q=e^{j\arg\left(\boldsymbol{U}\boldsymbol{\Sigma}^{1/2}\boldsymbol{r}_q\right)}$, $\boldsymbol{\Phi}_q=\bar{\boldsymbol{\phi}}_q\bar{\boldsymbol{\phi}}_q^H$
						\State Set $q^{\star}=\arg\max_q{z(\boldsymbol{\Phi}_q)}$, $\bar{\boldsymbol{\phi}}^\star=\bar{\boldsymbol{\phi}}_{q^{\star}}$
					\EndIf
					\State Set $\theta_l^\star=\arg(\phi_l^\star/\phi_{L+1}^\star), \ l=1,\dots,L$, construct $\boldsymbol{\phi}^{\star}$
					\State \textbf{output} $\boldsymbol{\phi}^{\star}$
				\end{algorithmic}
			\end{algorithm}
		\end{subsection}


		\begin{subsection}{Waveform and Splitting Ratio}
			Next, we jointly optimize both information and power waveforms $\boldsymbol{w}_{I/P}$ together with splitting ratio $\rho$ for any given IRS phase shift $\boldsymbol{\phi}$. As pointed out in \cite{Clerckx2018b}, the waveform design in frequency and spatial domain can be decoupled without performance loss, and the optimal spatial weight is given by Maximum-Ratio Transmission (MRT) beamformer
			\begin{equation}\label{eq:w_IP}
				\boldsymbol{w}_{I/P,n}=s_{I/P,n}\frac{\boldsymbol{h}_n}{\lVert{\boldsymbol{h}_n}\rVert}
			\end{equation}
			That is to say, for single-user MISO SWIPT, it is only necessary to determine the amplitudes $s_{I/P,n}$ at different tones. Hence, the original waveform optimization with $2MN$ complex variables is converted into a power allocation problem with $2N$ nonnegative real variables. Let $\boldsymbol{s}_{I/P}=[s_{I/P,1},\dots,s_{I/P,N}]^T \in \mathbb{C}^{N \times 1}$. At subband $n$, the effective channel gain is given by $\lVert{\boldsymbol{h}_n}\rVert$, and the power allocated to the modulated and unmodulated waveform are given by $s_{I,n}^2$ and $s_{P,n}^2$, respectively. With such an active beamformer selection, we have $\boldsymbol{h}_n^H\boldsymbol{w}_{I,n}=\lvert{\boldsymbol{h}_n^H\boldsymbol{w}_{I,n}}\rvert=\lVert{\boldsymbol{h}_n}\rVert s_{I,n}$ such that the rate and objective expressions further reduces to \ref{eq:R_waveform} and \ref{eq:z_waveform}.
			\begin{equation}\label{eq:R_waveform}
				R(\boldsymbol{s}_I,\rho) = \log_2\left(\prod_{n=1}^N\biggl(1+\frac{(1-\rho)\lVert{\boldsymbol{h}_n}\rVert^2 s_{I,n}^2}{\sigma_n^2}\biggr)\right)
			\end{equation}
			\begin{figure*}[b]
				\hrule
				\begin{equation}\label{eq:z_waveform}
					\begin{split}
						z(\boldsymbol{s}_I,\boldsymbol{s}_P,\rho)
						& = \frac{1}{2}{\beta_2}{\rho} \sum_{n=1}^N \lVert{\boldsymbol{h}_n}\rVert^2(s_{I,n}^2+s_{P,n}^2)\\
						& \quad + \frac{3}{8}{\beta_4}{\rho^2} \left( 2\sum_{n_1,n_2} \prod_{j=1}^2 \lVert{\boldsymbol{h}_{n_j}}\rVert^2 s_{I,{n_j}}^2 + \sum_{\substack{{n_1},{n_2},{n_3},{n_4}\\{n_1}+{n_2}={n_3}+{n_4}}} \prod_{j=1}^4 \lVert{\boldsymbol{h}_{n_j}}\rVert s_{P,{n_j}} \right)\\
						& \quad + \frac{3}{2}{\beta_4}{\rho^2} \left( \sum_{n_1,n_2} \lVert{\boldsymbol{h}_{n_1}}\rVert^2 s_{I,{n_1}}^2 \lVert{\boldsymbol{h}_{n_2}}\rVert^2 s_{P,{n_2}}^2 \right)
					\end{split}
				\end{equation}
			\end{figure*}
			Therefore, problem~\ref{op:original} is reduced to an amplitude optimization issue
			\begin{maxi!}
				{\boldsymbol{s}_I,\boldsymbol{s}_P,\rho}{z(\boldsymbol{s}_I,\boldsymbol{s}_P,\rho)}{\label{op:waveform}}{}
				\addConstraint{\frac{1}{2}\left(\lVert{\boldsymbol{s}_I}\rVert^2+\lVert{\boldsymbol{s}_P}\rVert^2\right)\le{P}}
				\addConstraint{R(\boldsymbol{s}_I,\rho) \ge \bar{R}}
			\end{maxi!}
			Since problem~\ref{op:waveform} involves the production of nonnegative real variables, we introduce auxiliary variables $t',\bar{\rho}$ and transform it into a reversed GP
			\begin{mini!}
				{\boldsymbol{s}_I,\boldsymbol{s}_P,\rho,\bar{\rho},t'}{\frac{1}{t'}}{\label{op:waveform_rgp}}{}
				\addConstraint{\frac{1}{2}\left(\lVert{\boldsymbol{s}_I}\rVert^2+\lVert{\boldsymbol{s}_P}\rVert^2\right) \le P}\label{co:waveform_power}
				\addConstraint{\frac{t'}{z(\boldsymbol{s}_I,\boldsymbol{s}_P,\rho)} \le 1}\label{co:waveform_objective}
				\addConstraint{\frac{2^{\bar{R}}}{\prod_{n=1}^N \left(1+{\bar{\rho}\lVert{\boldsymbol{h}_n}\rVert^2 s_{I,n}^2}/{\sigma_n^2}\right)} \le 1}\label{co:waveform_rate}
				\addConstraint{\rho + \bar{\rho} \le 1}
			\end{mini!}
			The denominators of \ref{co:waveform_objective}, \ref{co:waveform_rate} are posynomials \cite{Boyd2007}, and we further rewrite them as
			\begin{align}
				z(\boldsymbol{s}_I,\boldsymbol{s}_P,\rho)&=\sum_{m_P}{g_{m_P}(\boldsymbol{s}_I,\boldsymbol{s}_P,\rho)}\label{eq:posynomial_objective}\\
				1+\frac{\bar{\rho}\lVert{\boldsymbol{h}_n}\rVert^2 s_{I,n}^2}{\sigma_n^2}&=\sum_{m_{I,n}}g_{m_{I,n}}(s_{I,n},\bar{\rho})\label{eq:posynomial_rate}
			\end{align}
			where $m_P,m_{I,n}$ are the number of monomials in the corresponding posynomials (obviously $m_{I,n}=2$). Following \cite{Clerckx2018b,Chiang2005}, we upper bound posynomials \ref{eq:posynomial_objective} and \ref{eq:posynomial_rate} by Arithmetic Mean-Geometric Mean (AM-GM) inequality such that problem~\ref{op:waveform_rgp} reduces to
			\begin{mini}
				{\boldsymbol{s}_I,\boldsymbol{s}_P,\rho,\bar{\rho},t'}{\frac{1}{t'}}{\label{op:waveform_gp}}{}
				\addConstraint{\frac{1}{2}\left(\lVert{\boldsymbol{s}_I}\rVert^2+\lVert{\boldsymbol{s}_P}\rVert^2\right)\le{P}}
				\addConstraint{{t'}\prod_{m_P}{\left(\frac{g_{{m_P}}(\boldsymbol{s}_I,\boldsymbol{s}_P,\rho)}{\gamma_{{m_P}}}\right)^{-\gamma_{{m_P}}}}\le{1}}
				\addConstraint{2^{\bar{R}}\prod_{n}\prod_{m_{I,n}}\left(\frac{g_{m_{I,n}}(s_{I,n},\bar{\rho})}{\gamma_{m_{I,n}}}\right)^{-\gamma_{m_{I,n}}}\le{1}}
				\addConstraint{\rho + \bar{\rho} \le 1}
			\end{mini}
			where $\gamma_{m_P},\gamma_{m_{I,n}} \ge 0$, $\sum_{m_P}\gamma_{m_P}=\sum_{m_{I,n}}\gamma_{m_{I,n}}=1$. The tightness of the AM-GM inequality depends on $\{\gamma_{m_P},\gamma_{m_{I,n}}\}$ that require successive update. As suggested in \cite{Clerckx2018b}, a feasible choice at iteration $i$ is
			\begin{align}
				\gamma_{m_P}^{(i)} & = \frac{g_{m_P}(\boldsymbol{s}_I^{(i-1)},\boldsymbol{s}_P^{(i-1)},\rho^{(i-1)})}{z(\boldsymbol{s}_I^{(i-1)},\boldsymbol{s}_P^{(i-1)},\rho^{(i-1)})}\label{eq:gamma_P}\\
				\gamma_{m_{I,n}}^{(i)} & = \frac{g_{m_{I,n}}(s_{I,n}^{(i-1)},\bar{\rho}^{(i-1)})}{1+{\bar{\rho}^{(i-1)}\lVert{\boldsymbol{h}_n}\rVert^2 (s_{I,n}^{(i-1)})^2}/{\sigma_n^2}}\label{eq:gamma_I}
			\end{align}

			Problem~\ref{op:waveform_gp} can be solved using existing optimization tools such as CVX \cite{Grant2013}. $\boldsymbol{s}_I,\boldsymbol{s}_P,\rho$ are updated iteratively until convergence. The GP algorithm is summarized in Algorithm~\ref{al:waveform}.
			\begin{algorithm}[!t]
				\caption{GP: Waveform and Splitting Ratio}
				\label{al:waveform}
				\begin{algorithmic}[1]
					\State \textbf{Input} $\beta_2,\beta_4,\boldsymbol{h},P,\sigma_n,\bar{R},\epsilon$
					\State \textbf{Initialize} $i \gets 0$, $\boldsymbol{s}_{I/P}^{(0)}$, $\rho^{(0)}$
					\Repeat
						\State $i \gets i + 1$
						\State Update $\{\gamma_{m_P}^{(i)},\gamma_{m_{I,n}}^{(i)}\}$ by \ref{eq:gamma_P}, \ref{eq:gamma_I}
						\State Obtain $\boldsymbol{s}_{I/P}^{(i)},\rho^{(i)}$ by solving problem~\ref{op:waveform_gp}
						\State Compute $z^{(i)}$ by \ref{eq:z_waveform}
					\Until $\lvert z^{(i)} - z^{(i-1)} \rvert \le \epsilon$
					\State Set $\boldsymbol{s}_{I/P}^{\star}=\boldsymbol{s}_{I/P}^{(i)}$, $\rho^{\star}=\rho^{(i)}$, retrieve $\boldsymbol{w}_{I/P}^{\star}$ by \ref{eq:w_IP}
					\State \textbf{Output} $\boldsymbol{w}_{I/P}^{\star}, \rho^{\star}$
				\end{algorithmic}
			\end{algorithm}
		\end{subsection}


		\begin{subsection}{Alternating Optimization}
			For any direct, incident and reflective channels, we iteratively update the IRS phase shift by Algorithm~\ref{al:irs} and update the transmit waveform with receive splitting ratio by Algorithm~\ref{al:waveform} until convergence. The alternating algorithm is summarized in Algorithm~\ref{al:alternating}.
			\begin{algorithm}[!t]
				\caption{AO: Waveform, Active and Passive Beamforming}
				\label{al:alternating}
				\begin{algorithmic}[1]
					\State \textbf{Input} $\beta_2,\beta_4,\boldsymbol{h}_{D,n},\boldsymbol{H}_{I,n},\boldsymbol{h}_{R,n},P,\sigma_n,\bar{R},Q,\epsilon$
					\State \textbf{Initialize} $i \gets 0$, $\boldsymbol{\phi}^{(0)},\boldsymbol{w}_{I/P}^{(0)},\rho^{(0)}$
					\Repeat
						\State $i \gets i + 1$
						\State Fix $\boldsymbol{w}_{I/P}^{(i-1)},\rho^{(i-1)}$ and obtain $\boldsymbol{\phi}^{(i)}$ by Algorithm~\ref{al:irs}
						\State Fix $\boldsymbol{\phi}^{(i)}$, update $\boldsymbol{h}_n^{(i)}$ by \ref{eq:h_n}, and obtain $\boldsymbol{w}_{I/P}^{(i)}, \rho^{(i)}$ by Algorithm~\ref{al:waveform}
						\State Compute $z^{(i)}$ by \ref{eq:z_irs}
					\Until $\lvert z^{(i)} - z^{(i-1)} \rvert \le \epsilon$
					\State \textbf{Output} $\boldsymbol{\phi}^{\star}, \boldsymbol{w}_{I/P}^{\star}, \rho^{\star}$
				\end{algorithmic}
			\end{algorithm}
		\end{subsection}


		\begin{subsection}{Convergence}
			The proposed alternating algorithm iteratively employs SCA-based IRS phase shift Algorithm~\ref{al:irs} and GP-based waveform and splitting ratio Algorithm~\ref{al:waveform} until convergence.

			\begin{proposition}\label{pr:irs}
				For any feasible initial point, the proposed SCA-based Algorithm~\ref{al:irs} is guaranteed to converge to a stationary point of the IRS phase shift subproblem.
			\end{proposition}

			\begin{proof}\label{pf:irs}
				The objective function \ref{ob:irs} is non-decreasing over iterations because the solution of problem~\ref{op:irs} at iteration $i-1$ is still a feasible point at iteration $i$. Moreover, the sequence $\{\tilde{z}(\boldsymbol{\Phi}^{(i)})\}_{i=1}^{\infty}$ is bounded above due to the unit-modulus constraint \ref{co:irs_modulus}. Thus, Algorithm~\ref{al:irs} is guaranteed to converge. To prove $\boldsymbol{\Phi}^{(i)}$ converge to the set of stationary points of IRS subproblem, we notice that the SCA-based Algorithm~\ref{al:irs} is indeed an inner approximation algorithm \cite{Marks1978}, since $\tilde{z}(\boldsymbol{\Phi}) \le z(\boldsymbol{\Phi})$, $\partial\tilde{z}(\boldsymbol{\Phi}^{(i)})/\partial\boldsymbol{\Phi}=\partial z(\boldsymbol{\Phi}^{(i)})/\partial\boldsymbol{\Phi}$ and the approximation \ref{eq:taylor_1} -- \ref{eq:taylor_3} are asymptotically tight as $i \to \infty$ \cite{Adali2010,Li2013}. Therefore, Algorithm~\ref{al:irs} converges to a stationary point.
			\end{proof}

			\begin{proposition}\label{pr:waveform}
				For any feasible initial point, the GP-based Algorithm~\ref{al:waveform} is guaranteed to converge to a stationary point of the waveform and splitting ratio subproblem.
			\end{proposition}

			\begin{proof}\label{pf:waveform}
				See \cite{Clerckx2016a,Clerckx2018b}.
			\end{proof}

			\begin{proposition}\label{pr:ao}
				Every limit point $(\boldsymbol{\phi}^{\star},\boldsymbol{w}_I^{\star},\boldsymbol{w}_P^{\star},\rho^{\star})$ of the proposed alternating algorithm is a stationary point of the original problem~\ref{op:original}.
			\end{proposition}

			\begin{proof}\label{pf:ao}
				The objective function \ref{ob:original} is non-decreasing over iterations of Algorithm~\ref{al:alternating}, which is also upper-bounded due to the unit-modulus constraint \ref{co:original_modulus} and the average transmit power constraint \ref{co:original_power}. Thus, Algorithm~\ref{al:alternating} is guaranteed to converge, namely the sequence $\{\boldsymbol{\phi}^{(i)},\boldsymbol{w}_I^{(i)},\boldsymbol{w}_P^{(i)},\rho^{(i)}\}$ generated by optimizing $\boldsymbol{\phi}$ and $\boldsymbol{w}_I,\boldsymbol{w}_P,\rho$ alternatively has limit points. As demonstrated in \cite{Grippo2000,Hong2016,Li2013a}, the solution is a stationary point of problem~\ref{op:original}.
			\end{proof}
		\end{subsection}
	\end{section}


	\begin{section}{Performance Evaluations}\label{se:performance_evaluation}
		\begin{figure}[!t]
			\centering
			\def\svgwidth{\columnwidth}
			\import{assets/}{layout.pdf_tex}
			\caption{System layout}
			\label{fi:layout}
		\end{figure}
		To evaluate the performance of the proposed IRS-aided SWIPT system, we characterize the average R-E regions under typical setups. Consider a large open space WiFi-like environment at a center frequency of \SI{5.18}{\GHz} with reference bandwidth $B=1\,\si{\MHz}$. As shown in Fig.~\ref{fi:layout}, we assume the IRS moves along a horizontal line parallel to the AP-user path and let $d_H$, $d_V$ be the horizontal and vertical distances from the AP to the IRS, respectively. We also denote $d_D$, $d_I$, $d_R$ as the length of direct, incident, reflective paths such that $d_I=\sqrt{d_H^2+d_V^2}$, $d_R=\sqrt{(d_D-d_H)^2+d_V^2}$, and choose $d_D=15\,\si{\meter}$, $d_V=2\,\si{\meter}$ and reference $d_H=2\,\si{\meter}$. The path loss and fading parameters are obtained from IEEE TGn channel model D \cite{Erceg2004}. Reference path loss is set to $L_0=-35\,\si{\dB}$ at $d_0=1\,\si{\meter}$. For NLoS channels, all taps are modelled as i.i.d. CSCG random variables. For LoS channels, the first tap, whose power is significantly larger, is in circular uniform distribution while the remaining taps are in i.i.d. CSCG distribution. The average sum-power of all taps is unit such that the multipath response is normalized. All channels are regarded as NLoS unless otherwise mentioned. We choose number of subbands $N=16$ as reference and assume no spatial correlation across all elements. Rectenna parameters are taken as $k_2=0.0034$, $k_4=0.3829$, $R_{\text{ant}}=50\,\si{\ohm}$. With \SI{0}{\dBi} transmit antenna gain, the average Effective Isotropic Radiated Power (EIRP) is fixed to $MP=-36\,\si{\dBm}$ while the reference average noise power is $\sigma_n=-40\,\si{\dBm}$ at all subbands. We also assume \SI{0}{\dBi} IRS element gain and \SI{0}{\dBi} receive antenna gain. For the algorithm, the tolerance is $\epsilon=10^{-8}$, the number of candidates in the Gaussian randomization method is $Q=10^{3}$, and the R-E region is averaged over \num{200} channel realizations. In the R-E boundary, the leftmost point corresponds to WPT ($\rho=1$) where power can be allocated simultaneously to modulated and unmodulated waveform to maximize the average output DC current. On the other hand, the rightmost point corresponds to WIT ($\rho=0$) where the solution coincides with the Water-Filling (WF) algorithm that allocates all power to modulated waveform only. For a fair comparison, the $x$-axis of the plots has been normalized to per-subband rate.

		We first evaluate the performance of Algorithm~\ref{al:irs} under SDR. It is demonstrated that $\boldsymbol{\Theta}^{\star}$ is rank-\num{1} for all tested channel realizations with different $M$, $N$ and $L$. Therefore, $\boldsymbol{\theta}^{\star}$ can be directly obtained through EVD and we claim Algorithm~\ref{al:irs} converges to stationary points of problem~\ref{op:irs} without performance loss.

		\begin{figure}[!t]
			\centering
			\begin{adjustbox}{width=\linewidth}
				% This file was created by matlab2tikz.
%
%The latest updates can be retrieved from
%  http://www.mathworks.com/matlabcentral/fileexchange/22022-matlab2tikz-matlab2tikz
%where you can also make suggestions and rate matlab2tikz.
%
\definecolor{mycolor1}{rgb}{0.00000,0.44700,0.74100}%
\definecolor{mycolor2}{rgb}{0.85000,0.32500,0.09800}%
\definecolor{mycolor3}{rgb}{0.92900,0.69400,0.12500}%
\definecolor{mycolor4}{rgb}{0.49400,0.18400,0.55600}%
\definecolor{mycolor5}{rgb}{0.46600,0.67400,0.18800}%
%
\begin{tikzpicture}

\begin{axis}[%
width=4.521in,
height=3.566in,
at={(0.758in,0.481in)},
scale only axis,
xmin=0,
xlabel style={font=\color{white!15!black}},
xlabel={Average subband rate [bps/Hz]},
ymin=0,
ylabel style={font=\color{white!15!black}},
ylabel={Average output DC current [$\mu$A]},
axis background/.style={fill=white},
xmajorgrids,
ymajorgrids,
legend style={legend cell align=left, align=left, draw=white!15!black},
title style={font=\huge}, label style={font=\huge}, ticklabel style={font=\LARGE}, legend style={font=\LARGE}
]
\addplot [color=mycolor1, line width=2.0pt, mark=o, mark options={solid, mycolor1}]
  table[row sep=crcr]{%
9.46890267721896	0\\
9.14238879177698	2.06058892429301\\
8.8158749063743	4.26233503033247\\
8.48936102103128	6.33434645723537\\
8.16284713576798	8.16203030500679\\
7.83633325049239	9.71340734520188\\
7.50981936519989	10.9979195896073\\
7.1833054801987	12.0437545037854\\
6.85679159543229	12.8858479202057\\
6.53027771088155	13.557864299478\\
6.20376382466356	14.0905361206965\\
5.87724994099725	14.5114629726546\\
5.55073605390809	14.8436343133811\\
5.22422216911569	15.1051275349131\\
4.89770828521446	15.3112913075434\\
4.57119440091622	15.4740937813443\\
4.24468051635362	15.6029138072054\\
3.91816663393357	15.7040724609974\\
3.59165275164935	15.7842779112795\\
3.26513886843575	15.8473393325449\\
2.93862498713484	15.8970884319648\\
2.61211109980931	15.9362144970615\\
2.28559721653566	15.9670672610703\\
1.95908333557116	15.9914297630608\\
1.63256944981795	16.010446861626\\
1.30605557043835	16.0256363371831\\
0.979541703399084	16.0378385940564\\
0.65302781586059	16.0476946537369\\
0.326513968968278	16.0555691460589\\
0.000300638177805991	16.0619797346475\\
};
\addlegendentry{PS: $N = 1$}

\addplot [color=mycolor2, dashed, line width=2.0pt, mark=+, mark options={solid, mycolor2}]
  table[row sep=crcr]{%
8.46556304247255	0\\
8.17364707541111	1.82385107040865\\
7.88173110848213	3.78913601625201\\
7.58981514165366	5.68297817958176\\
7.29789917610517	7.40501954046902\\
7.00598321353032	8.9185135185827\\
6.71406726249954	10.2213610980912\\
6.42215133138764	11.3285469094081\\
6.13023544713028	12.2632260220374\\
5.83831962441125	13.0505889579397\\
5.54640391917556	13.7122976270942\\
5.25448828939039	14.2701107233491\\
4.96257285501171	14.7424255310282\\
4.67065770525496	15.1441823529951\\
4.37874287690642	15.487370979838\\
4.0868284316557	15.783870933659\\
3.79491497698335	16.0417614274795\\
3.50300251429766	16.2687778694633\\
3.21109256848399	16.4703484713191\\
2.9191874952923	16.6533865331907\\
2.62728937702051	16.8231731275352\\
2.33540602288393	16.9814817545436\\
2.04353656468252	17.1266281273727\\
1.75165530956211	17.2505954647495\\
1.45973896037783	17.3459520119533\\
1.16779034798659	17.4121247065342\\
0.875834585201547	17.4564119654346\\
0.583883883846118	17.4849200230768\\
0.291939038599665	17.5036816157706\\
7.71447095493525e-05	17.5161439271368\\
};
\addlegendentry{PS: $N = 2$}

\addplot [color=mycolor3, dotted, line width=2.0pt, mark=square, mark options={solid, mycolor3}]
  table[row sep=crcr]{%
7.46195803733478	0\\
7.20464913931057	1.64439023154687\\
6.94734024158824	3.44067169357655\\
6.69003134395344	5.21002424649614\\
6.43272244771776	6.85922551780511\\
6.17541355548316	8.34656602771163\\
5.91810467382765	9.66050230975805\\
5.66079581227179	10.8064714830068\\
5.40348699001214	11.7983392495901\\
5.14617821670923	12.6546662987696\\
4.88886951071646	13.3931853251685\\
4.63156089260085	14.0310038046515\\
4.37425240947079	14.5835265842281\\
4.11694402171808	15.0643528848531\\
3.8596357246911	15.4591014541796\\
3.60232785476079	15.8369941860022\\
3.34502070327645	16.1921224919211\\
3.08771484217803	16.5520876076447\\
2.8304091329605	16.9055291385717\\
2.57310508003245	17.2489707386037\\
2.31580226183127	17.5805219587305\\
2.05850115193541	17.9077907600047\\
1.80120443231233	18.2173139899078\\
1.54390849309841	18.5211321909545\\
1.28661198095323	18.8139178210308\\
1.02931376292115	19.0881615189512\\
0.772003206077341	19.3499997137566\\
0.51467033359545	19.593248312206\\
0.257337355367512	19.8392852366835\\
0.000107333323007389	20.3225736728487\\
};
\addlegendentry{PS: $N = 4$}

\addplot [color=mycolor4, dashdotted, line width=2.0pt, mark=x, mark options={solid, mycolor4}]
  table[row sep=crcr]{%
6.468480282759	0\\
6.24542923828008	1.43193464061974\\
6.02237819420643	3.00693278395837\\
5.79932715022391	4.58707183755718\\
5.57627610709771	6.09364469377994\\
5.3532250681748	7.4858611001229\\
5.1301740354772	8.74671302393403\\
4.9071230262896	9.87357631966346\\
4.68407203725829	10.8725748719672\\
4.46102111872044	11.7129130407899\\
4.23797022964661	12.5157913834373\\
4.01491947433255	13.2867190241435\\
3.79186886067379	14.0997259064165\\
3.56881854629962	14.9912508908831\\
3.34576824509665	15.9393566965622\\
3.12271865163752	16.9077354968867\\
2.89966916769362	17.881065196729\\
2.6766196684656	18.852407743946\\
2.45357100539822	19.8084432970183\\
2.2305219938759	20.7488886842733\\
2.00747416174855	21.6693557096402\\
1.78442648546261	22.5692074996784\\
1.56137800871321	23.4577025317854\\
1.33833052959572	24.3336002230737\\
1.1152813229072	25.202546974065\\
0.892233685196653	26.0776433896397\\
0.669189319216649	26.9829409170065\\
0.446148410985696	27.9598408401867\\
0.223106665476803	29.1106766925676\\
1.74471222037354e-05	31.5527795452716\\
};
\addlegendentry{PS: $N = 8$}

\addplot [color=mycolor5, line width=2.0pt, mark=triangle, mark options={solid, mycolor5}]
  table[row sep=crcr]{%
5.49562646641633	0\\
5.30612210476729	1.23746041784662\\
5.1166177445409	2.60120058943943\\
4.92711338383919	3.99097055885376\\
4.73760902399084	5.34441152908984\\
4.54810466888412	6.62495510900238\\
4.35860032295467	7.81342444016109\\
4.16909598823431	8.86853692644557\\
3.97959170299599	9.85513798409577\\
3.79008751064166	10.9283657303263\\
3.60058335828523	12.145395669226\\
3.41107955585456	13.5753761790259\\
3.2215760455884	15.138595296818\\
3.032072678146	16.7946529552345\\
2.84256981213309	18.5349855987174\\
2.6530662018636	20.3475920157419\\
2.46356321777605	22.2060738823148\\
2.27405987644601	24.0978723732673\\
2.08455601566542	26.0143291214322\\
1.89505322366167	27.9469475584398\\
1.70555052904291	29.8913389117734\\
1.51604849311684	31.8481400971468\\
1.32654598597251	33.8254406468065\\
1.13704489386409	35.8360300944003\\
0.947545105978283	37.896125945202\\
0.75804704071231	40.0375179695025\\
0.568551004200205	42.3178387457435\\
0.379060518273628	44.8571783284822\\
0.189564891479746	47.9711958997324\\
1.22728679936587e-06	54.9910262571505\\
};
\addlegendentry{PS: $N = 16$}

\addplot [color=red, dashed, line width=2.0pt, mark=triangle, mark options={solid, rotate=180, red}]
  table[row sep=crcr]{%
6.468480282759	0\\
5.79932715022391	4.58707183755718\\
5.57627610709771	6.09364469377994\\
5.3532250681748	7.4858611001229\\
5.1301740354772	8.74671302393403\\
4.9071230262896	9.87357631966346\\
4.68407203725829	10.8725748719672\\
1.74471222037354e-05	31.5527795452716\\
};
\addlegendentry{TS + PS: $N = 8$}

\addplot [color=black, dotted, line width=2.0pt, mark=o, mark options={solid, black}]
  table[row sep=crcr]{%
1.22728679936587e-06	54.9910262571505\\
5.49562646641633	0\\
};
\addlegendentry{TS: $N = 16$}

\end{axis}
\end{tikzpicture}%

			\end{adjustbox}
			\caption{Average R-E region versus the number of subbands}
			\label{fi:re_subband}
		\end{figure}

		\begin{figure}[!t]
			\centering
			\begin{adjustbox}{width=\linewidth}
				% This file was created by matlab2tikz.
%
%The latest updates can be retrieved from
%  http://www.mathworks.com/matlabcentral/fileexchange/22022-matlab2tikz-matlab2tikz
%where you can also make suggestions and rate matlab2tikz.
%
\definecolor{mycolor1}{rgb}{0.00000,0.44700,0.74100}%
\definecolor{mycolor2}{rgb}{0.85000,0.32500,0.09800}%
%
\begin{tikzpicture}

\begin{axis}[%
width=4.036in,
height=0.491in,
at={(0.677in,3.363in)},
scale only axis,
xmin=0,
xmax=2,
xtick={1},
ymin=0,
axis background/.style={fill=white},
xmajorgrids,
ymajorgrids,
legend style={legend cell align=left, align=left, draw=white!15!black},
title style={font=\huge}, label style={font=\huge}, ticklabel style={font=\LARGE}, legend style={font=\LARGE}
]
\addplot[ycomb, color=mycolor1, line width=2.0pt, mark=o, mark options={solid, mycolor1}] table[row sep=crcr] {%
1	2.821727013159\\
};
\addplot[forget plot, color=white!15!black, line width=2.0pt] table[row sep=crcr] {%
0	0\\
2	0\\
};
\addlegendentry{$s_I$}

\addplot[ycomb, color=mycolor2, line width=2.0pt, mark=x, mark options={solid, mycolor2}] table[row sep=crcr] {%
1	0.000259587806229199\\
};
\addplot[forget plot, color=white!15!black, line width=2.0pt] table[row sep=crcr] {%
0	0\\
2	0\\
};
\addlegendentry{$s_P$}

\end{axis}

\begin{axis}[%
width=4.036in,
height=0.491in,
at={(0.677in,2.637in)},
scale only axis,
xmin=0,
xmax=3,
xtick={1, 2},
ymin=0,
axis background/.style={fill=white},
xmajorgrids,
ymajorgrids,
title style={font=\huge}, label style={font=\huge}, ticklabel style={font=\LARGE}, legend style={font=\LARGE}
]
\addplot[ycomb, color=mycolor1, line width=2.0pt, mark=o, mark options={solid, mycolor1}, forget plot] table[row sep=crcr] {%
1	0.274209781976626\\
2	2.76985839517448\\
};
\addplot[forget plot, color=white!15!black, line width=2.0pt] table[row sep=crcr] {%
0	0\\
3	0\\
};
\addplot[ycomb, color=mycolor2, line width=2.0pt, mark=x, mark options={solid, mycolor2}, forget plot] table[row sep=crcr] {%
1	7.29255431750488e-06\\
2	7.41275007486677e-06\\
};
\addplot[forget plot, color=white!15!black, line width=2.0pt] table[row sep=crcr] {%
0	0\\
3	0\\
};
\end{axis}

\begin{axis}[%
width=4.036in,
height=0.491in,
at={(0.677in,1.911in)},
scale only axis,
xmin=0,
xmax=5,
xtick={1, 2, 3, 4},
ymin=0,
ylabel style={font=\color{white!15!black}},
ylabel={Waveform amplitude},
axis background/.style={fill=white},
xmajorgrids,
ymajorgrids,
title style={font=\huge}, label style={font=\huge}, ticklabel style={font=\LARGE}, legend style={font=\LARGE}
]
\addplot[ycomb, color=mycolor1, line width=2.0pt, mark=o, mark options={solid, mycolor1}, forget plot] table[row sep=crcr] {%
1	0.00906914399366857\\
2	0.0191780945021875\\
3	0.0728112091555182\\
4	1.15645756112447\\
};
\addplot[forget plot, color=white!15!black, line width=2.0pt] table[row sep=crcr] {%
0	0\\
5	0\\
};
\addplot[ycomb, color=mycolor2, line width=2.0pt, mark=x, mark options={solid, mycolor2}, forget plot] table[row sep=crcr] {%
1	0.65161979775128\\
2	0.797812683770198\\
3	0.949573699600353\\
4	1.29650400585771\\
};
\addplot[forget plot, color=white!15!black, line width=2.0pt] table[row sep=crcr] {%
0	0\\
5	0\\
};
\end{axis}

\begin{axis}[%
width=4.036in,
height=0.491in,
at={(0.677in,1.184in)},
scale only axis,
xmin=0,
xmax=9,
xtick={1, 2, 3, 4, 5, 6, 7, 8},
ymin=0,
axis background/.style={fill=white},
xmajorgrids,
ymajorgrids,
title style={font=\huge}, label style={font=\huge}, ticklabel style={font=\LARGE}, legend style={font=\LARGE}
]
\addplot[ycomb, color=mycolor1, line width=2.0pt, mark=o, mark options={solid, mycolor1}, forget plot] table[row sep=crcr] {%
1	0.00122290635412822\\
2	0.00162605944227885\\
3	0.00219881563728839\\
4	0.00308343653686168\\
5	0.00446835221941886\\
6	0.00730773383230549\\
7	0.018778542482244\\
8	0.156996809294451\\
};
\addplot[forget plot, color=white!15!black, line width=2.0pt] table[row sep=crcr] {%
0	0\\
9	0\\
};
\addplot[ycomb, color=mycolor2, line width=2.0pt, mark=x, mark options={solid, mycolor2}, forget plot] table[row sep=crcr] {%
1	0.604492899882458\\
2	0.741597181612563\\
3	0.847995036254347\\
4	0.93677567011772\\
5	1.00740091437702\\
6	1.06242371388309\\
7	1.10606359542287\\
8	1.16708516593263\\
};
\addplot[forget plot, color=white!15!black, line width=2.0pt] table[row sep=crcr] {%
0	0\\
9	0\\
};
\end{axis}

\begin{axis}[%
width=4.036in,
height=0.491in,
at={(0.677in,0.458in)},
scale only axis,
xmin=0,
xmax=17,
xtick={ 1,  2,  3,  4,  5,  6,  7,  8,  9, 10, 11, 12, 13, 14, 15, 16},
xlabel style={font=\color{white!15!black}},
xlabel={Sorted subband index},
ymin=0,
ytick={  0, 0.4, 0.8},
axis background/.style={fill=white},
xmajorgrids,
ymajorgrids,
title style={font=\huge}, label style={font=\huge}, ticklabel style={font=\LARGE}, legend style={font=\LARGE}
]
\addplot[ycomb, color=mycolor1, line width=2.0pt, mark=o, mark options={solid, mycolor1}, forget plot] table[row sep=crcr] {%
1	0.000441546073916646\\
2	0.000505313371521773\\
3	0.000579072140308781\\
4	0.000665955723611611\\
5	0.000770614004522113\\
6	0.000892853558400096\\
7	0.00104199146112475\\
8	0.00121812704068694\\
9	0.0014365060772557\\
10	0.00170548951516733\\
11	0.00204996483051139\\
12	0.00251037543116691\\
13	0.00316602847896381\\
14	0.00413663959831382\\
15	0.00569434862540457\\
16	0.00825748611983488\\
};
\addplot[forget plot, color=white!15!black, line width=2.0pt] table[row sep=crcr] {%
0	0\\
17	0\\
};
\addplot[ycomb, color=mycolor2, line width=2.0pt, mark=x, mark options={solid, mycolor2}, forget plot] table[row sep=crcr] {%
1	0.415766467786655\\
2	0.472429517994175\\
3	0.521046798847998\\
4	0.566100676905971\\
5	0.605416399739846\\
6	0.641331572087823\\
7	0.675743455874702\\
8	0.705629790631534\\
9	0.735011238747164\\
10	0.759950967519761\\
11	0.784031990499284\\
12	0.802981219380124\\
13	0.820186543633863\\
14	0.831767285739736\\
15	0.840123349700173\\
16	0.843558402787778\\
};
\addplot[forget plot, color=white!15!black, line width=2.0pt] table[row sep=crcr] {%
0	0\\
17	0\\
};
\end{axis}
\end{tikzpicture}%
			\end{adjustbox}
			\caption{Sorted waveform amplitude versus the number of subbands}
			\label{fi:waveform_subband}
		\end{figure}

		Fig.~\ref{fi:re_subband} illustrates the average R-E region versus the number of subband $N$. \textit{First}, it is observed that increasing $N$ reduces the per-subband rate but boosts the harvested energy. The reason is that although each subband receives a smaller proportion of the total power, more balanced terms are introduced to further amplify the output DC current, as suggested by the scaling laws in \cite{Clerckx2018b}. Waveform amplitude in Fig.~\ref{fi:waveform_subband} also confirmed that from the perspective of WPT, dedicated multisine waveform is unnecessary for a small $N$ but is required for a large $N$. As shown in \ref{eq:y_I4} and \ref{eq:y_P4}, the only difference of modulated and unmodulated waveform on $z$ exists in the fourth-order terms, where $\mathcal{E}\left\{\mathcal{A}\left\{y_{I}^4(t)\right\}\right\}$ has $N^2$ monomials with a modulation gain of \num{2} and $\mathcal{A}\left\{y_{P}^4(t)\right\}$ has $(2N^3+N)/3$ monomials without modulation gain. Therefore, superposed waveform enlarges the R-E region for a sufficiently large $N$ (typically no smaller than 4). However, an excessively large $N$ not only increases computation complexity but also operates out of the small-signal harvester model, thus become prohibitive. \textit{Second}, the R-E region is convex for $N = 2, 4$ and concave-convex for $N = 8, 16$. This has the consequence that PS outperforms TS for a small $N$ and is outperformed for a large $N$. When $N$ is in between, the optimal strategy is a combination of both, i.e. a time sharing between the WPT point and the tangent WIPT point obtained by PS. Compared with the linear harvester model that requires no dedicated power waveform and always prefer PS, the rectifier nonlinearity enlarges the R-E region by favouring a different waveform and transceiving strategy, both heavily depends on $N$.

		\begin{figure}[!t]
			\centering
			\begin{adjustbox}{width=\linewidth}
				% This file was created by matlab2tikz.
%
%The latest updates can be retrieved from
%  http://www.mathworks.com/matlabcentral/fileexchange/22022-matlab2tikz-matlab2tikz
%where you can also make suggestions and rate matlab2tikz.
%
\definecolor{mycolor1}{rgb}{0.00000,0.44700,0.74100}%
\definecolor{mycolor2}{rgb}{0.85000,0.32500,0.09800}%
\definecolor{mycolor3}{rgb}{0.92900,0.69400,0.12500}%
\definecolor{mycolor4}{rgb}{0.49400,0.18400,0.55600}%
\definecolor{mycolor5}{rgb}{0.46600,0.67400,0.18800}%
%
\begin{tikzpicture}

\begin{axis}[%
width=4.521in,
height=1.575in,
at={(0.758in,2.472in)},
scale only axis,
xmin=0,
xlabel style={font=\color{white!15!black}},
xlabel={Average subband rate [bps/Hz]},
ymin=0,
ylabel style={font=\color{white!15!black}},
ylabel={Average output DC current [$\mu$A]},
axis background/.style={fill=white},
xmajorgrids,
ymajorgrids,
legend style={legend cell align=left, align=left, draw=white!15!black}
]
\addplot [color=mycolor1, line width=2.0pt, mark=o, mark options={solid, mycolor1}]
  table[row sep=crcr]{%
0.622682642041422	0\\
0.601236423383352	0.442612037820587\\
0.579761711004782	0.857115028670277\\
0.558290719126105	1.3002840100234\\
0.536819839004198	1.77053678956013\\
0.515349132207132	2.26589396248944\\
0.493878132430448	2.78475076674697\\
0.472407376006214	3.32546234426105\\
0.450936449697703	3.88676999650049\\
0.429464854645822	4.46766792652547\\
0.407993518307919	5.06691103182337\\
0.386521924519791	5.68388153522306\\
0.365049859056882	6.31804286614888\\
0.343577584389047	6.96894384519356\\
0.322105442370751	7.63606951807862\\
0.300632550078712	8.31998885518326\\
0.279159464190666	9.02026203791522\\
0.257685931948121	9.73753129549601\\
0.236212085740386	10.4394986349111\\
0.214738021476891	11.1977253766711\\
0.193263559348426	11.988742494465\\
0.171789496067895	12.83583619407\\
0.15031607724049	13.9277662691858\\
0.128843316789918	15.3451122069557\\
0.107370359882918	17.1694401639499\\
0.0858975730487439	19.5391294761626\\
0.0644244830789251	22.6441712726074\\
0.0429508607439586	26.9675178724178\\
0.0214765471232306	33.6400632006213\\
5.06778216560452e-10	55.2498608411414\\
};
\addlegendentry{$\sigma_n = -20$ dBm}

\addplot [color=mycolor2, dashed, line width=2.0pt, mark=+, mark options={solid, mycolor2}]
  table[row sep=crcr]{%
2.56470642681832	0\\
2.47626831055351	0.691282640556407\\
2.387830186357	1.44513790626809\\
2.29939207008782	2.2409281055306\\
2.21095402675715	3.06083609866881\\
2.12251603799647	3.89091474290932\\
2.03407822232999	4.72035471652295\\
1.94564039571933	5.5409962569854\\
1.85720290865869	6.34677356338329\\
1.76876562935097	7.13339325386499\\
1.6803284894462	7.89803703165199\\
1.59189208049175	8.63909403158659\\
1.50345552402019	9.3448319100033\\
1.4150199229818	10.0387783384305\\
1.32658411352241	10.7710496769656\\
1.23814884948203	11.6032048905254\\
1.14971343224388	12.512619603015\\
1.06127777598699	13.5921667321839\\
0.972841563324578	14.785292588885\\
0.884404361061965	16.1443219805427\\
0.795965783137808	17.648354099216\\
0.707527837245262	19.3174955630911\\
0.619092170622039	21.1418375993355\\
0.530657657556759	23.1962248898976\\
0.442221593215359	25.5356769719498\\
0.353786526568767	28.2145276766198\\
0.265350622047181	31.3837489123374\\
0.176909832311022	35.3145862426546\\
0.0884629483033547	40.7306098421577\\
7.09458767637855e-09	55.2514897021657\\
};
\addlegendentry{$\sigma_n = -30$ dBm}

\addplot [color=mycolor3, dotted, line width=2.0pt, mark=square, mark options={solid, mycolor3}]
  table[row sep=crcr]{%
5.60300411213724	0\\
5.40979707331538	1.26843811177322\\
5.21659003536169	2.65385979750085\\
5.02338299708889	4.0602207494935\\
4.83017595911997	5.42751271379642\\
4.63696892252802	6.72039590229198\\
4.44376189075816	7.92020071651595\\
4.25055486625106	8.9950008677074\\
4.05734786428981	9.98060775158619\\
3.86414089721892	10.9968594302406\\
3.67093400701335	12.1648985760136\\
3.47772719142908	13.548935954263\\
3.28452047603366	15.079092340491\\
3.09131397743405	16.7194872271433\\
2.89810758096324	18.4460754171478\\
2.70490118297073	20.2459759098262\\
2.51169514328775	22.09261857596\\
2.31848908462228	23.9780130283661\\
2.12528259502642	25.8914433044044\\
1.93207676715409	27.8250153090131\\
1.73887165719365	29.7731852303823\\
1.54566619427866	31.7395746582437\\
1.35246024801155	33.7311369746514\\
1.15925512492905	35.7581935450606\\
0.966051654123267	37.8390494500372\\
0.772850338062731	40.0063224185348\\
0.579652258714572	42.3203089394301\\
0.386458772722115	44.9028452278983\\
0.193262507567315	48.0740503031957\\
8.46957263577546e-08	55.2522464484787\\
};
\addlegendentry{$\sigma_n = -40$ dBm}

\addplot [color=mycolor4, dashdotted, line width=2.0pt, mark=x, mark options={solid, mycolor4}]
  table[row sep=crcr]{%
8.89182888246906	0\\
8.58521409326477	1.96098834911378\\
8.27859930431421	4.05989105460428\\
7.97198451540705	6.06870755780236\\
7.66536972726048	7.88362863064171\\
7.35875494234336	9.43618959493692\\
7.05214016485443	10.8127072170366\\
6.74552540774476	12.4741684254563\\
6.43891067407054	14.5555443991645\\
6.13229598553153	16.8454799083811\\
5.82568129342707	19.2731113712525\\
5.51906660032758	21.7521854553867\\
5.21245185016244	24.2323824942928\\
4.90583701789151	26.6734795248149\\
4.59922211455772	29.0440750052004\\
4.29260724906173	31.3225893835949\\
3.98599239438153	33.494767742988\\
3.67937755072139	35.552209873297\\
3.37276277265554	37.4914932566724\\
3.06614802067162	39.3132279975345\\
2.75953331578265	41.0214965274521\\
2.45291862026744	42.6229668131923\\
2.14630391889746	44.1271804350843\\
1.83968928821506	45.5461924021602\\
1.53307479104111	46.8954585545062\\
1.22646079951217	48.1952726796818\\
0.91984920608438	49.475365369516\\
0.613244610257377	50.7900023468271\\
0.306658752243681	52.2704828331503\\
6.6181262262034e-07	55.2524060040588\\
};
\addlegendentry{$\sigma_n = -50$ dBm}

\addplot [color=mycolor5, line width=2.0pt, mark=triangle, mark options={solid, mycolor5}]
  table[row sep=crcr]{%
12.2103843782899	0\\
11.7893366410274	2.68351453934089\\
11.3682889039	5.45483815607794\\
10.947241166918	7.92794126529428\\
10.5261934307476	9.95702816188572\\
10.1051456973549	11.8429216334637\\
9.68409797010892	14.5286151230504\\
9.26305024885062	17.622066442443\\
8.84200252584905	20.885258126308\\
8.4209547804507	24.1687916303299\\
7.99990702893788	27.3673003036257\\
7.57885928223634	30.41127222453\\
7.15781153964667	33.2562718591345\\
6.7367638062682	35.8772207913038\\
6.31571607994176	38.2640213719196\\
5.89466835672578	40.4174521525468\\
5.47362063871286	42.3459255694841\\
5.05257293876591	44.062920439037\\
4.63152521066552	45.5848791625207\\
4.21047753365302	46.929914855262\\
3.78942990876459	48.116631980624\\
3.36838239815027	49.1634333799863\\
2.9473350693401	50.0883433431305\\
2.52628811920357	50.9088251018236\\
2.10524175679114	51.6419592760158\\
1.68419753993712	52.3045265139519\\
1.26315258863362	52.9150424715196\\
0.842103490569631	53.5003838164046\\
0.421065549028916	54.1153815172928\\
2.6080731753108e-06	55.2525883066671\\
};
\addlegendentry{$\sigma_n = -60$ dBm}

\end{axis}

\begin{axis}[%
width=4.521in,
height=1.575in,
at={(0.758in,0.481in)},
scale only axis,
xmin=0,
xlabel style={font=\color{white!15!black}},
xlabel={Average subband rate [bps/Hz]},
ymin=0,
ylabel style={font=\color{white!15!black}},
ylabel={Average power splitting ratio},
axis background/.style={fill=white},
xmajorgrids,
ymajorgrids
]
\addplot [color=mycolor1, line width=2.0pt, mark=o, mark options={solid, mycolor1}, forget plot]
  table[row sep=crcr]{%
0.622682642041422	2.22044604925031e-16\\
0.601236423383352	0.0546236495114467\\
0.579761711004782	0.0964545009725418\\
0.558290719126105	0.137498191229909\\
0.536819839004198	0.177873331395111\\
0.515349132207132	0.217566407350811\\
0.493878132430448	0.25659248076223\\
0.472407376006214	0.294979558835803\\
0.450936449697703	0.332729136590715\\
0.429464854645822	0.369849855983362\\
0.407993518307919	0.406348130696068\\
0.386521924519791	0.442241902837711\\
0.365049859056882	0.477534030986953\\
0.343577584389047	0.512240810836996\\
0.322105442370751	0.546395015564245\\
0.300632550078712	0.579983984873656\\
0.279159464190666	0.613049615192229\\
0.257685931948121	0.645590971437943\\
0.236212085740386	0.676568007560787\\
0.214738021476891	0.706164872353165\\
0.193263559348426	0.731106269948913\\
0.171789496067895	0.748040698737134\\
0.15031607724049	0.758931022693324\\
0.128843316789918	0.767716079839159\\
0.107370359882918	0.777604274930107\\
0.0858975730487439	0.791535823793534\\
0.0644244830789251	0.808031330391159\\
0.0429508607439586	0.83543509777309\\
0.0214765471232306	0.877998850731802\\
5.06778216560452e-10	0.999985032883765\\
};
\addplot [color=mycolor2, dashed, line width=2.0pt, mark=+, mark options={solid, mycolor2}, forget plot]
  table[row sep=crcr]{%
2.56470642681832	2.22044604925031e-16\\
2.47626831055351	0.0725237133154718\\
2.387830186357	0.139989893381437\\
2.29939207008782	0.203190010398952\\
2.21095402675715	0.262397568599277\\
2.12251603799647	0.317864920758931\\
2.03407822232999	0.369827729912416\\
1.94564039571933	0.418502741913097\\
1.85720290865869	0.464099928347048\\
1.76876562935097	0.506809146273129\\
1.6803284894462	0.546807630682464\\
1.59189208049175	0.584283049025273\\
1.50345552402019	0.618110623271806\\
1.4150199229818	0.647394330204554\\
1.32658411352241	0.670893774727408\\
1.23814884948203	0.690553072971344\\
1.14971343224388	0.702701986186805\\
1.06127777598699	0.716148089469128\\
0.972841563324578	0.727078931470642\\
0.884404361061965	0.739903954550748\\
0.795965783137808	0.753663476334433\\
0.707527837245262	0.769233707717023\\
0.619092170622039	0.781201543791352\\
0.530657657556759	0.794418338288306\\
0.442221593215359	0.81249848992407\\
0.353786526568767	0.82999498237396\\
0.265350622047181	0.851744742483584\\
0.176909832311022	0.877782883033547\\
0.0884629483033547	0.913510037776107\\
7.09458767637855e-09	0.999982799897391\\
};
\addplot [color=mycolor3, dotted, line width=2.0pt, mark=square, mark options={solid, mycolor3}, forget plot]
  table[row sep=crcr]{%
5.60300411213724	2.22044604925031e-16\\
5.40979707331538	0.127991916926748\\
5.21659003536169	0.239416990927143\\
5.02338299708889	0.336413625068893\\
4.83017595911997	0.420830440782178\\
4.63696892252802	0.494271888532176\\
4.44376189075816	0.558135082661057\\
4.25055486625106	0.613040094373272\\
4.05734786428981	0.656891364326641\\
3.86414089721892	0.685246304153997\\
3.67093400701335	0.700430855912646\\
3.47772719142908	0.713644190958766\\
3.28452047603366	0.72516085317073\\
3.09131397743405	0.737903773114186\\
2.89810758096324	0.749181879664\\
2.70490118297073	0.762366637344574\\
2.51169514328775	0.775294507005966\\
2.31848908462228	0.789172542623828\\
2.12528259502642	0.803664565393353\\
1.93207676715409	0.818173885976562\\
1.73887165719365	0.83208041597836\\
1.54566619427866	0.84556073679751\\
1.35246024801155	0.859374206159318\\
1.15925512492905	0.873428150035787\\
0.966051654123267	0.887865763572983\\
0.772850338062731	0.902725298727963\\
0.579652258714572	0.918377613608431\\
0.386458772722115	0.935537485994346\\
0.193262507567315	0.956079644597371\\
8.46957263577546e-08	0.999981676713924\\
};
\addplot [color=mycolor4, dashdotted, line width=2.0pt, mark=x, mark options={solid, mycolor4}, forget plot]
  table[row sep=crcr]{%
8.89182888246906	2.22044604925031e-16\\
8.58521409326477	0.191487432238718\\
8.27859930431421	0.345503736471672\\
7.97198451540705	0.46925839785151\\
7.66536972726048	0.568558598749526\\
7.35875494234336	0.647110730127443\\
7.05214016485443	0.696472565086636\\
6.74552540774476	0.712655672424247\\
6.43891067407054	0.727540366310054\\
6.13229598553153	0.740785667038446\\
5.82568129342707	0.75766832428581\\
5.51906660032758	0.775827335215796\\
5.21245185016244	0.794042717894206\\
4.90583701789151	0.812077054968388\\
4.59922211455772	0.829266540690191\\
4.29260724906173	0.845381261669418\\
3.98599239438153	0.860389789296263\\
3.67937755072139	0.874325290693399\\
3.37276277265554	0.887229794231026\\
3.06614802067162	0.899178257492019\\
2.75953331578265	0.910246607268167\\
2.45291862026744	0.920528559060485\\
2.14630391889746	0.930117604171488\\
1.83968928821506	0.939112975882499\\
1.53307479104111	0.947644054145816\\
1.22646079951217	0.955854596069092\\
0.91984920608438	0.9639479491882\\
0.613244610257377	0.972263740645756\\
0.306658752243681	0.981600147939692\\
6.6181262262034e-07	0.999983100248849\\
};
\addplot [color=mycolor5, line width=2.0pt, mark=triangle, mark options={solid, mycolor5}, forget plot]
  table[row sep=crcr]{%
12.2103843782899	2.22044604925031e-16\\
11.7893366410274	0.252525958601193\\
11.3682889039	0.439875951856179\\
10.947241166918	0.578499985739617\\
10.5261934307476	0.678980367517115\\
10.1051456973549	0.713937979894303\\
9.68409797010892	0.729747106425759\\
9.26305024885062	0.748420055249861\\
8.84200252584905	0.772587079654286\\
8.4209547804507	0.797784052008363\\
7.99990702893788	0.821673256514771\\
7.57885928223634	0.843493890712623\\
7.15781153964667	0.863107409000432\\
6.7367638062682	0.880575320498125\\
6.31571607994176	0.896043652119298\\
5.89466835672578	0.909677777423865\\
5.47362063871286	0.921666673422102\\
5.05257293876591	0.93219104083251\\
4.63152521066552	0.941415475340555\\
4.21047753365302	0.949511297321253\\
3.78942990876459	0.95662186335818\\
3.36838239815027	0.962881087104149\\
2.9473350693401	0.968411662140905\\
2.52628811920357	0.973327538539347\\
2.10524175679114	0.977735249322931\\
1.68419753993712	0.98174608824415\\
1.26315258863362	0.985478138623143\\
0.842103490569631	0.989097850258867\\
0.421065549028916	0.992935851267291\\
2.6080731753108e-06	0.999988512921205\\
};
\end{axis}
\end{tikzpicture}%
			\end{adjustbox}
			\caption{Average R-E region and splitting ratio versus the average noise power}
			\label{fi:re_noise}
		\end{figure}

		The influence of the average noise power on the average R-E region is investigated in Fig.~\ref{fi:re_noise}. We \textit{first} note that for a large number of subbands ($N=16$), the R-E region is approximately concave for a high noise level and approximately convex for a low noise level. Hence, TS is preferred at low SNR while PS is preferred at high SNR. This is because at a low SNR, the capacity-achieving WF algorithm tends to allocate more power to few strongest subbands. As the rate constraint $\bar{R}$ decreases, more subbands are activated to further boost the harvested energy via the coupling effect by rectifier nonlinearity. \textit{Second}, there exists a turning point in the R-E region especially for a small noise ($\sigma_n \le -40\,\si{\dBm}$). The reason is that when $\bar{R}$ departs slightly from the capacity, the algorithm mainly adjusts the splitting ratio $\rho$ rather than put more weight on the multisine waveform, since a small amplitude could be inefficient for energy maximization. On the other hand, as $\bar{R}$ further reduces, a modulated waveform with a very large $\rho$ could be outperformed by a superposed waveform with a smaller $\rho$, due to advantage the of multisine. The result highlights the necessity of joint optimization of waveform and splitting ratio.

		\begin{figure}[!t]
			\centering
			\begin{adjustbox}{width=\linewidth}
				% This file was created by matlab2tikz.
%
%The latest updates can be retrieved from
%  http://www.mathworks.com/matlabcentral/fileexchange/22022-matlab2tikz-matlab2tikz
%where you can also make suggestions and rate matlab2tikz.
%
\definecolor{mycolor1}{rgb}{0.00000,0.44700,0.74100}%
\definecolor{mycolor2}{rgb}{0.85000,0.32500,0.09800}%
\definecolor{mycolor3}{rgb}{0.92900,0.69400,0.12500}%
\definecolor{mycolor4}{rgb}{0.49400,0.18400,0.55600}%
\definecolor{mycolor5}{rgb}{0.46600,0.67400,0.18800}%
%
\begin{tikzpicture}

\begin{axis}[%
width=4.521in,
height=3.566in,
at={(0.758in,0.481in)},
scale only axis,
xmin=0,
xlabel style={font=\color{white!15!black}},
xlabel={Average subband rate [bps/Hz]},
ymin=0,
ylabel style={font=\color{white!15!black}},
ylabel={Average output DC current [$\mu$A]},
axis background/.style={fill=white},
xmajorgrids,
ymajorgrids,
legend style={legend cell align=left, align=left, draw=white!15!black}
]
\addplot [color=mycolor1, line width=2.0pt, mark=o, mark options={solid, mycolor1}]
  table[row sep=crcr]{%
5.62971756911577	0\\
5.43558937646659	1.33252799107006\\
5.24146118497288	2.7986795714249\\
5.04733299309185	4.29198893697339\\
4.85320480179634	5.74601077686804\\
4.65907661345474	7.12163050200202\\
4.46494842883311	8.39820581189421\\
4.27082026275745	9.53366125558458\\
4.07669211556099	10.5797051135205\\
3.88256401746303	11.6920496754887\\
3.68843602308717	12.9710500359586\\
3.49430816753776	14.4974171662899\\
3.30018042908507	16.1702388525605\\
3.10605317749813	17.9532763003293\\
2.91192589876244	19.8350654963931\\
2.7177987160973	21.7880322214965\\
2.52367139872176	23.792414204289\\
2.32954383228037	25.8340011729511\\
2.13541648786422	27.8990437784181\\
1.94128974567269	29.9821840417189\\
1.74716255671205	32.08278950124\\
1.55303566298019	34.2022429021717\\
1.35890896739747	36.3438523851885\\
1.16478225794241	38.5190086993999\\
0.970657441863005	40.7479905121655\\
0.776534800060461	43.0654408420295\\
0.582415639004176	45.5347624161656\\
0.38830114405591	48.2851042615184\\
0.194182852323423	51.6573336243653\\
7.79271316870932e-08	59.2703798102306\\
};
\addlegendentry{$d_H = 1$ m}

\addplot [color=mycolor2, dashed, line width=2.0pt, mark=+, mark options={solid, mycolor2}]
  table[row sep=crcr]{%
5.52514023798568	0\\
5.33461816034897	1.23273331799257\\
5.14409608359791	2.57955507507106\\
4.95357400662628	3.94707508500453\\
4.76305193040024	5.27693487101075\\
4.57252985880187	6.53477420320902\\
4.38200779461748	7.7024434052501\\
4.19148574595849	8.75020236521257\\
4.00096374912918	9.70605353577857\\
3.81044182631828	10.6929992652021\\
3.61992000280051	11.8386772871507\\
3.42939849960544	13.1908524285036\\
3.23887727721874	14.6844716541182\\
3.04835618887448	16.282932872661\\
2.8578356116359	17.9623550318536\\
2.66731453284533	19.7136155753527\\
2.47679377553042	21.5113251156866\\
2.28627256158812	23.3464744230169\\
2.09575078019503	25.208443517656\\
1.90523041718189	27.0893437087216\\
1.7147107635222	28.9847973160248\\
1.5241906844373	30.8971467824258\\
1.33366995321798	32.8332947409296\\
1.14315029717544	34.8045403173421\\
0.952632487343273	36.8277298186975\\
0.76211574717381	38.9348294930658\\
0.571601773590314	41.1837111803125\\
0.381093047411238	43.6922467546441\\
0.190580091996301	46.7713283860573\\
1.5855649339828e-06	53.7340774313462\\
};
\addlegendentry{$d_H = 2$ m}

\addplot [color=mycolor3, dotted, line width=2.0pt, mark=square, mark options={solid, mycolor3}]
  table[row sep=crcr]{%
5.37185480264753	0\\
5.18661842968689	1.10276860474339\\
5.00138205769196	2.29616523471063\\
4.81614568656564	3.50275595295885\\
4.63090931701202	4.67406840369621\\
4.44567295550072	5.78156253650663\\
4.26043662878479	6.81018036691245\\
4.07520035661757	7.74299873655263\\
3.88996414998658	8.58809996424829\\
3.70472821221378	9.43494618501664\\
3.5194926369243	10.407876944825\\
3.33425687697099	11.5404721289505\\
3.14902157804936	12.8133383113416\\
2.96378663175517	14.1789458245481\\
2.77855169555365	15.6116361147422\\
2.59331708679886	17.1060637764187\\
2.40808243923072	18.6481369478694\\
2.22284771264734	20.2235165695412\\
2.03761310866046	21.8254890576065\\
1.85237832748827	23.4472249187726\\
1.66714408480583	25.0873219820655\\
1.48191076791296	26.744648829895\\
1.29667621810219	28.4243702369402\\
1.11144285659041	30.1347323330858\\
0.926209727796795	31.8940083378838\\
0.740980293862373	33.7316008361788\\
0.555752953190297	35.6971716791412\\
0.370527991735988	37.8958571723873\\
0.185299369837713	40.6002442617824\\
2.23650569062746e-06	46.7264428411617\\
};
\addlegendentry{$d_H = 4$ m}

\addplot [color=mycolor4, dashdotted, line width=2.0pt, mark=x, mark options={solid, mycolor4}]
  table[row sep=crcr]{%
5.31502829750509	0\\
5.13175145947378	1.05901352989469\\
4.94847462213536	2.20127627133738\\
4.76519778727132	3.35444562759418\\
4.58192095498212	4.47320673968442\\
4.39864413809695	5.53089033208736\\
4.2153673877907	6.51341801702306\\
4.03209074337974	7.4151774895707\\
3.84881419109927	8.21362324059535\\
3.6655379814282	9.02061037991632\\
3.48226195349221	9.93751176683657\\
3.29898574146922	10.9969963517932\\
3.11571006540933	12.2005506919816\\
2.93243462766662	13.4902242808989\\
2.74915929577502	14.8452903585386\\
2.56588412673569	16.2529961583282\\
2.38260963853632	17.7126957770154\\
2.19933504265549	19.2038135614297\\
2.01605973895946	20.7211917089166\\
1.83278582061697	22.2575790318067\\
1.6495105058615	23.8121863387721\\
1.46623702749623	25.3843917773169\\
1.28296352674759	26.9798498541261\\
1.09968865770101	28.6055278796973\\
0.916415929992141	30.2769978175873\\
0.733145315789887	32.0261070169563\\
0.549876092927185	33.8987832442813\\
0.366611204391133	35.9941250164919\\
0.183341728257561	38.5736848231281\\
3.00580424622898e-06	44.4217569570318\\
};
\addlegendentry{$d_H = 6$ m}

\addplot [color=mycolor5, line width=2.0pt, mark=triangle, mark options={solid, mycolor5}]
  table[row sep=crcr]{%
5.28942153168888	0\\
5.1070276857406	1.04001905716027\\
4.92463384047887	2.16016645958468\\
4.74223999902017	3.29026764028289\\
4.55984617433321	4.38635316717864\\
4.37745234316496	5.42253933372905\\
4.19505862082739	6.38516994180292\\
4.01266505648352	7.26879907601144\\
3.83027155506989	8.04996478618108\\
3.64787838381793	8.84167636961934\\
3.4654852417826	9.73147830157504\\
3.2830920330011	10.7655868703336\\
3.10069940957405	11.9353836630271\\
2.91830720435323	13.1917314976326\\
2.73591478990386	14.5142328930052\\
2.55352320771064	15.8875641945599\\
2.37113140799457	17.3103101829804\\
2.18873988785427	18.7651015384578\\
2.00634828443065	20.2460468323242\\
1.82395692979393	21.7468749205771\\
1.64156529570313	23.2640332192424\\
1.45917350453234	24.8008519612018\\
1.27678351608541	26.3596084273071\\
1.09439223618738	27.9495772802957\\
0.91200116868915	29.5853397806375\\
0.729615013053546	31.2938014886304\\
0.54722975907755	33.125747479013\\
0.364847146807028	35.1762892874538\\
0.182459769165083	37.7021949556375\\
2.95411880011399e-06	43.4299683708002\\
};
\addlegendentry{$d_H = 7.5$ m}

\end{axis}
\end{tikzpicture}%
			\end{adjustbox}
			\caption{Average R-E region versus AP-IRS horizontal distance}
			\label{fi:re_distance}
		\end{figure}

		\begin{figure}[!t]
			\centering
			\begin{adjustbox}{width=\linewidth}
				% This file was created by matlab2tikz.
%
%The latest updates can be retrieved from
%  http://www.mathworks.com/matlabcentral/fileexchange/22022-matlab2tikz-matlab2tikz
%where you can also make suggestions and rate matlab2tikz.
%
\definecolor{mycolor1}{rgb}{0.00000,0.44700,0.74100}%
%
\begin{tikzpicture}

\begin{axis}[%
width=4.521in,
height=1.537in,
at={(0.758in,2.51in)},
scale only axis,
xmin=0,
xmax=15,
xlabel style={font=\color{white!15!black}},
xlabel={AP-user distance [m]},
ymin=40,
ymax=61.297120940948,
ylabel style={font=\color{white!15!black}},
ylabel={Path loss [dB]},
axis background/.style={fill=white},
xmajorgrids,
ymajorgrids,
legend style={at={(0.97,0.03)}, anchor=south east, legend cell align=left, align=left, draw=white!15!black}
]
\addplot [color=mycolor1, line width=2.0pt]
  table[row sep=crcr]{%
0	41.0205999132796\\
0.151515151515152	41.045453734224\\
0.303030303030303	41.119173024099\\
0.454545454545455	41.2393241202031\\
0.606060606060606	41.4021396619088\\
0.757575757575758	41.6028866197661\\
0.909090909090909	41.8362747679595\\
1.06060606060606	42.0968414276089\\
1.21212121212121	42.3792680932436\\
1.36363636363636	42.6786086486662\\
1.51515151515152	42.9904292915937\\
1.66666666666667	43.3108732557144\\
1.81818181818182	43.6366689469662\\
1.96969696969697	43.9651002221593\\
2.12121212121212	44.2939546618792\\
2.27272727272727	44.6214617583271\\
2.42424242424242	44.9462291379549\\
2.57575757575758	45.2671818185767\\
2.72727272727273	45.5835071980429\\
2.87878787878788	45.894606909446\\
3.03030303030303	46.2000556990542\\
3.18181818181818	46.4995669240429\\
3.33333333333333	46.7929639893089\\
3.48484848484848	47.0801569417483\\
3.63636363636364	47.3611234431104\\
3.78787878787879	47.6358934005081\\
3.93939393939394	47.9045366164644\\
4.09090909090909	48.1671529099816\\
4.24242424242424	48.4238642465639\\
4.39393939393939	48.6748084934752\\
4.54545454545455	48.9201344848418\\
4.6969696969697	49.1599981393195\\
4.84848484848485	49.3945594215869\\
5	49.6239799789896\\
5.15151515151515	49.8484213174239\\
5.3030303030303	50.0680434071776\\
5.45454545454545	50.2830036309842\\
5.60606060606061	50.4934560039122\\
5.75757575757576	50.6995506086871\\
5.90909090909091	50.9014332012627\\
6.06060606060606	51.0992449504655\\
6.21212121212121	51.2931222827625\\
6.36363636363636	51.4831968089947\\
6.51515151515152	51.6695953145642\\
6.66666666666667	51.8524397982926\\
6.81818181818182	52.0318475481556\\
6.96969696969697	52.2079312445011\\
7.12121212121212	52.3807990832831\\
7.27272727272727	52.5505549133907\\
7.42424242424242	52.7172983833954\\
7.57575757575758	52.8811250940335\\
7.72727272727273	53.0421267535465\\
7.87878787878788	53.2003913336394\\
8.03030303030303	53.356003224335\\
8.18181818181818	53.5090433864144\\
8.33333333333333	53.6595895004631\\
8.48484848484848	53.8077161118052\\
8.63636363636364	53.9534947708184\\
8.78787878787879	54.0969941682874\\
8.93939393939394	54.2382802655863\\
9.09090909090909	54.3774164195812\\
9.24242424242424	54.5144635022256\\
9.39393939393939	54.6494800148849\\
9.54545454545454	54.7825221974697\\
9.6969696969697	54.9136441324962\\
9.84848484848485	55.0750712273808\\
10	55.2980834377287\\
10.1515151515152	55.5179981888053\\
10.3030303030303	55.7348966761577\\
10.4545454545455	55.9488571069713\\
10.6060606060606	56.1599548349908\\
10.7575757575758	56.3682624886991\\
10.9090909090909	56.5738500930783\\
11.0606060606061	56.7767851852634\\
11.2121212121212	56.9771329243941\\
11.3636363636364	57.1749561959588\\
11.5151515151515	57.3703157109124\\
11.6666666666667	57.5632700998407\\
11.8181818181818	57.7538760024303\\
11.969696969697	57.9421881524909\\
12.1212121212121	58.1282594587669\\
12.2727272727273	58.3121410817628\\
12.4242424242424	58.4938825067928\\
12.5757575757576	58.6735316134598\\
12.7272727272727	58.8511347417525\\
12.8787878787879	59.0267367549428\\
13.030303030303	59.2003810994536\\
13.1818181818182	59.3721098618608\\
13.3333333333333	59.5419638231796\\
13.4848484848485	59.709982510582\\
13.6363636363636	59.8762042466802\\
13.7878787878788	60.040666196506\\
13.9393939393939	60.2034044123058\\
14.0909090909091	60.3644538762683\\
14.2424242424242	60.5238485412904\\
14.3939393939394	60.681621369886\\
14.5454545454545	60.837804371332\\
14.6969696969697	60.9924286371433\\
14.8484848484848	61.145524374963\\
15	61.297120940948\\
};
\addlegendentry{$\Lambda_{\text{D}}$}

\end{axis}

\begin{axis}[%
width=4.521in,
height=1.537in,
at={(0.758in,0.481in)},
scale only axis,
xmin=0,
xmax=15,
xlabel style={font=\color{white!15!black}},
xlabel={AP-IRS horizontal distance [m]},
ymin=102,
ymax=106,
ylabel style={font=\color{white!15!black}},
ylabel={Path loss product [dB]},
axis background/.style={fill=white},
xmajorgrids,
ymajorgrids,
legend style={legend cell align=left, align=left, draw=white!15!black}
]
\addplot [color=mycolor1, line width=2.0pt]
  table[row sep=crcr]{%
0	102.317720854228\\
0.151515151515152	102.190978109187\\
0.303030303030303	102.111601661242\\
0.454545454545455	102.077128491535\\
0.606060606060606	102.083761031795\\
0.757575757575758	102.126735161056\\
0.909090909090909	102.200728644228\\
1.06060606060606	102.300245839915\\
1.21212121212121	102.41993428975\\
1.36363636363636	102.554812895346\\
1.51515151515152	102.700411802176\\
1.66666666666667	102.852837078894\\
1.81818181818182	103.008778808827\\
1.96969696969697	103.165481321613\\
2.12121212121212	103.320691416822\\
2.27272727272727	103.47259650008\\
2.42424242424242	103.619760751415\\
2.57575757575758	103.761064325369\\
2.72727272727273	103.895648279806\\
2.87878787878788	104.022866368213\\
3.03030303030303	104.142243851545\\
3.18181818181818	104.253442926473\\
3.33333333333333	104.35623408915\\
3.48484848484848	104.450472652661\\
3.63636363636364	104.536079639069\\
3.78787878787879	104.613026324902\\
3.93939393939394	104.681321801728\\
4.09090909090909	104.74100300306\\
4.24242424242424	104.792126735263\\
4.39393939393939	104.834763328466\\
4.54545454545455	104.868991591813\\
4.6969696969697	104.894894815477\\
4.84848484848485	104.912557610392\\
5	104.922063416718\\
5.15151515151515	104.923492544805\\
5.3030303030303	104.981687539674\\
5.45454545454545	105.065525828454\\
5.60606060606061	105.142936018797\\
5.75757575757576	105.214014110913\\
5.90909090909091	105.278849620844\\
6.06060606060606	105.337525216052\\
6.21212121212121	105.39011645105\\
6.36363636363636	105.436691579813\\
6.51515151515152	105.47731142637\\
6.66666666666667	105.512029298756\\
6.81818181818182	105.54089093457\\
6.96969696969697	105.563934468836\\
7.12121212121212	105.581190416922\\
7.27272727272727	105.592681666937\\
7.42424242424242	105.598423477429\\
7.57575757575758	105.598423477429\\
7.72727272727273	105.592681666937\\
7.87878787878788	105.581190416922\\
8.03030303030303	105.563934468836\\
8.18181818181818	105.54089093457\\
8.33333333333333	105.512029298756\\
8.48484848484848	105.47731142637\\
8.63636363636364	105.436691579813\\
8.78787878787879	105.39011645105\\
8.93939393939394	105.337525216052\\
9.09090909090909	105.278849620844\\
9.24242424242424	105.214014110913\\
9.39393939393939	105.142936018797\\
9.54545454545454	105.065525828454\\
9.6969696969697	104.981687539674\\
9.84848484848485	104.923492544805\\
10	104.922063416718\\
10.1515151515152	104.912557610392\\
10.3030303030303	104.894894815477\\
10.4545454545455	104.868991591813\\
10.6060606060606	104.834763328466\\
10.7575757575758	104.792126735263\\
10.9090909090909	104.74100300306\\
11.0606060606061	104.681321801728\\
11.2121212121212	104.613026324902\\
11.3636363636364	104.536079639069\\
11.5151515151515	104.450472652661\\
11.6666666666667	104.35623408915\\
11.8181818181818	104.253442926473\\
11.969696969697	104.142243851545\\
12.1212121212121	104.022866368213\\
12.2727272727273	103.895648279806\\
12.4242424242424	103.761064325369\\
12.5757575757576	103.619760751415\\
12.7272727272727	103.47259650008\\
12.8787878787879	103.320691416822\\
13.030303030303	103.165481321613\\
13.1818181818182	103.008778808827\\
13.3333333333333	102.852837078894\\
13.4848484848485	102.700411802176\\
13.6363636363636	102.554812895346\\
13.7878787878788	102.41993428975\\
13.9393939393939	102.300245839915\\
14.0909090909091	102.200728644228\\
14.2424242424242	102.126735161056\\
14.3939393939394	102.083761031795\\
14.5454545454545	102.077128491535\\
14.6969696969697	102.111601661242\\
14.8484848484848	102.190978109187\\
15	102.317720854228\\
};
\addlegendentry{$\Lambda_{\text{I}}\Lambda_{\text{R}}$}

\end{axis}
\end{tikzpicture}%

			\end{adjustbox}
			\caption{Path loss versus distance}
			\label{fi:path_loss}
		\end{figure}

		In Fig.~\ref{fi:re_distance}, we compare the average R-E region achieved by different AP-IRS horizontal distance $d_H$. A \textit{first} observation is that placing the IRS closer to either the AP or the user would improve the R-E tradeoff. This origins from the product-distance path loss model that applies to finite-size element reflection. As shown in Fig.~\ref{fi:path_loss}, although the piecewise TGn path loss model further penalizes large distance (greater than \SI{10}{\meter} for model D), it is still beneficial to have a short-long or long-short transmission setup, since signal attenuation increases fast at a short distance and experiences marginal effect at a long distance. On the other hand, it also suggests that developing an IRS next to the AP can effectively extend the operation range of SWIPT systems. Considering the passive characteristic of IRS, opportunities are that it can be directly supported by the SWIPT network. A \textit{second} observation is that there exists two optimal IRS development locations that maximizes the path loss production $\Lambda_I\Lambda_R$. It implies that more than one IRS may be implemented to further enlarge the R-E region, one attached to the AP and one attached to the IRS.

		\begin{figure}[!t]
			\centering
			\begin{adjustbox}{width=\linewidth}
				% This file was created by matlab2tikz.
%
%The latest updates can be retrieved from
%  http://www.mathworks.com/matlabcentral/fileexchange/22022-matlab2tikz-matlab2tikz
%where you can also make suggestions and rate matlab2tikz.
%
\definecolor{mycolor1}{rgb}{0.00000,0.44700,0.74100}%
\definecolor{mycolor2}{rgb}{0.85000,0.32500,0.09800}%
\definecolor{mycolor3}{rgb}{0.92900,0.69400,0.12500}%
\definecolor{mycolor4}{rgb}{0.49400,0.18400,0.55600}%
\definecolor{mycolor5}{rgb}{0.46600,0.67400,0.18800}%
%
\begin{tikzpicture}

\begin{axis}[%
width=4.521in,
height=3.566in,
at={(0.758in,0.481in)},
scale only axis,
xmin=0,
xlabel style={font=\color{white!15!black}},
xlabel={Average subband rate [bps/Hz]},
ymin=0,
ylabel style={font=\color{white!15!black}},
ylabel={Average output DC current [$\mu$A]},
axis background/.style={fill=white},
xmajorgrids,
ymajorgrids,
legend style={legend cell align=left, align=left, draw=white!15!black}
]
\addplot [color=mycolor1, line width=2.0pt, mark=o, mark options={solid, mycolor1}]
  table[row sep=crcr]{%
5.39398168528569	0\\
5.20798231612376	1.22068934125607\\
5.02198294841078	2.56950277527446\\
4.83598358031069	3.94504339005404\\
4.64998421548369	5.28407711465017\\
4.46398485764056	6.54942877072522\\
4.27798553126039	7.72144130681083\\
4.091986243543	8.7752876871929\\
3.90598709676167	9.74071738963898\\
3.71998810918807	10.7458731914163\\
3.53398949723431	12.010638947443\\
3.34799110781442	13.4346431937548\\
3.16199303634301	15.0037791719369\\
2.9759947374505	16.6753590900413\\
2.78999717951576	18.4253824654167\\
2.6039993003971	20.2429126752568\\
2.41800303993126	22.0993478965388\\
2.2320066213693	23.9888020825773\\
2.046009851812	25.9018778833084\\
1.86001408578827	27.8288584295265\\
1.67401901678915	29.7691546576733\\
1.48802158109351	31.7204258928399\\
1.30202381771407	33.6924372577175\\
1.11602625014286	35.6923496649217\\
0.930030812393232	37.7383792852281\\
0.74403768773903	39.8630695849754\\
0.558045288723006	42.125552446907\\
0.372053838720171	44.6406653795928\\
0.186057713724565	47.7219566310904\\
2.36932454981797e-06	54.7172968182554\\
};
\addlegendentry{$M = 1, L = 20$}

\addplot [color=mycolor2, dashed, line width=2.0pt, mark=+, mark options={solid, mycolor2}]
  table[row sep=crcr]{%
6.38939005284257	0\\
6.16906625750341	2.69287628984093\\
5.94874246292913	5.95082528791035\\
5.72841866804983	9.39307366824786\\
5.50809487323903	12.7873969053184\\
5.28777107891079	15.9999752865681\\
5.06744728549881	18.9611565604305\\
4.84712349547719	21.4864276452412\\
4.62679970994188	24.3316448730531\\
4.4064759398445	27.8656715446675\\
4.18615219636047	32.0493991155135\\
3.96582846588238	36.7434997655354\\
3.74550472804516	41.7559199866439\\
3.52518113267417	46.9774640185028\\
3.30485743916887	52.3629740854898\\
3.08453361835898	57.8492242308327\\
2.86420980675741	63.3856051619952\\
2.64388597983508	68.9353678592478\\
2.42356279142047	74.4703959123556\\
2.20323938505362	79.9696019972744\\
1.98291595605503	85.4231515850158\\
1.76259236068624	90.8355884421288\\
1.54226920137643	96.2170582818137\\
1.3219459417765	101.593889756663\\
1.10162335893682	107.012420776369\\
0.881301296260641	112.549536501417\\
0.660981736659531	118.348264205722\\
0.440666053512689	124.691980975248\\
0.220352992187177	132.330613664532\\
2.1203184642114e-08	149.312998721595\\
};
\addlegendentry{$M = 1, L = 40$}

\addplot [color=mycolor3, dotted, line width=2.0pt, mark=square, mark options={solid, mycolor3}]
  table[row sep=crcr]{%
7.43070367742958	0\\
7.17447251579894	6.76223494050246\\
6.91824135479948	16.0225139073115\\
6.66201019351539	26.1805852226025\\
6.40577903225118	36.2911655949891\\
6.14954787101543	45.8283591559144\\
5.89331670989148	54.2491141254435\\
5.6370855491387	62.6162727914728\\
5.38085439061751	74.0389608242132\\
5.12462323114612	88.4560971397017\\
4.86839207145889	104.518543166208\\
4.61216090795218	121.694160515101\\
4.35592974312383	139.613360953075\\
4.09969858088462	158.000731345434\\
3.84346741947112	176.622895040107\\
3.58723625822777	195.283720833368\\
3.33100509696229	213.823680843438\\
3.07477393576479	232.118479095191\\
2.8185427745383	250.078379785211\\
2.56231161339942	267.645727871055\\
2.30608045224395	284.795747297748\\
2.04984929120957	301.532866603133\\
1.793618129938	317.900066212108\\
1.53738697002282	333.975473915193\\
1.28115582337954	349.896043264659\\
1.02492478597244	365.879637297305\\
0.768694280513084	382.303920868734\\
0.512465816096588	399.921379429795\\
0.256241685289694	420.70263297094\\
1.01524736902585e-09	465.923450281919\\
};
\addlegendentry{$M = 1, L = 80$}

\addplot [color=mycolor4, dashdotted, line width=2.0pt, mark=x, mark options={solid, mycolor4}]
  table[row sep=crcr]{%
8.23880138607269	0\\
7.95470478628136	15.1034292760118\\
7.67060818695366	38.0861397821963\\
7.38651158746603	63.9035612748088\\
7.10241498798263	89.6638776069751\\
6.81831838848447	113.81579900219\\
6.53422178905932	133.897096443946\\
6.25012519006625	160.331974362531\\
5.96602859022063	198.465197550024\\
5.68193199043858	241.636182196109\\
5.39783539091209	288.216282398954\\
5.11373879138408	337.18320550881\\
4.82964219187496	387.63710929526\\
4.54554559239449	438.792723671441\\
4.26144899289152	489.983130047552\\
3.97735239340111	540.660956718383\\
3.69325579393235	590.395104529155\\
3.4091591944058	638.864257508494\\
3.12506259508036	685.84846477669\\
2.84096599575534	731.221682106316\\
2.55686939635633	774.943072674845\\
2.27277279689	817.054764246695\\
1.98867619749331	857.682676799196\\
1.70457959786	897.046588953194\\
1.42048299833069	935.487168236536\\
1.13638639946855	973.530871179723\\
0.852289819533533	1012.04747029419\\
0.56819359674422	1052.71426971506\\
0.284100445163735	1099.89249843865\\
5.63069237482087e-11	1200.52634620485\\
};
\addlegendentry{$M = 1, L = 120$}

\addplot [color=mycolor5, line width=2.0pt, mark=triangle, mark options={solid, mycolor5}]
  table[row sep=crcr]{%
8.84101363520959	0\\
8.53615109571498	28.7625878560105\\
8.23128855686155	75.8616032293243\\
7.92642601774056	129.38968578531\\
7.62156347861622	182.691348226463\\
7.31670093948751	232.31008339363\\
7.0118384004381	271.459476915312\\
6.70697586121548	343.07066409621\\
6.4021133220667	430.805069014299\\
6.09725078293508	527.760855838789\\
5.7923882437851	631.335275812944\\
5.48752570465246	739.225936188985\\
5.18266316551724	849.393692404821\\
4.87780062641785	960.083229275764\\
4.57293808732327	1069.83525672043\\
4.26807554822051	1177.48379802341\\
3.96321300908769	1282.14078304626\\
3.65835047004049	1383.1731858147\\
3.35348793116348	1480.1771810739\\
3.04862539208979	1572.95181834235\\
2.74376285297585	1661.47842505597\\
2.43890031382818	1745.90541473069\\
2.13403777468744	1826.53891097427\\
1.82917523555294	1903.87016503731\\
1.52431269648603	1978.60510517558\\
1.21945015748736	2051.78283734276\\
0.914587619880958	2125.06463581438\\
0.609725136883748	2201.54796518127\\
0.304864284502245	2289.18056397853\\
2.20539018341826e-11	2473.30891344661\\
};
\addlegendentry{$M = 1, L = 160$}

\end{axis}
\end{tikzpicture}%
			\end{adjustbox}
			\caption{Average R-E region versus the number of reflectors}
			\label{fi:re_reflector}
		\end{figure}

		\begin{figure}[!t]
			\centering
			\begin{adjustbox}{width=\linewidth}
				% This file was created by matlab2tikz.
%
%The latest updates can be retrieved from
%  http://www.mathworks.com/matlabcentral/fileexchange/22022-matlab2tikz-matlab2tikz
%where you can also make suggestions and rate matlab2tikz.
%
\definecolor{mycolor1}{rgb}{0.00000,0.44700,0.74100}%
\definecolor{mycolor2}{rgb}{0.85000,0.32500,0.09800}%
\definecolor{mycolor3}{rgb}{0.92900,0.69400,0.12500}%
\definecolor{mycolor4}{rgb}{0.49400,0.18400,0.55600}%
\definecolor{mycolor5}{rgb}{0.46600,0.67400,0.18800}%
%
\begin{tikzpicture}

\begin{axis}[%
width=4.521in,
height=3.566in,
at={(0.758in,0.481in)},
scale only axis,
xmin=0,
xlabel style={font=\color{white!15!black}},
xlabel={Average subband rate [bps/Hz]},
ymin=0,
ylabel style={font=\color{white!15!black}},
ylabel={Average output DC current [$\mu$A]},
axis background/.style={fill=white},
xmajorgrids,
ymajorgrids,
legend style={legend cell align=left, align=left, draw=white!15!black}
]
\addplot [color=mycolor1, line width=2.0pt, mark=o, mark options={solid, mycolor1}]
  table[row sep=crcr]{%
5.50914838827946	0\\
5.31917775376218	1.21238415018016\\
5.12920712006203	2.53237434713813\\
4.93923648620775	3.87072056595873\\
4.74926585314339	5.17155623912808\\
4.55929522495152	6.4019337604629\\
4.36932460435472	7.54442965987761\\
4.17935399971781	8.58051451999257\\
3.98938344864228	9.49393206192153\\
3.79941297282513	10.4449974695456\\
3.60944259743222	11.5482026531884\\
3.41947256120648	12.8547904568156\\
3.22950281213667	14.2994951635005\\
3.03953319078188	15.8477053110534\\
2.84956411149923	17.4772022230921\\
2.65959455954133	19.1765527931851\\
2.46962522493195	20.9227426883177\\
2.27965547891626	22.7064304053862\\
2.08968519753875	24.5175977524265\\
1.89971635717942	26.3485393662365\\
1.70974811834305	28.1947945918833\\
1.51977953341583	30.0588513628518\\
1.32981021004462	31.9485579940201\\
1.13984192648845	33.8728277377253\\
0.949875845432248	35.8491157957908\\
0.759910653496104	37.9086273864258\\
0.569948171233342	40.1074108283297\\
0.379990602447721	42.5619962028736\\
0.190028583875184	45.578028320887\\
1.63891137764227e-06	52.4027633496848\\
};
\addlegendentry{$M = 1, L = 20$}

\addplot [color=mycolor2, dashed, line width=2.0pt, mark=+, mark options={solid, mycolor2}]
  table[row sep=crcr]{%
6.46189873526159	0\\
6.23907464042722	2.73723959929336\\
6.01625054658382	6.01987480905357\\
5.79342645229771	9.47936996496051\\
5.57060235843012	12.8897796145773\\
5.34777826472142	16.1201587119815\\
5.12495417340215	19.1019371373066\\
4.90213008936536	21.631149830557\\
4.67930600688948	24.3460462059138\\
4.45648195743752	27.7094305898286\\
4.23365790829412	31.8404343394148\\
4.01083391230115	36.4680743353983\\
3.7880099818419	41.418903535318\\
3.56518612806095	46.5912021403416\\
3.34236218293948	51.9392639192043\\
3.11953814885029	57.3912235139604\\
2.89671403345102	62.9025009809857\\
2.67388999187562	68.4355872838161\\
2.45106593830422	73.9614345462637\\
2.22824199991204	79.4619653726271\\
2.00541789538026	84.9273480334271\\
1.78259387154366	90.3578141323815\\
1.55977018263179	95.7652746152504\\
1.33694680303105	101.175808249317\\
1.11412389726573	106.636071206736\\
0.891301011824557	112.224900218251\\
0.668480992178671	118.086028912171\\
0.445665011213708	124.5081383894\\
0.222853670486381	132.25498170037\\
1.99460263980914e-08	149.492785651764\\
};
\addlegendentry{$M = 2, L = 20$}

\addplot [color=mycolor3, dotted, line width=2.0pt, mark=square, mark options={solid, mycolor3}]
  table[row sep=crcr]{%
7.35669951826879	0\\
7.10302022415528	6.20695838254019\\
6.84934093071593	14.5555150709397\\
6.59566163702678	23.6766693832912\\
6.34198234332767	32.7550836230748\\
6.08830304966753	41.3328722174662\\
5.83462375612751	49.1392393654032\\
5.58094446313422	55.8653798217785\\
5.32726517237947	65.6302151933533\\
5.07358588237067	78.1136870014957\\
4.81990659564493	92.1002689046529\\
4.56622729499044	107.116906156919\\
4.31254799927282	122.826848866696\\
4.05886870647144	138.996067378272\\
3.80518940974032	155.416860397166\\
3.55151011482067	171.915548506208\\
3.29783081976253	188.350381672684\\
3.04415152550868	204.610110942687\\
2.79047223159697	220.613072675702\\
2.53679293788656	236.306164338218\\
2.28311364442439	251.664248254292\\
2.02943435096141	266.690744843053\\
1.77575505752962	281.419721976425\\
1.52207576431207	295.922027696879\\
1.26839648013976	310.318563303724\\
1.01471727849096	324.808151289411\\
0.761038467157238	339.736675131913\\
0.507361550712717	355.792353558055\\
0.253689569734145	374.785607066888\\
2.10823254814809e-09	416.256686325004\\
};
\addlegendentry{$M = 4, L = 20$}

\addplot [color=mycolor4, dashdotted, line width=2.0pt, mark=x, mark options={solid, mycolor4}]
  table[row sep=crcr]{%
7.89718716435744	0\\
7.62487036522532	10.3237068848587\\
7.35255356675235	25.1926293047823\\
7.08023676800697	41.7190780171817\\
6.80791996925535	58.2148953649991\\
6.53560317052696	73.7527375673274\\
6.26328637185538	87.2714623973032\\
5.99096957371599	100.951262556214\\
5.71865277584552	122.571571344387\\
5.44633597819243	148.145893283286\\
5.17401917800472	176.057935773479\\
4.90170237819323	205.626472804135\\
4.62938557918985	236.307908454541\\
4.35706878040348	267.633070250063\\
4.08475198167879	299.198435611311\\
3.81243518291843	330.666820968081\\
3.54011838419245	361.766674372583\\
3.26780158546565	392.28977879866\\
2.99548478676102	422.087902710721\\
2.72316798817262	451.069257777827\\
2.45085118955515	479.195748238163\\
2.17853439092251	506.481916162961\\
1.90621759223693	532.997313212395\\
1.63390079351373	558.875513396727\\
1.36158399534812	584.333595825453\\
1.08926720608478	609.719850016771\\
0.81695051265075	635.624505674896\\
0.544634677967558	663.203322509861\\
0.272322224679745	695.477885815851\\
3.99906817076994e-10	765.047653590634\\
};
\addlegendentry{$M = 6, L = 20$}

\addplot [color=mycolor5, line width=2.0pt, mark=triangle, mark options={solid, mycolor5}]
  table[row sep=crcr]{%
8.36494421020624	0\\
8.07649785782375	17.0906432523215\\
7.78805150599832	43.4349226354144\\
7.49960515394812	73.1098964996926\\
7.21115880188307	102.729032880964\\
6.92271244984901	130.482089171242\\
6.63426609786475	153.103823907926\\
6.34581974609309	184.404455315327\\
6.05737339380685	228.95309474102\\
5.76892704253152	279.041577395493\\
5.48048068964625	333.030792139764\\
5.19203433741295	389.719782244971\\
4.90358798534091	448.061758998046\\
4.61514163329533	507.143846038953\\
4.32669528126863	566.192369447304\\
4.03824892921281	624.574052919183\\
3.74980257715848	681.791904363296\\
3.46135622514287	737.47719023877\\
3.17290987323309	791.379374309841\\
2.88446352124382	843.3559192679\\
2.59601716935408	893.363636162664\\
2.30757081731741	941.453504883204\\
2.01912446525775	987.771447076769\\
1.73067811318711	1032.56934839507\\
1.442231761333	1076.2366674772\\
1.15378541250716	1119.37248534768\\
0.8653390988339	1162.96529125834\\
0.576893004754171	1208.90845079472\\
0.288448898824029	1262.09156431317\\
1.32759421907693e-10	1375.29943262755\\
};
\addlegendentry{$M = 8, L = 20$}

\end{axis}
\end{tikzpicture}%
			\end{adjustbox}
			\caption{Average R-E region versus the number of transmit antennas}
			\label{fi:re_tx}
		\end{figure}

		The impact of the number of transmit antennas $M$ and IRS reflectors $L$ on the average R-E tradeoff is revealed in Fig.~\ref{fi:re_reflector} and \ref{fi:re_tx}. A \textit{first} contrast indicates that adding either active or passive elements benefits both information and power transmission while preserving the concavity-convexity of the R-E region. This is because increasing $M$ or $L$ indeed enhances the equivalent composite channel strength such that the magnitude of the components in \ref{eq:z_expand} is amplified while the amount remains unchanged. Therefore, we conclude that number of transmit antennas and reflectors have basically no influence on the waveform preference and transceiving strategy. A \textit{second} contrast suggests that the system performance is more sensitive to the variation of $M$, compared with $L$. Interestingly, the active MRT beamformer only has a transmit array gain of $M$. In contrast, the IRS collects $L$ signal copies with a receive array gain $L$, then performs an equal gain reflection with a reflect array gain $L$, achieving a total array gain of $L^2$. However, the system performance in our setup is dominated by the direct link. As shown in Fig.~\ref{fi:path_loss}, the direct path loss $\Lambda_D$ is in the scope of \num{e-7} while the extra path loss product $\Lambda_I\Lambda_R$ is below \num{e-10}. This has the consequence that despite increasing $L$ can effectively enhance the AP-IRS-user extra channel, its amplitude is still too small compared with the AP-user channel such that increasing $M$ is more effective to improve the system performance.

		\begin{figure}[!t]
			\centering
			\begin{adjustbox}{width=\linewidth}
				% This file was created by matlab2tikz.
%
%The latest updates can be retrieved from
%  http://www.mathworks.com/matlabcentral/fileexchange/22022-matlab2tikz-matlab2tikz
%where you can also make suggestions and rate matlab2tikz.
%
\definecolor{mycolor1}{rgb}{0.00000,0.44700,0.74100}%
\definecolor{mycolor2}{rgb}{0.85000,0.32500,0.09800}%
\definecolor{mycolor3}{rgb}{0.92900,0.69400,0.12500}%
\definecolor{mycolor4}{rgb}{0.49400,0.18400,0.55600}%
\definecolor{mycolor5}{rgb}{0.46600,0.67400,0.18800}%
%
\begin{tikzpicture}

\begin{axis}[%
width=4.521in,
height=3.566in,
at={(0.758in,0.481in)},
scale only axis,
xmin=0,
xlabel style={font=\color{white!15!black}},
xlabel={Per-subband rate [bps/Hz]},
ymin=0,
ylabel style={font=\color{white!15!black}},
ylabel={Average output DC current [$\mu$A]},
axis background/.style={fill=white},
xmajorgrids,
ymajorgrids,
legend style={legend cell align=left, align=left, draw=white!15!black}
]
\addplot [color=mycolor1, line width=2.0pt, mark=o, mark options={solid, mycolor1}]
  table[row sep=crcr]{%
5.31970266706356	0\\
5.1362657213446	1.09604570660911\\
4.95282823695109	2.30514705995326\\
4.76938972115321	3.57332170750531\\
4.58595270912889	4.86308240644402\\
4.40251666378583	6.15569797033203\\
4.21907897430894	7.43614659310811\\
4.03564021926926	8.69740475140351\\
3.85220459633288	9.92089287134074\\
3.66876813890084	11.1370910244661\\
3.48533008247367	12.423570825327\\
3.30189068209451	13.7607598397124\\
3.11845323044667	15.266701470989\\
2.93501607307262	16.8774769604374\\
2.75157918178914	18.5586457442137\\
2.56814369572461	20.2975426576306\\
2.38470665824526	22.1690451283419\\
2.20127021471967	24.1310244907244\\
2.01783382794571	26.1647313151309\\
1.83439768454551	28.2781001254715\\
1.65096191072867	30.4848053398786\\
1.46752532671364	32.7360786391647\\
1.28408887158457	35.0854533733248\\
1.10065232977768	37.4713089765641\\
0.917219058579686	39.8846934780441\\
0.733788220387792	42.3500598941755\\
0.550355596259195	44.8567334648507\\
0.366921887824348	47.4943503371756\\
0.183480147954394	50.4592652886436\\
1.4438036339175e-06	56.1547191664797\\
};
\addlegendentry{Adaptive IRS}

\addplot [color=mycolor2, dashed, line width=2.0pt, mark=+, mark options={solid, mycolor2}]
  table[row sep=crcr]{%
5.6278027769738	0\\
5.43374072819082	1.35230136741796\\
5.23967867472939	2.8694209708053\\
5.04561668340728	4.46779681031452\\
4.85155474754104	6.09337585590751\\
4.65749289844586	7.71060909688446\\
4.46343119233529	9.29962691597598\\
4.26936968622296	10.8481606275978\\
4.07530850026345	12.3085716075278\\
3.88124727392873	13.7909056816444\\
3.68718614798559	15.2743022806056\\
3.49312464219046	16.9202306188447\\
3.29906568031663	18.7432956719355\\
3.10500926042539	20.6706941644464\\
2.91095118599523	22.7033781809729\\
2.71689819658157	24.8518324998376\\
2.52284466882218	27.1081715890367\\
2.3287934056686	29.4255550534737\\
2.13474427385823	31.800983980669\\
1.94069585726117	34.2068225314465\\
1.74664701799339	36.6445568721269\\
1.55259569349599	39.1210867408959\\
1.35854969746434	41.6431249035602\\
1.16450132964412	44.2101307772055\\
0.970453331858585	46.8347243193916\\
0.776399600093652	49.5454965153065\\
0.582345175306443	52.3935751573781\\
0.388273604553801	55.4793212697464\\
0.194190998189759	59.0430860959658\\
8.98028051389367e-05	66.1167888781429\\
};
\addlegendentry{Ideal FS IRS}

\addplot [color=mycolor3, dotted, line width=2.0pt, mark=square, mark options={solid, mycolor3}]
  table[row sep=crcr]{%
5.31970266706356	0\\
5.1362672133745	1.09362394709189\\
4.95282990474662	2.29788239712992\\
4.76939444475229	3.5563997537948\\
4.585958060628	4.831281674129\\
4.40252080595436	6.09847832353013\\
4.21908460096133	7.3434580983534\\
4.03564775626836	8.55830223430177\\
3.85221033580802	9.71211067922247\\
3.66877355007817	10.8469500287905\\
3.4853357147468	12.0046812560003\\
3.30189949555557	13.2703522517267\\
3.11846333795172	14.6562336331658\\
2.93503165025336	16.1214084117769\\
2.75160496285405	17.6633236551563\\
2.56817566497636	19.2925901030685\\
2.38475126608965	20.9847926877339\\
2.20132884045614	22.7177390224098\\
2.01790632123127	24.5225154822561\\
1.83448368222069	26.3623888006216\\
1.6510564845439	28.2320841840748\\
1.46763445629152	30.134728166163\\
1.28421445197826	32.061532223951\\
1.10079278087135	34.0453355009605\\
0.917368811110489	36.0738852329198\\
0.733935526768701	38.1809014536692\\
0.550505776743871	40.3998913056724\\
0.367058076225364	42.8167041414743\\
0.183589030483856	45.6301242985029\\
0.000123196899679623	51.2415507815106\\
};
\addlegendentry{WIT-optimized IRS}

\addplot [color=mycolor4, dashdotted, line width=2.0pt, mark=x, mark options={solid, mycolor4}]
  table[row sep=crcr]{%
5.13175013332208	0\\
4.95484638700898	1.13158003667145\\
4.77788129927089	2.37259914334427\\
4.60091364213886	3.69559994296276\\
4.4239506358599	5.06400360479036\\
4.24699021708218	6.45217513819988\\
4.07003163233392	7.84202878711384\\
3.89307268917821	9.22278140488796\\
3.7161164797994	10.5715454190425\\
3.53915798808049	11.9172592628762\\
3.36220049492727	13.2628303583231\\
3.18524254696462	14.7187983285235\\
3.00828679240369	16.3017960885387\\
2.83133215720001	17.949409662178\\
2.65438500906595	19.6825716884321\\
2.47743883719768	21.5050350781675\\
2.30049115938385	23.4381495633152\\
2.12354918832381	25.4120541733927\\
1.94660247040953	27.4326423958397\\
1.76965681915655	29.4670200643564\\
1.59270725675443	31.5826001706471\\
1.41575922434152	33.7292049598193\\
1.23881090312352	35.8982812335572\\
1.06185881860677	38.0961713089271\\
0.884912079451479	40.3285887274279\\
0.707961367901523	42.6265596001755\\
0.531007025974092	45.0260620178966\\
0.354048314635249	47.5863603538198\\
0.17706899514504	50.497439776212\\
7.15632110154294e-05	56.1405767318039\\
};
\addlegendentry{WPT-optimized IRS}

\addplot [color=mycolor5, line width=2.0pt, mark=triangle, mark options={solid, mycolor5}]
  table[row sep=crcr]{%
4.56085733293474	0\\
4.40373156417764	0.708780060508522\\
4.24644397339939	1.4306718165602\\
4.08915645467109	2.17611066045952\\
3.93188768852749	2.92890628488715\\
3.77461183023039	3.68013056327589\\
3.61733235691911	4.42332062752283\\
3.46005666583408	5.15416198136018\\
3.30278261304356	5.87142822873624\\
3.14550714231148	6.54908721810862\\
2.9882249321301	7.2653265235186\\
2.8309542653952	7.94699207962432\\
2.67367576152558	8.68844016432465\\
2.51640775038085	9.49013081394478\\
2.35913869830228	10.3096306048589\\
2.20187457349624	11.1583907149677\\
2.04461418514831	12.0537500309256\\
1.88736264607434	12.9837332549217\\
1.7301095312347	13.9449514545049\\
1.57285373446079	14.9273734079604\\
1.41560183974797	15.9122410172936\\
1.25834829794967	16.9494985852504\\
1.10109756536592	18.0035209735667\\
0.943841319652312	19.0876208992045\\
0.786583309703923	20.2027521718667\\
0.62932843361796	21.351870012521\\
0.472054769151119	22.5696241340477\\
0.314759394872696	23.8861813119793\\
0.157434679323937	25.3906631854878\\
0.000273625270677518	28.3457738927225\\
};
\addlegendentry{No IRS}

\end{axis}
\end{tikzpicture}%
			\end{adjustbox}
			\caption{Average R-E region for adaptive, ideal, fixed and no IRS over $B=10\,\si{\MHz}$}
			\label{fi:re_irs}
		\end{figure}

		Fig.~\ref{fi:re_irs} and [TODO] explore the average R-E region under different IRS configuration for narrowband transmission ($B=1\,\si{\MHz}$) and broadband transmission ($B=10\,\si{\MHz}$). The adaptive scheme optimizes the IRS and waveform alternatively for each points in the R-E boundary. In comparison, the WIT/WPT-based schemes only perform alternating optimization for the right-most/left-most points, then fix the IRS and update the waveform to obtain the R-E curve. To gain some insight into the IRS behavior, we compare the results above to that of no IRS and the ideal FS IRS, where we assume each element has independent and adjustable reflection coefficients over all subbands such that the ideal IRS has a total DoF of $NL$. Since the IRS only adapts the phase of the extra channel, the optimal strategy for each FS reflector in the SISO case would be aligning the AP-IRS-user and AP-user channel over all subbands, namely
		\begin{equation}
			\theta_{l,n}^{\star}=e^{j\arg{\left(h_{D,n}/(h_{I,n,l}h_{R,n,l})\right)}}
		\end{equation}
		\textit{First}, it is observed that the presence of IRS effectively enlarges the achievable R-E region in both cases. This is because the IRS tweaks the weak extra channels such that they add constructively to enhance the composite channel. \textit{Second}, the performance gaps of the adaptive, ideal and fixed IRS are negligible for narrowband transmission but noticeable for broadband transmission. The reason is that when $B=1\,\si{\MHz}$, all channels are approximately flat and the response of all subbands are roughly the same. In such cases, the additional DoF provided by the frequency selectivity of the ideal IRS would be unnecessary as the the optimal reflection coefficients of each element would almost align at all subbands. Similarly, WIT-optimized and WPT-optimized IRS boil down to optimizing a single term representing the composite channel response at all subbands. Therefore, all IRS strategies coincide with each other, and the optimal IRS for narrowband SISO SWIPT can be approximated by any candidate that roughly align the AP-IRS-user channel with the AP-user channel simultaneously over all subbands. On the other hand, the channel frequency selectivity becomes significant for $B=10\,\si{\MHz}$, and the ideal FS IRS outperforms the others as it requires no tradeoff among subchannels. Moreover, the WIT-optimized IRS tends to equalize the channel strength for all subbands as possible, due to the preference of WF strategy at high SNR. In comparison, the WPT-optimized IRS beams less towards few weakest subchannels, due to the inefficiency of small-amplitude tones in energy harvesting. Therefore, the adaptive IRS design is more suitable for broadband SWIPT. Note that the tradeoff for practical IRS indeed origins from frequency selectivity rather than number of subbands.
	\end{section}


	\bibliographystyle{IEEEtran}
	\bibliography{IEEEabrv,library.bib}
\end{document}
