\documentclass{IEEEtran}

\title{IRS-aided SWIPT}
\author{Yang Zhao}
\date{\today}

\usepackage{amsfonts}
\usepackage{amsmath}
\usepackage{amssymb}
\usepackage{cuted}
\usepackage{hyperref}
\usepackage{mathtools}
\usepackage{siunitx}
\usepackage{stfloats}
% \usepackage{nidanfloat}


\begin{document}

\begin{section}{System Model}
	Consider an IRS-aided multiuser SISO SWIPT system where the IRS not only assists the primal transmission but also retrieves Channel State Information (CSI) and harvests energy for its own operation. The single-antenna transmitter delivers information and power simultaneously, through the $L$-reflector IRS, to $K$ single-antenna users over $N$ orthogonal subbands. Assume a total bandwidth $B$ and evenly-spaced carriers around center frequency $f_0$. Denote the frequency of the $n$-th subband as $f_n$ ($n=1,\dots,N$). Suppose each subband is allocated to one user per time slot. It is assumed that the IRS performs channel estimation in the first subframe and supports information and power transfer in the second subframe \cite{Zheng2019}. Due to the passive characteristics of IRS, we consider a Time-Division Duplexing (TDD) protocol where the CSI can be obtained by exploiting channel reciprocity. Perfect CSI is assumed at the AP and IRS to investigate the analytical upper-bound of the proposed system. A quasi-static frequency-selective model is used for both the AP-user and AP-IRS-user links where the channels are assumed unchanged within each transmission frame. The signals reflected by IRS for two and more times are assumed negligible and thus not considered. Note that although Frequency Selective Surface (FSS) has received much attention for wideband communications, active FSS requires RF-chains thus becomes prohibitive in IRS \cite{Kim2006,Xu2014}. Since passive FSS is not reconfigurable with fixed physical characteristics \cite{Anwar2018}, we assume a frequency-flat IRS with the same reflection coefficients for all subbands. Since a deterministic multisine waveform can boost the energy transfer efficiency \cite{Clerckx2016a} and creates no interference to the information signal \cite{Clerckx2018b}, we use a superposition of multicarrier modulated and unmodulated waveforms, both transmitted on the same frequency bands, to maximize the rate-energy tradeoff. Two practical receiver architectures proposed in \cite{Zhang2013}, namely Time Switching (TS) and Power Splitting (PS), are investigated for the co-located information decoder and energy harvester. In the TS strategy, each transmission subframe is further divided into orthogonal data and energy slots, with duration ratio $(1-\lambda)$ and $\lambda$ respectively. Hence, the achievable rate-energy region can be obtained through a time sharing between wireless power transfer (WPT) with $\lambda=1$ and wireless information transfer (WIT) with $\lambda=0$. The adjustment of $\lambda$ has no impact on the transmit waveform and IRS elements design as they are optimized individually in data and energy slots. In comparison, the PS scheme splits the received signal into data and energy streams with power ratio $(1-\rho)$ and $\rho$ such that the PS ratio is coupled with waveform design. Perfect synchronization is assumed among the three parties in both scenarios.

	\begin{subsection}{Transmit Signal}
		Denote $\tilde{x}_{I,n}(t)$ as the information symbol transmitted over subband $n$, which belongs to one user at time $t$ and follows a capacity-achieving i.i.d. Circular Symmetric Complex Gaussian (CSCG) distribution $\tilde{x}_{I,n}\sim\mathcal{CN}(0,1)$. Let $\alpha_{k,n}$ be the allocation indicator, namely if subband $n$ is given to user $k$ ($k=1,\dots,K$), we have $\alpha_{k,n}=1$ and $\alpha_{k',n}=0 \ \forall k' \ne k$. Let $\boldsymbol{\alpha}_k=[\alpha_{k,1},\dots,\alpha_{k,N}]^T \in \mathbb{C}^{N \times 1}$. The superposed transmit signal at time $t$ is
		\begin{equation}\label{eq:x}
			x(t)=\Re\left\{\sum_{n=1}^N\left({w_{I,n}\tilde{x}_{I,n}(t)}+w_{P,n}\right){e^{j2{\pi}{f_n}{t}}}\right\}
		\end{equation}
		where $w_{I/P,n}=s_{I/P,n}e^{j\psi_{I/P,n}}$ collects the magnitude and phase of the information and power signal at frequency $n$. We further define the waveform vectors $\boldsymbol{w}_{I/P}=[w_{I/P,1},\dots,w_{I/P,N}]^T \in \mathbb{C}^{N{\times}1}$.
	\end{subsection}

	\begin{subsection}{Composite Channel Model}
		Denote the frequency response of the AP-user $k$ direct link as $\boldsymbol{h}_{D,k}=[h_{D,k,1},\dots,h_{D,k,N}]^T \in \mathbb{C}^{N \times 1}$. Let $[\boldsymbol{h}_{I,1},\dots,\boldsymbol{h}_{I,N}] \in \mathbb{C}^{L \times N}$ be the frequency response of AP-IRS incident channel, where $\boldsymbol{h}_{I,n} \in \mathbb{C}^{L \times 1}$ corresponds to the $n$-th incident channel. Similarly, let $[\boldsymbol{h}_{R,k,1},\dots,\boldsymbol{h}_{R,k,N}]^H \in \mathbb{C}^{N \times L}$ be the frequency response of IRS-user $k$ reflective channel, where $\boldsymbol{h}_{R,k,n}^H \in \mathbb{C}^{1 \times L}$ corresponds to the $n$-th reflective channel. At the IRS, element $l$ ($l=1,\dots,L$) redistributes the received signal by adjusting the amplitude reflection coefficient $\beta_l \in [0,1]$ and phase shift $\theta_l \in [0,2\pi)$ \footnote{To investigate the performance upper bound of IRS, we suppose the reflection coefficient is maximized $\beta_l=1 \ \forall l$ while the phase shift is a continuous variable over $[0,2\pi)$.}. On top of this, the IRS matrix is constructed by collecting the reflection coefficients onto the main diagonal entries as $\boldsymbol{\Theta} = \mathrm{diag}\left\{\beta_1 e^{j \theta_1}, \dots, \beta_L e^{j \theta_L}\right\} \in \mathbb{C}^{L \times L}$. The IRS-aided link can be modeled as a concatenation of the AP-IRS channel, IRS reflection, and IRS-user $k$ channel, which for user $k$ over subband $n$ is
		\begin{equation}\label{eq:h_{E,k,n}}
			h_{E,k,n} = \boldsymbol{h}_{R,k,n}^H \boldsymbol{\Theta} \boldsymbol{h}_{I,n} = \boldsymbol{v}_{k,n}^H \boldsymbol{\phi}
		\end{equation}
		where $\boldsymbol{v}_{k,n}^H=\boldsymbol{h}_{R,k,n}^H \mathrm{diag}(\boldsymbol{h}_{I,n}) \in \mathbb{C}^{1 \times L}$ and $\boldsymbol{\phi}=[e^{j{\theta_1}},\dots,e^{j{\theta_L}}] \in \mathbb{C}^{L \times 1}$. Both direct and extra link contributes to the corresponding composite channel as
		\begin{equation}
			h_{k,n} = A_{k,n} e^{j\bar{\psi}_{k,n}}= h_{D,k,n} + \boldsymbol{v}_{k,n}^H \boldsymbol{\phi}
		\end{equation}
		where $A_{k,n}$ and $\bar{\psi}_{k,n}$ are the amplitude and phase of the composite channel of user $k$ at subband $n$. Let $\boldsymbol{V}_k^H=[\boldsymbol{v}_{k,1},\dots,\boldsymbol{v}_{k,N}]^H \in \mathbb{C}^{N \times L}$, the extra link for user $k$ is $\boldsymbol{h}_{E,k}=[h_{E,k,1},\dots,h_{E,k,N}]^T=\boldsymbol{V}_k^H \boldsymbol{\phi} \in \mathbb{C}^{N \times 1}$. Therefore, the composite channel of user $k$ is
		\begin{equation}\label{eq:h_k}
			\boldsymbol{h}_k = \boldsymbol{h}_{D,k} + \boldsymbol{V}_k^H \boldsymbol{\phi}
		\end{equation}
	\end{subsection}

	\begin{subsection}{Receive Signal}
		The RF signal received by user $k$ captures the contribution of information and power waveforms through both direct and IRS-aided links as
		\begin{equation}\label{eq:y_k}
			y_k(t)=\Re\left\{\sum_{n=1}^N{h_{{k,n}}}\left({w_{I,n}\tilde{x}_{I,n}(t)}+w_{P,n}\right){e^{j2{\pi}{f_n}{t}}}\right\}
		\end{equation}
		which can be divided into
		\begin{align}\label{eq:y_{I/P,k}}
			y_{I,k}(t) & = \Re\left\{\sum_{n=1}^N{h_{{k,n}}}{w_{I,n}\tilde{x}_{I,n}(t)}{e^{j2{\pi}{f_n}{t}}}\right\}\\
			y_{P,k}(t) & = \Re\left\{\sum_{n=1}^N{h_{{k,n}}}w_{P,n}{e^{j2{\pi}{f_n}{t}}}\right\}
		\end{align}
	\end{subsection}

	\begin{subsection}{Information Decoder}
		% ? OFDM interference
		A major benefit of the superposed waveform is that the power component $y_{P,k}(t)$ creates no interference to the information component $y_{I,k}(t)$. Hence, the achievable rate of user $k$ is
		\begin{equation}\label{eq:R_k}
			R_k(\boldsymbol{w}_I,\boldsymbol{\phi},\rho,\boldsymbol{\alpha}_k)=\sum_{n=1}^N\alpha_{k,n}{\log_2\left(1+\frac{(1-\rho)\lvert h_{k,n}w_{I,n} \rvert^2}{\sigma_n^2}\right)}
		\end{equation}
		where $\sigma_n^2$ is the variance of the noise at RF band and during RF-to-BB conversion on tone $n$. Rate \ref{eq:R_k} is achievable with either waveform cancellation or translated demodulation \cite{Clerckx2018b}.
	\end{subsection}

	\begin{subsection}{Energy Harvester}
		Consider a nonlinear diode model based on the Taylor expansion of a small signal model \cite{Clerckx2016a,Clerckx2018b}, which highlights the dependency of harvester output DC current on the received waveform of user $k$ as
		\begin{equation}\label{eq:i_k}
			i_k(\boldsymbol{w}_I,\boldsymbol{w}_P,\boldsymbol{\phi},\rho)\approx\sum_{i=0}^{\infty}{k_i'}{\rho^{i/2}}{R_{\text{ant}}^{i/2}}\mathcal{E}\left\{{\mathcal{A}\left\{y_k(t)^i\right\}}\right\}
		\end{equation}
		where $R_{\text{ant}}$ is the impedance of the receive antenna, $k_0'=i_s(e^{-i_kR_{\text{ant}}/nv_t}-1)$, $k_i'=i_se^{-i_kR_{\text{ant}}/nv_t}/i!(nv_t)^i$ for $i=1,\dots,\infty$, $i_s$ is saturation current, $n$ is diode ideality factor, $v_t$ is thermal voltage. For a fixed channel and waveform, $\mathcal{A}\left\{.\right\}$ extracts the DC component of the received signal while $\mathcal{E}\left\{.\right\}$ covers the expectation over $\tilde{x}_{I,k,n}$.

		With the assumption of evenly spaced frequencies, we have $\mathcal{E}\left\{y_k(t)^i\right\}=0 \ \forall i \ \mathrm{odd}$ such that the related terms has no contribution to DC components. For simplicity, we truncate the infinite series to the $n_0$-th order. Maximizing a truncated \ref{eq:i_k} is equivalent to maximizing a monotonic function \cite{Clerckx2016a}
		\begin{equation}\label{eq:z_k}
			z_k(\boldsymbol{w}_I,\boldsymbol{w}_P,\boldsymbol{\phi},\rho)=\sum_{i\,\text{even},i\ge2}^{n_0}{k_i}{\rho^{i/2}}{R_{\text{ant}}^{i/2}}{\mathcal{E}\left\{\mathcal{A}\left\{y_k(t)^i\right\}\right\}}
		\end{equation}
		where $k_i=i_s/i!(nv_t)^i$. We choose $n_0=4$ to investigate the fundamental impact of diode nonlinearity on waveform design. Nonzero terms in \ref{eq:z_k} are given in \ref{eq:z_k_terms_begin} -- \ref{eq:z_k_terms_end}. Note that $\mathcal{E}\left\{\lvert\tilde{x}_{I,n}\rvert^2\right\}=1$ and $\mathcal{E}\left\{\lvert\tilde{x}_{I,n}\rvert^4\right\}=2$, which can be interpreted as a modulation gain on the nonlinear terms of the output DC current. On top of this, the current expression \ref{eq:z_k} reduces to \ref{eq:z_k_expand}.
		\begin{figure*}[b]
			\hrule
			\begin{align}
				\mathcal{E}\left\{\mathcal{A}\left\{y_{I,k}^2(t)\right\}\right\}
				& = \frac{1}{2}\sum_n{(h_{k,n}w_{I,n})(h_{k,n}w_{I,n})^*}\label{eq:z_k_terms_begin}\\
				& = \frac{1}{2}\boldsymbol{h}_k^H\boldsymbol{W}_{I,0}\boldsymbol{h}_k = \frac{1}{2}\boldsymbol{w}_I^H\boldsymbol{H}_{k,0}\boldsymbol{w}_I\\
				\mathcal{E}\left\{\mathcal{A}\left\{y_{I,k}^4(t)\right\}\right\}
				& = \frac{3}{4}\sum_{\substack{{n_1},{n_2},{n_3},{n_4}\\{n_1}+{n_2}={n_3}+{n_4}}}{(h_{k,{n_1}}w_{I,{n_1}})(h_{k,{n_2}}w_{I,{n_2}})(h_{k,{n_3}}w_{I,{n_3}})^*(h_{k,{n_4}}w_{I,{n_4}})^*}\\
				& = \frac{3}{4}\sum_{n=-N+1}^{N-1}(\boldsymbol{h}_k^H\boldsymbol{W}_{I,n}^*\boldsymbol{h}_k)(\boldsymbol{h}_k^H\boldsymbol{W}_{I,n}^*\boldsymbol{h}_k)^* = \frac{3}{4}\sum_{n=-N+1}^{N-1}(\boldsymbol{w}_I^H\boldsymbol{H}_{k,n}^*\boldsymbol{w}_I)(\boldsymbol{w}_I^H\boldsymbol{H}_{k,n}^*\boldsymbol{w}_I)^*\\
				\mathcal{A}\left\{y_{P,k}^2(t)\right\}
				& = \frac{1}{2}\sum_n{(h_{k,n}w_{P,n})(h_{k,n}w_{P,n})^*}\\
				& = \frac{1}{2}\boldsymbol{h}_k^H\boldsymbol{W}_{P,0}\boldsymbol{h}_k = \frac{1}{2}\boldsymbol{w}_P^H\boldsymbol{H}_{k,0}\boldsymbol{w}_P\\
				\mathcal{A}\left\{y_{P,k}^4(t)\right\}
				& = \frac{3}{8}\sum_{\substack{{n_1},{n_2},{n_3},{n_4}\\{n_1}+{n_2}={n_3}+{n_4}}}{(h_{k,{n_1}}w_{P,{n_1}})(h_{k,{n_2}}w_{P,{n_2}})(h_{k,{n_3}}w_{P,{n_3}})^*(h_{k,{n_4}}w_{P,{n_4}})^*}\\
				& = \frac{3}{8}\sum_{n=-N+1}^{N-1}(\boldsymbol{h}_k^H\boldsymbol{W}_{P,n}^*\boldsymbol{h}_k)(\boldsymbol{h}_k^H\boldsymbol{W}_{P,n}^*\boldsymbol{h}_k)^* = \frac{3}{4}\sum_{n=-N+1}^{N-1}(\boldsymbol{w}_P^H\boldsymbol{H}_{k,n}^*\boldsymbol{w}_P)(\boldsymbol{w}_P^H\boldsymbol{H}_{k,n}^*\boldsymbol{w}_P)^*\label{eq:z_k_terms_end}
			\end{align}
		\end{figure*}
		\begin{figure*}[b]
			\hrule
			\begin{align}
				z_k(\boldsymbol{w}_I,\boldsymbol{w}_P,\boldsymbol{\phi},\rho)
				& = \beta_2\left(\mathcal{E}\left\{\mathcal{A}\left\{y_{I,k}^2(t)\right\}\right\}+\mathcal{A}\left\{y_{P,k}^2(t)\right\}\right)+\beta_4\left(\mathcal{E}\left\{\mathcal{A}\left\{y_{I,k}^4(t)\right\}\right\}+\mathcal{A}\left\{y_{P,k}^4(t)\right\}+6\mathcal{E}\left\{\mathcal{A}\left\{y_{I,k}^2(t)\right\}\right\}\mathcal{A}\left\{y_{P,k}^2(t)\right\}\right)\label{eq:z_k_expand}\\
				& = \frac{1}{2}\beta_2(\boldsymbol{h}_k^H\boldsymbol{W}_{I,0}\boldsymbol{h}_k+\boldsymbol{h}_k^H\boldsymbol{W}_{P,0}\boldsymbol{h}_k)\nonumber\\
				& \quad+ \frac{3}{8}\beta_4\sum_{n=-N+1}^{N-1}\left(2(\boldsymbol{h}_k^H\boldsymbol{W}_{I,n}^*\boldsymbol{h}_k)(\boldsymbol{h}_k^H\boldsymbol{W}_{I,n}^*\boldsymbol{h}_k)^* + (\boldsymbol{h}_k^H\boldsymbol{W}_{P,n}^*\boldsymbol{h}_k)(\boldsymbol{h}_k^H\boldsymbol{W}_{P,n}^*\boldsymbol{h}_k)^* \right)\nonumber\\
				& \quad+ \frac{3}{2}\beta_4(\boldsymbol{h}_k^H\boldsymbol{W}_{I,0}\boldsymbol{h}_k)(\boldsymbol{h}_k^H\boldsymbol{W}_{P,0}\boldsymbol{h}_k)\\
				& = \frac{1}{2}\beta_2(\boldsymbol{w}_I^H\boldsymbol{H}_{k,0}\boldsymbol{w}_I+\boldsymbol{w}_P^H\boldsymbol{H}_{k,0}\boldsymbol{w}_P)\nonumber\\
				& \quad+ \frac{3}{8}\beta_4\sum_{n=-N+1}^{N-1}\left(2(\boldsymbol{w}_I^H\boldsymbol{H}_{k,n}^*\boldsymbol{w}_I)(\boldsymbol{w}_I^H\boldsymbol{H}_{k,n}^*\boldsymbol{w}_I)^* + (\boldsymbol{w}_P^H\boldsymbol{H}_{k,n}^*\boldsymbol{w}_P)(\boldsymbol{w}_P^H\boldsymbol{H}_{k,n}^*\boldsymbol{w}_P)^* \right)\nonumber\\
				& \quad+ \frac{3}{2}\beta_4(\boldsymbol{w}_I^H\boldsymbol{H}_{k,0}\boldsymbol{w}_I)(\boldsymbol{w}_P^H\boldsymbol{H}_{k,0}\boldsymbol{w}_P)
			\end{align}
		\end{figure*}
	\end{subsection}
\end{section}

\begin{section}{Problem Formulation}
	In this section, we first define the weighted sum rate-energy (WSR-E) region then formulate waveform and IRS optimization problems.
	\begin{subsection}{Weighted Sum Rate-Energy Region}
		We define the achievable WSR-E region as
		\begin{equation}
			\begin{split}
				C_{R-I}(P)
				&\triangleq \biggl\{(R,I):R\le\sum_{k=1}^K{u_{I,k}R_k},I\le\sum_{k=1}^K u_{P,k}z_k,\\
				&\quad \frac{1}{2}({\boldsymbol{w}_I^H}{\boldsymbol{w}_I}+{\boldsymbol{w}_P^H}{\boldsymbol{w}_P}){\le}P\biggr\}
				% &\quad \frac{1}{2}\left(\Vert{\boldsymbol{W}_I}\Vert_F^2+\Vert{\boldsymbol{W}_P}\Vert_F^2\right){\le}P\biggr\}
			\end{split}
		\end{equation}
		where $u_{I,k},u_{P,k}$ denote the information and power weight of user $k$.
	\end{subsection}

	\begin{subsection}{Single-User Optimization}
		In this section, we first characterize the rate-energy region in the single-user setup. Let $h_n={A_n}{e^{j{\bar{\psi}_n}}}$, $w_{I,n}=s_{I,n}e^{j{\phi}_n}$, $w_{P,n}=s_{P,n}e^{j{\phi}_n}$. The WSR-E region can be obtained through a rate maximization problem subject to transmit power and output DC current constraints
		\begin{maxi}
			% {\boldsymbol{w}_I,\boldsymbol{w}_P,\boldsymbol{\Theta}_0,\rho}{\sum_{n}{\log_2\left(1+\frac{(1-\rho)\lvert{(h_{D,n}+\boldsymbol{h}_{R,n}\boldsymbol{\Theta}_0\boldsymbol{h}_{I,n})w_{I,n}}\rvert^2}{\sigma_n^2}\right)}}{\label{op:1}}{}
			{\boldsymbol{w}_I,\boldsymbol{w}_P,\boldsymbol{v},\rho}{\sum_{n}{\log_2\left(1+\frac{(1-\rho)\lvert(h_{D,n}+\boldsymbol{\Phi}_n^H\boldsymbol{v})w_{I,n}\rvert^2}{\sigma_n^2}\right)}}{\label{op:1}}{}
			\addConstraint{\frac{1}{2}({\boldsymbol{w}_I^H}{\boldsymbol{w}_I}+{\boldsymbol{w}_P^H}{\boldsymbol{w}_P})\le{P}}
			\addConstraint{z(\boldsymbol{w}_I,\boldsymbol{w}_P,\boldsymbol{v},\rho)\ge{\bar{z}}}
			\addConstraint{\lvert{v_l}\rvert=1,\quad l=1,\dots,L}
			% \addConstraint{0\le{\theta_l}\le{2{\pi}},\quad l=1,\dots,L}
		\end{maxi}
		Problem \ref{op:1} is intricate due to the non-convex rate function, unit-modulus constraint, and non-convex DC current requirement with coupled variables.

		\begin{subsubsection}{Frequency-Selective IRS}
			In frequency-selective IRS, each element is expected to provide adjustable subband-dependent reflection coefficients such that the total degree of freedom (DoF) of IRS is $NL$. Therefore, $\boldsymbol{v}_n$ replaces $\boldsymbol{v}$ in \ref{op:1}. Note that $\lvert{(h_{D,n}+\boldsymbol{\Phi}_n^H\boldsymbol{v}_n)w_{I,n}}\rvert \le \lvert{h_{D,n}w_{I,n}}\rvert+\lvert{\boldsymbol{\Phi}_n^H\boldsymbol{v}_nw_{I,n}}\rvert$ where the equality holds if and only if the direct and IRS-aided links are aligned by
			\begin{equation}
				\theta_{n,l}^\star = \angle{h}_{D,n} - \angle{h_{R,n,l}}-\angle{h_{I,n,l}}
			\end{equation}
			That is to say, the optimal phase shift is obtained in closed form in the single-user scenario. On top of this, it can be observed from \ref{eq:gamma_{k,n}}, \ref{eq:R_k} and \ref{eq:z_k_truncated} that the optimal phase of information and power waveforms are matched to composite frequency response as
			\begin{equation}\label{eq:phi_n}
				\phi_{I,n}^\star=\phi_{P,n}^\star=-\bar{\psi}_n
			\end{equation}
			With all the phases determined, the original problem \ref{op:1} is reduced to
			\begin{maxi}
				% {\boldsymbol{s}_I,\boldsymbol{s}_P,\rho}{\sum_{n}{\log_2\left(1+\frac{(1-\rho)(A_{D,n}+\boldsymbol{A}_{R,n}\boldsymbol{A}_{I,n})^2{s_{I,n}^2}}{\sigma_n^2}\right)}}{\label{op:2}}{}
				{\boldsymbol{s}_I,\boldsymbol{s}_P,\rho}{\sum_{n}{\log_2\left(1+\frac{(1-\rho)A_n^2{s_{I,n}^2}}{\sigma_n^2}\right)}}{\label{op:2}}{}
				\addConstraint{\frac{1}{2}({\boldsymbol{s}_I^H}{\boldsymbol{s}_I}+{\boldsymbol{s}_P^H}{\boldsymbol{s}_P})\le{P}}
				\addConstraint{z(\boldsymbol{s}_I,\boldsymbol{s}_P,\rho)\ge{\bar{z}}}
			\end{maxi}
			\begin{figure*}[b]
				\hrule
				\begin{equation}\label{eq:z_gp}
					\begin{split}
						z
						&=\frac{1}{2}{k_2}{\rho}{R_{\text{ant}}}\sum_n{A_n^2(s_{I,n}^2+s_{P,n}^2)}+\frac{3}{8}{k_4}{\rho^2}{R_{\text{ant}}^2}\sum_{\substack{{n_1},{n_2},{n_3},{n_4}\\{n_1}+{n_2}={n_3}+{n_4}}}{A_{n_1}A_{n_2}A_{n_3}A_{n_4}(2s_{I,n_1}s_{I,n_2}s_{I,n_3}s_{I,n_4}+s_{P,n_1}s_{P,n_2}s_{P,n_3}s_{P,n_4})}\\
						&\quad+\frac{3}{2}{k_4}{\rho^2}{R_{\text{ant}}^2}\sum_n{{A_n^4}{s_{I,n}^2}{s_{P,n}^2}}
					\end{split}
				\end{equation}
			\end{figure*}
			with $z$ given by \ref{eq:z_gp}. It can be transformed to an equivalent problem by introducing an auxiliary variable $t_0$
			\begin{mini}
				{\boldsymbol{s}_I,\boldsymbol{s}_P,\rho}{\frac{1}{t_0}}{\label{op:3}}{}
				\addConstraint{\frac{1}{2}({\boldsymbol{s}_I^H}{\boldsymbol{s}_I}+{\boldsymbol{s}_P^H}{\boldsymbol{s}_P})\le{P}}
				\addConstraint{\frac{t_0}{\prod_{n}{\left(1+\frac{(1-\rho){A_n^2}{s_{I,n}^2}}{\sigma_n^2}\right)}}\le{1}}
				\addConstraint{\frac{\bar{z}}{z(\boldsymbol{s}_I,\boldsymbol{s}_P,\rho)}\le{1}}
			\end{mini}
			\ref{op:3} is a Reversed Geometric Program which can be transformed to standard Geometric Program (GP). The basic idea is to decompose the information and power posynomials as sum of monomials, then derive their upper bounds using Arithmetic Mean-Geometric Mean (AM-GM) inequality \cite{Clerckx2018b,Chiang2005}. Let $z(\boldsymbol{s}_I,\boldsymbol{s}_P,\rho)=\sum_{{m_P}=1}^{M_P}{g_{P,m_P}(\boldsymbol{s}_I,\boldsymbol{s}_P,\rho)}$, problem \ref{op:3} is equivalent to
			% $1+\bar{\rho}{A_n^2}{s_{I,n}^2}/\sigma_n^2=\sum_{{m_{I,n}}=1}^{M_{I,n}}{g_{I,m_{I,n}}(\boldsymbol{s}_I,\bar{\rho})}$
			\begin{mini}
				{\boldsymbol{s}_I,\boldsymbol{s}_P,\rho}{\frac{1}{t_0}}{\label{op:4}}{}
				\addConstraint{\frac{1}{2}({\boldsymbol{s}_I^H}{\boldsymbol{s}_I}+{\boldsymbol{s}_P^H}{\boldsymbol{s}_P})\le{P}}
				\addConstraint{{t_0}\prod_{n}{\left(\frac{1}{\gamma_{I,n,1}}\right)^{-\gamma_{I,n,1}} \left(\frac{\bar{\rho}{A_n^2}{s_{I,n}^2}}{{\sigma_n^2}{\gamma_{I,n,2}}}\right)^{-\gamma_{I,n,2}}}\le{1}}
				\addConstraint{\bar{z}\prod_{m_P}{\left(\frac{g_{P,m_P}(\boldsymbol{s}_I,\boldsymbol{s}_P,\rho)}{\gamma_{P,m_P}}\right)^{-\gamma_{P,m_P}}}\le{1}}
				\addConstraint{\rho+\bar{\rho}\le{1}}
			\end{mini}
			where $\gamma_{I,n,1},\gamma_{I,n,2} \ge 0$, $\gamma_{I,n,1}+\gamma_{I,n,2}=1$, $\gamma_{P,m_P} \ge 0,\forall m_P=1,\dots,M_P$ and $\sum_{{m_P}=1}^{M_P}{\gamma_{m_P}}=1$. As suggested in \cite{Clerckx2018b}, the tightness of the AM-GM inequality depends on $\{\gamma_{I,n},\gamma_P\}$ that require iterative update. At iteration $i$, we choose
			\begin{align}
				\gamma_{I,n,1}^{(i)} & =\frac{1}{1+\frac{\bar{\rho}^{(i-1)}{A_n^2}{s_{I,n}^{(i-1)}}^2}{\sigma_n^2}},                                                                         &  & n=1,\dots,N,    \\
				\gamma_{I,n,2}^{(i)} & =\frac{\frac{\bar{\rho}^{(i-1)}{A_n^2}{s_{I,n}^{(i-1)}}^2}{\sigma_n^2}}{1+\frac{\bar{\rho}^{(i-1)}{A_n^2}{s_{I,n}^{(i-1)}}^2}{\sigma_n^2}},           &  & n=1,\dots,N,    \\
				\gamma_{P,m_P}^{(i)} & =\frac{g_{P,m_P}(\boldsymbol{s}_I^{(i-1)},\boldsymbol{s}_P^{(i-1)},\rho^{(i-1)})}{z(\boldsymbol{s}_I^{(i-1)},\boldsymbol{s}_P^{(i-1)},\rho^{(i-1)})}, &  & m_P=1,\dots,M_P
			\end{align}
			and then solve problem \ref{op:4}.
		\end{subsubsection}

		\begin{subsubsection}{Frequency-Flat IRS}
			In contrast, frequency-flat IRS reflects all subbands equally with a DoF of $L$. We observe that
			\begin{equation}
				\begin{split}
					\lvert{h_{D,n}+\boldsymbol{\Phi}_n^H\boldsymbol{v}}\rvert^2
					&=\lvert{h_{D,n}}\rvert^2+h_{D,n}^*\boldsymbol{\Phi}_n^H\boldsymbol{v}+\boldsymbol{v}^H\boldsymbol{\Phi}_n{h_{D,n}}+\boldsymbol{v}^H\boldsymbol{\Phi}\boldsymbol{\Phi}^H\boldsymbol{v}\\
					&=\bar{\boldsymbol{v}}^H\boldsymbol{R}\bar{\boldsymbol{v}}=\mathrm{Tr}(\boldsymbol{R}_n\bar{\boldsymbol{v}}\bar{\boldsymbol{v}}^H)=\mathrm{Tr}(\boldsymbol{R}_n\boldsymbol{V})
				\end{split}
			\end{equation}
			where $t$ is an auxiliary variable with unit modulus and
			\begin{equation}
				\boldsymbol{R}_n=
				\begin{bmatrix}
					\boldsymbol{\Phi}_n\boldsymbol{\Phi}_n^H & \boldsymbol{\Phi}_n{h_{D,n}} \\
					h_{D,n}^*{\boldsymbol{\Phi}_n^H}         & h_{D,n}^*{h_{D,n}}
				\end{bmatrix},
				\quad \bar{\boldsymbol{v}}=\
				\begin{bmatrix}
					\boldsymbol{v} \\
					t
				\end{bmatrix},
				\quad \bar{\boldsymbol{V}}=\bar{\boldsymbol{v}}\bar{\boldsymbol{v}}^H
			\end{equation}
			Hence, \ref{op:1} is transformed to
			\begin{maxi}
				{\boldsymbol{w}_I,\boldsymbol{w}_P,\boldsymbol{V},\rho}{\sum_{n}{\log_2\left(1+\frac{(1-\rho)\lvert{w}_{I,n}\rvert^2\mathrm{Tr}(\boldsymbol{R}_n\boldsymbol{V})}{\sigma_n^2}\right)}}{\label{op:temp3}}{}
				\addConstraint{\frac{1}{2}({\boldsymbol{w}_I^H}{\boldsymbol{w}_I}+{\boldsymbol{w}_P^H}{\boldsymbol{w}_P})\le{P}}
				\addConstraint{z(\boldsymbol{w}_I,\boldsymbol{w}_P,\boldsymbol{V},\rho)\ge{\bar{z}}}
				\addConstraint{\boldsymbol{V}_{l,l}=1,l=1,\dots,L+1}
				\addConstraint{\boldsymbol{V}\succeq{0}}
				\addConstraint{\mathrm{rank}(\boldsymbol{V})=1}
			\end{maxi}
			To reduce the design complexity, we propose an suboptimal alternating optimization algorithm that iteratively updates the phase shifts and the waveforms with the other being fixed.

			% ? AO: optimize phase shifts with given waveform
			The phase optimization subproblem is formed as follows. For a given waveform $\boldsymbol{w}_I,\boldsymbol{w}_P,\rho$, problem \ref{op:temp3} reduces to
			\begin{maxi}
				{\boldsymbol{V}}{\sum_{n}{\log_2\left(1+\frac{(1-\rho)\lvert{w}_{I,n}\rvert^2\mathrm{Tr}(\boldsymbol{R}_n\boldsymbol{V})}{\sigma_n^2}\right)}}{\label{op:temp4}}{}
				\addConstraint{z(\boldsymbol{V})\ge{\bar{z}}}
				\addConstraint{\boldsymbol{V}_{l,l}=1,l=1,\dots,L+1}
				\addConstraint{\boldsymbol{V}\succeq{0}}
				\addConstraint{\mathrm{rank}(\boldsymbol{V})=1}
			\end{maxi}
			To tackle the non-convex current constraint, we collect the channel matrices into $\boldsymbol{\Phi}^H=[\boldsymbol{\Phi}_1,\dots,\boldsymbol{\Phi}_N]^H \in \mathbb{C}^{N \times L}$ and $\boldsymbol{h}=\boldsymbol{h}_D+\boldsymbol{\Phi}^H\boldsymbol{v} \in \mathbb{C}^{N \times 1}$. The fixed waveform matrices are defined as $\boldsymbol{M}_{I/P}=\boldsymbol{w}_{I/P}^*\boldsymbol{w}_{I/P}^T$. On top of this, let $\boldsymbol{M}_{I/P, n}$ keep the $n$-th ($n=-N+1,\dots,N-1$) diagonal of $\boldsymbol{M}_{I/P}$ and null the remaining entries. Due to the positive definiteness of $\boldsymbol{M}_{I/P}$, we have $\boldsymbol{M}_{I/P,-n}=\boldsymbol{M}_{I/P,n}^H$. Hence, the output current expression \ref{eq:z_k_truncated} is transformed to \ref{eq:z_k_irs}.
			\begin{figure*}[b]
				\hrule
				\begin{equation}\label{eq:z_k_irs}
					\begin{split}
						z
						&=\frac{1}{2}{k_2}{\rho}{R_{\text{ant}}}(\boldsymbol{h}^H\boldsymbol{M}_{I,0}\boldsymbol{h}+\boldsymbol{h}^H\boldsymbol{M}_{P,0}\boldsymbol{h})\\
						% &\quad+\frac{3}{8}{k_4}{\rho^2}{R_{\text{ant}}^2}\left({\boldsymbol{w}_I^H\boldsymbol{M}_0}\boldsymbol{w}_I({\boldsymbol{w}_I^H\boldsymbol{M}_0}\boldsymbol{w}_I)^H+{\boldsymbol{w}_P^H\boldsymbol{M}_0}\boldsymbol{w}_P({\boldsymbol{w}_P^H\boldsymbol{M}_0}\boldsymbol{w}_P)^H\right)
						% &\quad+\frac{3}{4}{k_4}{\rho^2}{R_{\text{ant}}^2}\left({\boldsymbol{w}_I^H\boldsymbol{M}_n}\boldsymbol{w}_I({\boldsymbol{w}_I^H\boldsymbol{M}_0}\boldsymbol{w}_I)^H+{\boldsymbol{w}_P^H\boldsymbol{M}_0}\boldsymbol{w}_P({\boldsymbol{w}_P^H\boldsymbol{M}_0}\boldsymbol{w}_P)^H\right)
						&\quad+\frac{3}{8}{k_4}{\rho^2}{R_{\text{ant}}^2}\sum_{n=-N+1}^{N-1}{2(\boldsymbol{h}^H\boldsymbol{M}_{I,n}\boldsymbol{h})(\boldsymbol{h}^H\boldsymbol{M}_{I,n}\boldsymbol{h})^*+(\boldsymbol{h}^H\boldsymbol{M}_{P,n}\boldsymbol{h})(\boldsymbol{h}^H\boldsymbol{M}_{P,n}\boldsymbol{h})^*}\\
						&\quad+\frac{3}{2}{k_4}{\rho^2}{R_{\text{ant}}^2}{(\boldsymbol{h}^H\boldsymbol{M}_{I,0}\boldsymbol{h})(\boldsymbol{h}^H\boldsymbol{M}_{P,0}\boldsymbol{h})}
					\end{split}
				\end{equation}
			\end{figure*}
			We further simplify it by introducing auxiliary variables
			\begin{equation}\label{eq:t}
				\begin{split}
					t_{I/P,n}
					&=\boldsymbol{h}^H\boldsymbol{M}_{I/P,n}\boldsymbol{h}\\
					&=\mathrm{Tr}(\boldsymbol{h}\boldsymbol{h}^H\boldsymbol{M}_{I/P,n})\\
					&=\mathrm{Tr}(\boldsymbol{R}^H\boldsymbol{V}\boldsymbol{R}\boldsymbol{M}_{I/P,n})\\
					&=\mathrm{Tr}(\boldsymbol{R}\boldsymbol{M}_{I/P,n}\boldsymbol{R}^H\boldsymbol{V})\\
					&=\mathrm{Tr}(\boldsymbol{C}_{I/P,n}\boldsymbol{V})
				\end{split}
			\end{equation}
			where $\boldsymbol{R}^H=[\boldsymbol{\Phi}^H,\boldsymbol{h}_D] \in \mathbb{C}^{N \times (L+1)}$ and $\boldsymbol{C}_{I/P,n}=\boldsymbol{R}\boldsymbol{M}_{I/P,n}\boldsymbol{R}^H \in \mathbb{C}^{(L+1)\times(L+1)}$. Therefore, \ref{eq:z_k_irs} rewrites as
			\begin{equation}\label{eq:z_k_irs_t}
				\begin{split}
					z
					&=\frac{1}{2}{k_2}{\rho}{R_{\text{ant}}}(t_{I,0}+t_{P,0})\\
					&\quad+\frac{3}{8}{k_4}{\rho^2}{R_{\text{ant}}^2}\sum_{n=-N+1}^{N-1}{2t_{I,n}t_{I,n}^*+t_{P,n}t_{P,n}^*}\\
					&\quad+\frac{3}{2}{k_4}{\rho^2}{R_{\text{ant}}^2}t_{I,0}t_{P,0}
				\end{split}
			\end{equation}

			% It holds that $t_{I/P,n}^*t_{I/P,n}=t_{I/P,-n}^*t_{I/P,-n}$ since $\boldsymbol{M}_{I/P,-n}=\boldsymbol{M}_{I/P,n}^H$. On top of this, denote $\boldsymbol{t}_{I/P}=[t_{I/P,1},\dots,t_{I/P,N-1}]^T \in \mathbb{C}^{(N-1) \times 1}$ and rewrite \ref{eq:z_k_irs} as
			% \begin{equation}\label{eq:z_k_irs_t}
			% 	\begin{split}
			% 		z
			% 		&=\frac{1}{2}{k_2}{\rho}{R_{\text{ant}}}(t_{I,0}+t_{P,0})\\
			% 		&\quad+\frac{3}{8}{k_4}{\rho^2}{R_{\text{ant}}^2}(2t_{I,0}^*t_{I,0}+t_{P,0}^*t_{P,0}+4\boldsymbol{t}_I^H\boldsymbol{t}_I+2\boldsymbol{t}_P^H\boldsymbol{t}_P)\\
			% 		&\quad+\frac{3}{2}{k_4}{\rho^2}{R_{\text{ant}}^2}t_{I,0}t_{P,0}
			% 	\end{split}
			% \end{equation}
			We use first-order Taylor expansion to approximate the second-order terms in \ref{eq:z_k_irs_t}. Based on the variables optimized at iteration $k - 1$, the local approximations at iteration $k$ are \cite{Adali2010}
			\begin{equation}
				\begin{split}
					t_{I/P,n}^{(k)} (t_{I/P,n}^{(k)})^*
					& \approx 2 \Re\left\{t_{I/P,n}^{(k)} (t_{I/P,n}^{(k-1)})^*\right\} - t_{I/P,n}^{(k-1)} (t_{I/P,n}^{(k-1)})^* \\
					& = 2 \Re \left\{\mathrm{Tr}\left((t_{I/P,n}^{(k-1)})^*\boldsymbol{C}_{I/P,n}\boldsymbol{V}^{(k)}\right)\right\} - t_{I/P,n}^{(k-1)} (t_{I/P,n}^{(k-1)})^*\\
					& = \mathrm{Tr}\left((t_{I/P,n}^{(k-1)})^*\boldsymbol{C}_{I/P,n}\boldsymbol{V}^{(k)}\right) + \mathrm{Tr}(t_{I/P,n}^{(k-1)}\boldsymbol{C}_{I/P,n}^H\boldsymbol{V}^{(k)})\\
					& \quad- t_{I/P,n}^{(k-1)} (t_{I/P,n}^{(k-1)})^*
				\end{split}
			\end{equation}
			and
			\begin{equation}
				\begin{split}
					t_{I,0}^{(k)} t_{P,0}^{(k)}
					& \approx t_{I,0}^{(k)} t_{P,0}^{(k-1)} + t_{I,0}^{(k-1)} t_{P,0}^{(k)} - t_{I,0}^{(k-1)} t_{P,0}^{(k-1)}\\
					& = \mathrm{Tr}(t_{P,0}^{(k-1)}\boldsymbol{C}_{I,0}\boldsymbol{V}^{(k)}) + \mathrm{Tr}(t_{I,0}^{(k-1)}\boldsymbol{C}_{P,0}\boldsymbol{V}^{(k)}) - t_{I,0}^{(k-1)} t_{P,0}^{(k-1)}
				\end{split}
			\end{equation}
			The second approximation holds since $t_{I/P,0}$ are real. Therefore, we have
			\begin{equation}\label{eq:z_irs_approx}
				\begin{split}
					\tilde{z}(\boldsymbol{V}^{(k)})
					& = \mathrm{Tr}(\boldsymbol{A}^{(k)}\boldsymbol{V}^{(k)}) \\
					& - \frac{3 k_4 \rho^2 R_{\text{ant}}^2}{8} \sum_{n=-N+1}^{N-1} 2t_{I,n}^{(k-1)} (t_{I,n}^{(k-1)})^* + t_{P,n}^{(k-1)} (t_{P,n}^{(k-1)})^* \\
					& - \frac{3{k_4}{\rho^2}{R_{\text{ant}}^2}}{2} t_{I,0}^{(k-1)} t_{P,0}^{(k-1)}
				\end{split}
			\end{equation}
			where the Hermitian matrix $\boldsymbol{A}^{(k)}$ is
			\begin{equation}\label{eq:A_k}
				\begin{split}
					\boldsymbol{A}^{(k)}
					& = \frac{k_2 \rho R_{\text{ant}}}{2}(\boldsymbol{C}_{I,0}+\boldsymbol{C}_{P,0})\\
					& \quad+ \frac{3 k_4 \rho^2 R_{\text{ant}}^2}{8}\sum_{n=-N+1}^{N-1} 2\left((t_{I,n}^{(k-1)})^*\boldsymbol{C}_{I,n} + t_{I,n}^{(k-1)}\boldsymbol{C}_{I,n}^H\right)\\
					& \quad \quad + \left((t_{P,n}^{(k-1)})^*\boldsymbol{C}_{P,n} + t_{P,n}^{(k-1)}\boldsymbol{C}_{P,n}^H\right)\\
					& \quad+ \frac{3{k_4}{\rho^2}{R_{\text{ant}}^2}}{2} (t_{P,0}^{(k-1)}\boldsymbol{C}_{I,0} + t_{I,0}^{(k-1)}\boldsymbol{C}_{P,0})
				\end{split}
			\end{equation}
			Plug \ref{eq:z_irs_approx} into problem \ref{op:temp4} and solve it by iterative interior-point method. Denote the optimal IRS matrix as $\boldsymbol{V}^{\star}$. If $\mathrm{rank}(\boldsymbol{V}^{\star})=1$, the optimal phase shift vector $\bar{\boldsymbol{v}}^\star$ is attained by eigenvalue decomposition (EVD). Otherwise, a best feasible candidate $\bar{\boldsymbol{v}}^\star$ can be extracted through Gaussian randomization method \cite{Huang2010}. First, we perform EVD on $\boldsymbol{V}^{\star}$ as $\boldsymbol{V}^{\star}=\boldsymbol{U}_{\boldsymbol{V}^{\star}}\boldsymbol{\Sigma}_{\boldsymbol{V}^{\star}}\boldsymbol{U}_{\boldsymbol{V}^{\star}}^H$. Then, we generate $R$ random CSCG vectors $\boldsymbol{r}_r \sim \mathcal{CN}(\boldsymbol{0},\boldsymbol{I}_{L+1}),\ r=1,\dots,R$ and construct the corresponding candidates $\bar{\boldsymbol{v}}_r=\boldsymbol{U}_{\boldsymbol{V}^{\star}}\boldsymbol{\Sigma}_{\boldsymbol{V}^{\star}}^{\frac{1}{2}}\boldsymbol{r}_r$. Next, the optimal solution $\bar{\boldsymbol{v}}^\star$ is approximated by the one achieving maximum objective value of \ref{op:temp4}. Finally, we can retrieve the phase shift as $\theta_l=\arg(v_l^\star/v_{L+1}^\star)$. The algorithm for the phase optimization subproblem is summarized in Algorithm \ref{alg:irs}.

			\begin{algorithm}
				\caption{IRS Optimization}
				\label{alg:irs}
				\begin{algorithmic}[1]
					\State \textbf{Input} $\boldsymbol{w}_I,\boldsymbol{w}_P,\rho$
					\State \textbf{Initialize} $k=0$, $\boldsymbol{V}^{(0)}$, $t_{I/P,n}^{(0)} \ \forall n$
					\Repeat
					\State $k = k + 1$
					\State Update SDR matrix $\boldsymbol{A}^{(k)}$ by \ref{eq:A_k}
					\State Obtain IRS matrix $\boldsymbol{V}^{(k)}$ by solving problem \ref{op:temp4}
					\State Update auxiliary $t_{I/P,n}^{(k)} \forall n$ by \ref{eq:t} for SCA
					% \Until $\lVert \boldsymbol{V}^{(k)} - \boldsymbol{V}^{(k - 1)} \rVert_F / \lVert \boldsymbol{V}^{(k)} \rVert_F \le \epsilon$
					\Until $R^{(k)}-R^{(k-1)} \le \epsilon$
					\State Generate CSCG random vectors $\boldsymbol{r}_r \sim \mathcal{CN}(\boldsymbol{0},\boldsymbol{I}_{L+1}) \ \forall r$
					\State Perform EVD $\boldsymbol{V}^{\star}=\boldsymbol{U}_{\boldsymbol{V}^{\star}}\boldsymbol{\Sigma}_{\boldsymbol{V}^{\star}}\boldsymbol{U}_{\boldsymbol{V}^{\star}}^H$
					\State Construct candidate IRS vectors $\bar{\boldsymbol{v}}_r=\boldsymbol{U}_{\boldsymbol{V}^{\star}}\boldsymbol{\Sigma}_{\boldsymbol{V}^{\star}}^{\frac{1}{2}}\boldsymbol{r}_r \forall r$ and corresponding matrices $\boldsymbol{V}_r=\bar{\boldsymbol{v}}_r\bar{\boldsymbol{v}}_r^H  \ \forall r$
					\State Select the best solution $\boldsymbol{V}_r^\star$ for problem \ref{op:temp4} and corresponding $\boldsymbol{v}_r^\star$
					\State Compute IRS phase shift by $\theta_l=\arg(v_l^\star/v_{L+1}^\star) \ \forall l$
					\State \textbf{Output} $\theta_l, l = 1 , \dots L$
				\end{algorithmic}
			\end{algorithm}

			The waveform optimization subproblem is formed as follows. For a given $\boldsymbol{V}$, problem \ref{op:temp3} reduces to
			\begin{maxi}
				{\boldsymbol{w}_I,\boldsymbol{w}_P,\rho}{\sum_{n}{\log_2\left(1+\frac{(1-\rho)\lvert{w}_{I,n}\rvert^2\mathrm{Tr}(\boldsymbol{R}_n\boldsymbol{V})}{\sigma_n^2}\right)}}{\label{op:temp5}}{}
				\addConstraint{\frac{1}{2}({\boldsymbol{w}_I^H}{\boldsymbol{w}_I}+{\boldsymbol{w}_P^H}{\boldsymbol{w}_P})\le{P}}
				\addConstraint{z(\boldsymbol{w}_I,\boldsymbol{w}_P,\rho)\ge{\bar{z}}}
			\end{maxi}
			Similarly, define $\boldsymbol{M}=\boldsymbol{h}^*\boldsymbol{h}^T$ and let $\boldsymbol{M}_n$ keep the $n$-th ($n=-N+1,\dots,N-1$) diagonal of $\boldsymbol{M}$ and null the remaining entries. Since $\boldsymbol{M} \succ 0$, we have $\boldsymbol{M}_{-n}=\boldsymbol{M}_n^H$. Hence, \ref{eq:z_k_truncated} is transformed to \ref{eq:z_k_waveform}.
			\begin{figure*}[b]
				\hrule
				\begin{equation}\label{eq:z_k_waveform}
					\begin{split}
						z
						&=\frac{1}{2}{k_2}{\rho}{R_{\text{ant}}}(\boldsymbol{w}_I^H\boldsymbol{M}_0\boldsymbol{w}_I+\boldsymbol{w}_P^H\boldsymbol{M}_0\boldsymbol{w}_P)\\
						&\quad+\frac{3}{8}{k_4}{\rho^2}{R_{\text{ant}}^2}\sum_{n=-N+1}^{N-1}{2(\boldsymbol{w}_I^H\boldsymbol{M}_{n}\boldsymbol{w}_I)(\boldsymbol{w}_I^H\boldsymbol{M}_{n}\boldsymbol{w}_I)^*+(\boldsymbol{w}_P^H\boldsymbol{M}_{n}\boldsymbol{w}_P)(\boldsymbol{w}_P^H\boldsymbol{M}_{n}\boldsymbol{w}_P)^*}\\
						&\quad+\frac{3}{2}{k_4}{\rho^2}{R_{\text{ant}}^2}{(\boldsymbol{w}_I^H\boldsymbol{M}_{0}\boldsymbol{w}_I)(\boldsymbol{w}_P^H\boldsymbol{M}_{0}\boldsymbol{w}_P)}
					\end{split}
				\end{equation}
			\end{figure*}
			Introduce auxiliary variables
			\begin{equation}\label{eq:t'}
				\begin{split}
					t'_{I/P,n}
					& = \boldsymbol{w}_{I/P}^H\boldsymbol{M}_{n}\boldsymbol{w}_{I/P}\\
					& = \mathrm{Tr}(\boldsymbol{M}_{n}\boldsymbol{w}_{I/P}\boldsymbol{w}_{I/P}^H)\\
					& = \mathrm{Tr}(\boldsymbol{M}_{n}\boldsymbol{W}_{I/P})
				\end{split}
			\end{equation}
			where $\boldsymbol{W}_{I/P}=\boldsymbol{w}_{I/P}\boldsymbol{w}_{I/P}^H$. Therefore, \ref{eq:z_k_waveform} rewrites as
			\begin{equation}\label{eq:z_k_waveform_t}
				\begin{split}
					z
					&=\frac{1}{2}{k_2}{\rho}{R_{\text{ant}}}(t'_{I,0}+t'_{P,0})\\
					&\quad+\frac{3}{8}{k_4}{\rho^2}{R_{\text{ant}}^2}\sum_{n=-N+1}^{N-1}{2t'_{I,n}t_{I,n}^{\prime \ *}+t'_{P,n}t_{P,n}^{\prime \ *}}\\
					&\quad+\frac{3}{2}{k_4}{\rho^2}{R_{\text{ant}}^2}t'_{I,0}t'_{P,0}
				\end{split}
			\end{equation}
			At iteration $k$, the second-order terms are approximated by first-order Taylor series as
			\begin{equation}
				\begin{split}
					t_{I/P,n}^{\prime \ (k)} (t_{I/P,n}^{\prime \ (k)})^*
					& \approx 2 \Re\left\{t_{I/P,n}^{\prime \ (k)} (t_{I/P,n}^{\prime (k-1)})^*\right\} - t_{I/P,n}^{\prime (k-1)} (t_{I/P,n}^{\prime (k-1)})^* \\
					& = 2 \Re \left\{\mathrm{Tr}((t_{I/P,n}^{\prime (k-1)})^*\boldsymbol{M}_{n}\boldsymbol{W}_{I/P}^{(k)})\right\} - t_{I/P,n}^{\prime (k-1)} (t_{I/P,n}^{\prime (k-1)})^*\\
					& = \mathrm{Tr}\left((t_{I/P,n}^{\prime (k-1)})^*\boldsymbol{M}_{n}\boldsymbol{W}_{I/P}^{(k)}\right) + \mathrm{Tr}(t_{I/P,n}^{\prime (k-1)}\boldsymbol{M}_{n}^H\boldsymbol{W}_{I/P}^{(k)})\\
					& \quad- t_{I/P,n}^{\prime (k-1)} (t_{I/P,n}^{\prime (k-1)})^*
				\end{split}
			\end{equation}
			and
			\begin{equation}
				\begin{split}
					t_{I,0}^{\prime \ (k)} t_{P,0}^{\prime \ (k)}
					& \approx t_{I,0}^{\prime \ (k)} t_{P,0}^{\prime (k-1)} + t_{I,0}^{\prime (k-1)} t_{P,0}^{\prime \ (k)} - t_{I,0}^{\prime (k-1)} t_{P,0}^{\prime (k-1)}\\
					& = \mathrm{Tr}\left(t_{P,0}^{\prime (k-1)}\boldsymbol{M}_{0}\boldsymbol{W}_{I,0}^{(k)}\right) + \mathrm{Tr}(t_{I,0}^{\prime (k-1)}\boldsymbol{M}_{0}\boldsymbol{W}_{P,0}^{(k)}) - t_{I,0}^{\prime (k-1)} t_{P,0}^{\prime (k-1)}
				\end{split}
			\end{equation}
			Therefore, we have
			\begin{equation}\label{eq:z_waveform_approx}
				\begin{split}
					\tilde{z}(\boldsymbol{W}_I^{(k)},\boldsymbol{W}_P^{(k)},\rho^{(k-1)})
					& = \mathrm{Tr}(\boldsymbol{A}_I^{(k)}\boldsymbol{W}_I^{(k)}) + \mathrm{Tr}(\boldsymbol{A}_P^{(k)}\boldsymbol{W}_P^{(k)}) \\
					& - \frac{3 k_4 (\rho^{(k-1)})^2 R_{\text{ant}}^2}{8} \sum_{n=-N+1}^{N-1} 2t_{I,n}^{\prime (k-1)} (t_{I,n}^{\prime (k-1)})^* + t_{P,n}^{\prime (k-1)} (t_{P,n}^{\prime (k-1)})^* \\
					& - \frac{3{k_4}{(\rho^{(k-1)})^2}{R_{\text{ant}}^2}}{2} t_{I,0}^{\prime (k-1)} t_{P,0}^{\prime (k-1)}
				\end{split}
			\end{equation}
			where the Hermitian matrices $\boldsymbol{A}_I^{(k)}$ and $\boldsymbol{A}_P^{(k)}$ are
			\begin{equation}\label{eq:A_I,k}
				\begin{split}
					\boldsymbol{A}_I^{(k)}
					& = \frac{k_2 \rho^{(k-1)} R_{\text{ant}}}{2}\boldsymbol{M}_0 \\
					& + \frac{3 k_4 (\rho^{(k-1)})^2 R_{\text{ant}}^2}{8} \sum_{n=-N+1}^{N-1} 2\left((t_{I,n}^{\prime (k-1)})^*\boldsymbol{M}_{n} + t_{I,n}^{\prime (k-1)}\boldsymbol{M}_{n}^H\right) \\
					& + \frac{3{k_4}{(\rho^{(k-1)})^2}{R_{\text{ant}}^2}}{2} t_{P,0}^{\prime (k-1)}\boldsymbol{M}_{0}
				\end{split}
			\end{equation}
			\begin{equation}\label{eq:A_P,k}
				\begin{split}
					\boldsymbol{A}_P^{(k)}
					& = \frac{k_2 \rho^{(k-1)} R_{\text{ant}}}{2}\boldsymbol{M}_0 \\
					& + \frac{3 k_4 (\rho^{(k-1)})^2 R_{\text{ant}}^2}{8} \sum_{n=-N+1}^{N-1} (t_{P,n}^{\prime (k-1)})^*\boldsymbol{M}_{n} + t_{P,n}^{\prime (k-1)}\boldsymbol{M}_{n}^H \\
					& + \frac{3{k_4}{(\rho^{(k-1)})^2}{R_{\text{ant}}^2}}{2} t_{I,0}^{\prime (k-1)}\boldsymbol{M}_{0}
				\end{split}
			\end{equation}
			Plug \ref{eq:z_waveform_approx} into problem \ref{op:temp5} and solve it by iterative interior-point method. Given the optimal waveform matrices $\boldsymbol{W}_I$ and $\boldsymbol{W}_P$, the best rank-1 solution $\boldsymbol{w}_I$ and $\boldsymbol{w}_P$ can be obtained through randomization method. Details are omitted here. The algorithm for the waveform optimization subproblem is summarized in Algorithm ~.

			\begin{algorithm}
				\caption{Waveform Optimization}
				\label{alg:waveform}
				\begin{algorithmic}[1]
					\State \textbf{Input} $\boldsymbol{V}$
					\State \textbf{Initialize} $k=0$, $\boldsymbol{w}_{I/P}^{(0)}$, $\rho^{(0)}$, $t_{I/P,n}^{\prime (0)} \ \forall n$
					\Repeat
					\State $k = k + 1$
					\State Update SDR matrices $\boldsymbol{A}_{I/P}^{(k)}$ by \ref{eq:A_I,k} and \ref{eq:A_P,k}
					\State Obtain waveform matrices $\boldsymbol{W}_{I/P}^{(k)}$ and $\rho^{(k)}$ by solving problem \ref{op:temp5}
					\State Update auxiliary $t_{I/P,n}^{\prime (k)} \forall n$ by \ref{eq:t'} for SCA
					% \Until $\lVert \boldsymbol{W}_I^{(k)} - \boldsymbol{W}_I^{(k - 1)} \rVert_F / \lVert \boldsymbol{W}_I^{(k)} \rVert_F \le \epsilon$
					\Until $R^{(k)}-R^{(k-1)} \le \epsilon$
					\State Generate CSCG random vectors $\boldsymbol{r}_r \sim \mathcal{CN}(\boldsymbol{0},\boldsymbol{I}_{N}) \ \forall r$
					\State Perform EVD $\boldsymbol{W}_{I/P}^{\star}=\boldsymbol{U}_{\boldsymbol{W}_{I/P}^{\star}}\boldsymbol{\Sigma}_{\boldsymbol{W}_{I/P}^{\star}}\boldsymbol{U}_{\boldsymbol{W}_{I/P}^{\star}}^H$
					\State Construct candidate waveform vectors $\boldsymbol{w}_{I/P,r}=\boldsymbol{U}_{\boldsymbol{W}_{I/P}^{\star}}\boldsymbol{\Sigma}_{\boldsymbol{W}_{I/P}^{\star}}^{\frac{1}{2}}\boldsymbol{r}_r \ \forall r$ and corresponding matrices $\boldsymbol{W}_{I/P,r}=\boldsymbol{w}_{I/P,r}\boldsymbol{w}_{I/P,r}^H  \ \forall r$
					\State Select the best solution pair $\boldsymbol{W}_{I,r}^\star, \boldsymbol{W}_{P,r}^\star$ for problem \ref{op:temp5} and corresponding $\boldsymbol{w}_{I,r}^\star, \boldsymbol{w}_{P,r}^\star$
					\State \textbf{Output} $\boldsymbol{w}_I^\star, \boldsymbol{w}_P^\star, \rho^\star$
				\end{algorithmic}
			\end{algorithm}


			% * It holds that $\boldsymbol{M}_{I/P,0} \succeq 0$ and $\boldsymbol{M}_{I/P,-n}=\boldsymbol{M}_{I/P,n}^H$ for $n{\ne}0$.

			% To tackle the non-convex current constraint, we expand and rewrite the coupled channel terms in \ref{eq:z_k_truncated} as
			% \begin{equation}
			% 	\begin{split}
			% 		{h_{n_3}^*}{h_{n_1}}
			% 		&=(h_{D,n_3}^*+\boldsymbol{\Phi}_{n_3}^T\boldsymbol{v}^*)(h_{D,n_1}+\boldsymbol{v}^T{\boldsymbol{\Phi}_{n_1}^*})\\
			% 		&={h_{D,n_1}}{h_{D,n_3}^*}+{h_{D,n_1}}\boldsymbol{\Phi}_{n_3}^T\boldsymbol{v}^*+\boldsymbol{v}^T{\boldsymbol{\Phi}_{n_1}^*}h_{D,n_3}^*+\boldsymbol{v}^T{\boldsymbol{\Phi}_{n_1}^*}\boldsymbol{\Phi}_{n_3}^T\boldsymbol{v}^*\\
			% 		&=\bar{\boldsymbol{v}}^T\boldsymbol{R}_{n_3,n_1}\bar{\boldsymbol{v}}^*\\
			% 		&=\mathrm{Tr}(\boldsymbol{R}_{n_3,n_1}\boldsymbol{V}^*)
			% 	\end{split}
			% \end{equation}
			% where
			% \begin{equation}
			% 	\boldsymbol{R}_{n_3,n_1}=
			% 	\begin{bmatrix}
			% 		\boldsymbol{\Phi}_{n_1}^*\boldsymbol{\Phi}_{n_3}^T & \boldsymbol{\Phi}_{n_1}^*{h_{D,n_3}^*} \\
			% 		h_{D,n_1}{\boldsymbol{\Phi}_{n_3}^T}               & h_{D,n_1}{h_{D,n_3}^*}
			% 	\end{bmatrix}
			% \end{equation}
			% Similarly, we have ${h_{n_4}^*}{h_{n_2}}=\mathrm{Tr}(\boldsymbol{R}_{n_4,n_2}\boldsymbol{V}^*)$ and ${h_n^*}{h_n}=\mathrm{Tr}(\boldsymbol{R}_{n,n}\boldsymbol{V}^*)$. Thus, \ref{eq:z_k_truncated} is equivalent to
			% \begin{figure*}[b]
			% 	\hrule
			% 	\begin{equation}\label{eq:z_k_su_temp}
			% 		\begin{split}
			% 			z
			% 			&=\frac{1}{2}{k_2}{\rho}{R_{\text{ant}}}\sum_n{\mathrm{Tr}(\boldsymbol{R}_{n,n}\boldsymbol{V}^*)(\boldsymbol{w}_{I,n}^H\boldsymbol{w}_{I,n}+\boldsymbol{w}_{P,n}^H\boldsymbol{w}_{P,n})}\\
			% 			&\quad+\frac{3}{8}{k_4}{\rho^2}{R_{\text{ant}}^2}\sum_{\substack{{n_1},{n_2},{n_3},{n_4}\\{n_1}+{n_2}={n_3}+{n_4}}}\mathrm{Tr}(\boldsymbol{R}_{n_3,n_1}\boldsymbol{V}^*)\mathrm{Tr}(\boldsymbol{R}_{n_4,n_2}\boldsymbol{V}^*)\left(2(\boldsymbol{w}_{I,n_3}^H\boldsymbol{w}_{I,n_1})(\boldsymbol{w}_{I,n_4}^H\boldsymbol{w}_{I,n_2})+(\boldsymbol{w}_{P,n_3}^H\boldsymbol{w}_{P,n_1})(\boldsymbol{w}_{P,n_4}^H\boldsymbol{w}_{P,n_2})\right)\\
			% 			&\quad+\frac{3}{2}{k_4}{\rho^2}{R_{\text{ant}}^2}\sum_n{\mathrm{Tr}(\boldsymbol{R}_{n,n}\boldsymbol{V}^*)^2(\boldsymbol{w}_{I,n}^H\boldsymbol{w}_{I,n})(\boldsymbol{w}_{P,n}^H\boldsymbol{w}_{P,n})}
			% 		\end{split}
			% 	\end{equation}
			% \end{figure*}
		\end{subsubsection}


		% % ? AO: optimize weights with given phase shifts
		% To tackle the non-convex current constraint, we introduce a channel matrix $\boldsymbol{M}=\boldsymbol{h}^*\boldsymbol{h}^T \in \mathbb{C}^{N \times N}$ whose $(n_1,n_2)$ entry equals
		% \begin{equation}
		% 	\begin{split}
		% 		{h_{n_1}^*}{h_{n_2}}
		% 		&=(h_{D,n_1}^*+\boldsymbol{\Phi}_{n_1}^T\boldsymbol{v}^*)(h_{D,n_2}+\boldsymbol{v}^T{\boldsymbol{\Phi}_{n_2}^*})\\
		% 		&={h_{D,n_2}}{h_{D,n_1}^*}+{h_{D,n_2}}\boldsymbol{\Phi}_{n_1}^T\boldsymbol{v}^*+\boldsymbol{v}^T{\boldsymbol{\Phi}_{n_2}^*}h_{D,n_1}^*+\boldsymbol{v}^T{\boldsymbol{\Phi}_{n_2}^*}\boldsymbol{\Phi}_{n_1}^T\boldsymbol{v}^*\\
		% 		&=\bar{\boldsymbol{v}}^T\boldsymbol{R}_{n_1,n_2}\bar{\boldsymbol{v}}^*\\
		% 		&=\mathrm{Tr}(\boldsymbol{R}_{n_1,n_2}\boldsymbol{V}^*)
		% 	\end{split}
		% \end{equation}
		% where
		% \begin{equation}
		% 	\boldsymbol{R}_{n_1,n_2}=
		% 	\begin{bmatrix}
		% 		\boldsymbol{\Phi}_{n_2}^*\boldsymbol{\Phi}_{n_1}^T & \boldsymbol{\Phi}_{n_2}^*{h_{D,n_1}^*} \\
		% 		h_{D,n_2}{\boldsymbol{\Phi}_{n_1}^T}               & h_{D,n_2}{h_{D,n_1}^*}
		% 	\end{bmatrix}
		% \end{equation}
		% On top of this, $\boldsymbol{M}_n$ keeps the $n$-th ($n=-N+1,\dots,N-1$) diagonal of $\boldsymbol{M}$ and nulls the remaining entries. It holds that $\boldsymbol{M}_0 \succeq 0$ and $\boldsymbol{M}_{-n}=\boldsymbol{M}_{n}^H$ for $n{\ne}0$ \cite{Huang2017}. Therefore, \ref{eq:z_k_truncated} is transformed to \ref{eq:z_k_waveform}.
		% \begin{figure*}[b]
		% 	\hrule
		% 	\begin{equation}\label{eq:z_k_waveform}
		% 		\begin{split}
		% 			z
		% 			&=\frac{1}{2}{k_2}{\rho}{R_{\text{ant}}}(\boldsymbol{w}_I^H\boldsymbol{M}_0\boldsymbol{w}_I+\boldsymbol{w}_P^H\boldsymbol{M}_0\boldsymbol{w}_P)\\
		% 			% &\quad+\frac{3}{8}{k_4}{\rho^2}{R_{\text{ant}}^2}\left({\boldsymbol{w}_I^H\boldsymbol{M}_0}\boldsymbol{w}_I({\boldsymbol{w}_I^H\boldsymbol{M}_0}\boldsymbol{w}_I)^H+{\boldsymbol{w}_P^H\boldsymbol{M}_0}\boldsymbol{w}_P({\boldsymbol{w}_P^H\boldsymbol{M}_0}\boldsymbol{w}_P)^H\right)
		% 			% &\quad+\frac{3}{4}{k_4}{\rho^2}{R_{\text{ant}}^2}\left({\boldsymbol{w}_I^H\boldsymbol{M}_n}\boldsymbol{w}_I({\boldsymbol{w}_I^H\boldsymbol{M}_0}\boldsymbol{w}_I)^H+{\boldsymbol{w}_P^H\boldsymbol{M}_0}\boldsymbol{w}_P({\boldsymbol{w}_P^H\boldsymbol{M}_0}\boldsymbol{w}_P)^H\right)
		% 			&\quad+\frac{3}{8}{k_4}{\rho^2}{R_{\text{ant}}^2}\sum_{n=-N+1}^{N-1}{2(\boldsymbol{w}_I^H\boldsymbol{M}_n\boldsymbol{w}_I)(\boldsymbol{w}_I^H\boldsymbol{M}_n\boldsymbol{w}_I)^H+(\boldsymbol{w}_P^H\boldsymbol{M}_n\boldsymbol{w}_P)(\boldsymbol{w}_P^H\boldsymbol{M}_n\boldsymbol{w}_P)^H}\\
		% 			&\quad+\frac{3}{2}{k_4}{\rho^2}{R_{\text{ant}}^2}{(\boldsymbol{w}_I^H\boldsymbol{M}_0\boldsymbol{w}_I)(\boldsymbol{w}_P^H\boldsymbol{M}_0\boldsymbol{w}_P)}
		% 		\end{split}
		% 	\end{equation}
		% \end{figure*}
		% We then introduce auxiliary vectors $\boldsymbol{t}_{I/P}=[t_{{I/P},0},\dots,t_{{I/P},N-1}]^T \in \mathbb{C}^{N \times 1}$ with $t_{{I/P},n}=\boldsymbol{w}_{I/P}^H\boldsymbol{M}_n\boldsymbol{w}_{I/P}=\mathrm{Tr}(\boldsymbol{M}_n\boldsymbol{W}_{I/P})$ and transform \ref{eq:z_k_waveform} to
		% % and define coefficient matrix $\boldsymbol{A}_0=\mathrm{diag}\{\}$
		% \begin{equation}\label{eq:z_k_su_compact}
		% 	\begin{split}
		% 		z
		% 		&=\frac{1}{2}{k_2}{\rho}{R_{\text{ant}}}(t_{I,0}+t_{P,0})\\
		% 		&\quad+\frac{3}{8}{k_4}{\rho^2}{R_{\text{ant}}^2}(4\boldsymbol{t}_I^H\boldsymbol{t}_I+2\boldsymbol{t}_P^H\boldsymbol{t}_P-2t_{I,0}t_{I,0}^*-t_{P,0}t_{P,0}^*)\\
		% 		&\quad+\frac{3}{2}{k_4}{\rho^2}{R_{\text{ant}}^2}t_{I,0}t_{P,0}
		% 	\end{split}
		% \end{equation}






		% A semidefinite programming (SDP) problem is then formulated by dropping the rank-one constraint


		% which can be solved by softwares as CVX \cite{Grant2013}. [Use random vector or rank reduction method to extract $\boldsymbol{v}$ from $\boldsymbol{V}$]
		% It leads to a local optimum and is sensitive to the initialization of $\Theta_0$, $\boldsymbol{w}_I$ and $\boldsymbol{w}_P$.
		% Moreover, a feasible phase design strategy that maximizes the effective composite channel is used for phase shift initialization.
		% Although the problem is non-convex, the target function can be reformulated using the triangle inequality
		% \begin{equation}
		% 	\lvert{(h_{D,n}+\boldsymbol{h}_{R,n}\boldsymbol{\Theta}_0\boldsymbol{h}_{I,n})w_{I,n}}\rvert \le \lvert{h_{D,n}w_{I,n}}\rvert+\lvert{\boldsymbol{h}_{R,n}\boldsymbol{\Theta}_0\boldsymbol{h}_{I,n}w_{I,n}}\rvert
		% \end{equation}
		% where the equality holds if and only if the direct and IRS-aided links align with each other. In such case, we have $A_n=A_{D,n}+\boldsymbol{A}_{R,n}\boldsymbol{A}_{I,n}$ and and $\bar{\psi}_n=\angle{h_{D,n}}=\angle{\boldsymbol{h}_{R,n}\boldsymbol{\Theta}_0\boldsymbol{h}_{I,n}}$. Therefore, the optimal phase shift at the $l$-th reflect element is obtained in closed form
		% \begin{equation}
		% 	\theta_l^\star=\angle{h_{D,n}}-\angle{h_{R,n,l}}-\angle{h_{I,n,l}}
		% \end{equation}
		% Besides, it can be concluded from \ref{eq:R_k} and \ref{eq:z_k_truncated} that the optimal information and power phases coincide at subband $n$, being matched to composite frequency response as
		% \begin{equation}\label{eq:phi_n}
		% 	\phi_{I,n}^\star=\phi_{P,n}^\star=-\bar{\psi}_n
		% \end{equation}


	\end{subsection}
\end{section}

\bibliographystyle{IEEEtran}
\bibliography{library.bib}
\end{document}
