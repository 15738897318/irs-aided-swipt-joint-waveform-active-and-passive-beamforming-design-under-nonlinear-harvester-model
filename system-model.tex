\documentclass{IEEEtran}

\title{IRS-aided SWIPT}
\author{Yang Zhao}
\date{\today}

\usepackage{amsfonts}
\usepackage{amsmath}
\usepackage{amssymb}
\usepackage{cuted}
\usepackage{hyperref}
\usepackage{mathtools}
\usepackage{siunitx}
\usepackage{stfloats}
% \usepackage{nidanfloat}


\begin{document}

\begin{section} {System Model}
	Consider an IRS-aided multiuser MISO SWIPT system where the IRS not only assists the primal transmission but also retrieves Channel State Information (CSI) and harvests energy for its own operation. The $M$-antenna transmitter delivers information and power simultaneously, through the $U$-reflector IRS, to $K$ single-antenna users over $N$ orthogonal subbands. It is assumed that the IRS performs channel estimation in the first subframe and supports information and power transfer in the second subframe \cite{Zheng2019}. Due to the passive characteristics of IRS, we consider a Time-Division Duplexing (TDD) protocol where the CSI can be obtained by exploiting channel reciprocity. Perfect CSI is assumed at the AP and IRS to investigate the analytical upper-bound of the proposed system. The signals reflected by IRS for two and more times are assumed negligible and thus not considered. A quasi-static frequency-selective model is used for both the AP-user and AP-IRS-user links where the channels are assumed unchanged within each transmission frame. A superposition of multicarrier modulated and unmodulated waveforms, both transmitted on the same frequency bands, are designed adaptively to maximize the rate-energy tradeoff \cite{Clerckx2018b}. Note that although Frequency Selective Surface (FSS) has received much attention for wideband communications, active FSS requires RF-chains thus becomes prohibitive in IRS \cite{Kim2006,Xu2014}. In contrast, passive FSS is not reconfigurable with fixed physical characteristics \cite{Anwar2018}. Therefore, we assume a frequency-flat IRS with same reflection coefficients for all subbands. Two practical receiver architectures proposed in \cite{Zhang2013}, namely Time Switching (TS) and Power Splitting (PS), are investigated for the co-located information decoder and energy harvester. In the TS strategy, each transmission subframe is further divided into orthogonal data and energy slots with duration ratio $(1 - \alpha)$ and $\alpha$ respectively. Hence, the achievable rate-energy region can be obtained through a direct time sharing between wireless power transfer (WPT) with $\alpha=1$ and wireless information transfer (WIT) with $\alpha=0$. The adjustment of $\alpha$ has no impact on the transmit waveform and IRS elements design, since they are optimized individually in data and energy slots. In comparison, the PS scheme splits the received signal into data and energy streams with power ratio $(1 - \rho)$ and $\rho$. Note that $\rho$ is coupled with waveform and IRS design. Therefore, the rate-energy region is disclosed by placing different rate constraints and optimizing waveform, PS ratio and reflectors accordingly. Perfect synchronization is assumed among the three parties in both operations.

	\begin{subsection}	{Transmitted Signal}

		\begin{subsubsection} {Modulated Information Waveform}
			Assume the carrier are evenly spaced with equal bandwidth $B_s$ such that the frequency of the $n$-th subband is $f_n = f_0+(n-1) \Delta f$ ($n = 1,\dots,N$). The information symbol $\tilde{x}_{I,n}$ on subband $n$ is a capacity-achieving i.i.d. Circular Symmetric Complex Gaussian (CSCG) variable with zero mean and unit variance (i.e. $\tilde{x}_{I,n} \sim \mathcal{CN}(0,1)$). Denote $w_{I,n,m} = s_{I,n,m} e^{j\phi_{I,n,m}}$ as the information weight on the $n$-th subcarrier of the $m$-th ($m = 1,\dots,M$) transmit antenna with magnitude $s_{I,n,m}$ and phase $\phi_{I,n,m}$, the corresponding modulated symbol can be expressed as $x_{I,n,m} = \tilde{s}_{I,n,m} e^{j\tilde{\phi}_{I,n,m}}$, with $\tilde{s}_{I,n,m} = s_{I,n,m} \lvert\tilde{x}_{I,n}\rvert$ and $\tilde{\phi}_{I,n,m} = \phi_{I,n,m} + \angle{\tilde{x}_n}$, which also follows zero-mean CSCG distribution with variance equal to the subcarrier power (i.e. $x_{I,n,m} \sim \mathcal{CN}(0,s_{I,n,m}^2)$). Therefore, the transmit information waveform on the antenna $m$ at time $t$ writes as
			\begin{equation}	\label{eq:x_{I,m}(t)}
				\begin{split}
					x_{I,m}(t)
					&= \sum_{n=1}^{N} \tilde{s}_{I,n,m}(t) \cos \left( 2 \pi f_n t + \tilde{\phi}_{I,n,m}(t) \right)	\\
					&= \Re \left\{ \sum_{n=1}^{N} x_{I,n,m}(t) e^{j2 \pi f_n t} \right\}	\\
					&= \Re \left\{ \sum_{n=1}^{N} w_{I,n,m} \tilde{x}_{I,n}(t) e^{j2 \pi f_n t} \right\}
				\end{split}
			\end{equation}
			On top of this, the information waveform vector is spread over $M$ antennas
			\begin{equation}	\label{eq:x_I(t)}
				\boldsymbol{x}_I(t) = \Re \left\{ \sum_{n=1}^{N} \boldsymbol{w}_{I,n}\tilde{x}_n(t) e^{j2{\pi}{f_n}t} \right\}
			\end{equation}
			where $\boldsymbol{w}_{I,n}=[w_{I,n,1},\dots,w_{I,n,M}]^T \in \mathbb{C}^{M \times 1}$.
		\end{subsubsection}

		\begin{subsubsection} {Unmodulated Power Waveform}
			As suggested in \cite{Clerckx2018b,Clerckx2016a}, we use deterministic multisine waveform to boost the harvested energy. It has no randomness over time and the equivalent input power symbol $\tilde{x}_{P,n}$ is a constant on subband $n$ (i.e. $\tilde{x}_{P,n} = 1$). Therefore, on the $n$-th subcarrier of the $m$-th transmit antenna, the power weight $w_{P,n,m} = s_{P,n,m} e^{j\phi_{P,n,m}}$ represents the complex-valued sinewave (i.e. $x_{P,n,m}(t) = w_{P,n,m}$) in the baseband. The transmit power waveform on the antenna $m$ at time $t$ is given by
			\begin{equation}	\label{eq:x{P,m}(t)}
				\begin{split}
					x_{P,m}(t)
					&= \sum_{n=1}^{N} s_{P,n,m}(t) \cos \left( 2 \pi f_n t + \phi_{P,n,m}(t) \right)	\\
					&= \Re \left\{ \sum_{n=1}^{N} x_{P,n,m}(t) e^{j2 \pi f_n t} \right\}	\\
					&= \Re \left\{ \sum_{n=1}^{N} w_{P,n,m} e^{j2 \pi f_n t} \right\}
				\end{split}
			\end{equation}
			The corresponding power waveform vector is stacked across $M$ transmit antennas as
			\begin{equation}	\label{eq:x_P(t)}
				\boldsymbol{x}_P(t) = \Re \left\{ \sum_{n=1}^{N} \boldsymbol{w}_{P,n} e^{j2{\pi}{f_n}t} \right\}
			\end{equation}
			where $\boldsymbol{w}_{P,n}=[w_{P,n,1},\dots,w_{P,n,M}]^T \in \mathbb{C}^{M \times 1}$.
		\end{subsubsection}

		\begin{subsubsection} {Superposed Waveform}
			At time $t$, a superposition of the information and power waveform on antenna $m$ over subband $n$ writes as
			\begin{equation}	\label{eq:x_{n,m}(t)}
				\begin{split}
					x_{n,m}(t)
					&= x_{I,n,m}(t)+x_{P,n,m}(t)	\\
					&= w_{I,n,m} \tilde{x}_{I,n}(t)+w_{P,n,m}
				\end{split}
			\end{equation}
			The transmitted signal on antenna $m$ at time $t$ is a summation over all subbands
			\begin{equation}	\label{eq:x_m(t)}
				\begin{split}
					x_m(t)
					&= x_{I,m}(t)+x_{P,m}(t)	\\
					&= \Re \left\{ \sum_{n=1}^{N} (w_{I,n,m} \tilde{x}_{I,n}(t)+w_{P,n,m}) e^{j2 \pi f_n t} \right\}
				\end{split}
			\end{equation}
			Hence, the composite transmit signal vector is
			\begin{equation}	\label{x(t)}
				\begin{split}
					\boldsymbol{x}(t)
					&=\boldsymbol{x}_I(t)+\boldsymbol{x}_P(t)	\\
					&=\Re \left\{ \sum_{n=1}^{N} (\boldsymbol{w}_{I,n}\tilde{x}_n(t)+\boldsymbol{w}_{P,n}) e^{j2{\pi}{f_n}t} \right\}
				\end{split}
			\end{equation}
		\end{subsubsection}

	\end{subsection}

	\begin{subsection}	{Composite Channel Model}
		% Consider a single-IRS scenario in a multipath environment. The frequency response of the AP-user (i.e. direct) channel between antenna $m$ and user $k$ at subband $n$ is $h_{D,n,m,k}=A_{D,n,m,k}e^{j\bar{\psi}_{D,n,m,k}}=\sum_{l_{D,k}=1}^{L_{D,k}} \alpha_{l_{D,k}} e^{j\left( -2 \pi f_n \tau_{l_{D,k}}+\varsigma_{l_{D,k},n,m,k} \right)}$, where $L_{D,k}$ is the number of paths in the direct link of user $k$, $\alpha_{l_{D,k}}$ and $\tau_{l_{D,k}}$ are the delay and amplitude gain of the $l_{D,k}$-th path, $\varsigma_{l_{D,k},n,m,k}$ is the phase shift of the $l_{D,k}$-path between transmit antenna $m$ and the receive antenna of user $k$ at subband $n$. Similarly, the AP-IRS (i.e. incident) channel between antenna $m$ and element $u$ at subband $n$ writes as $h_{I,n,m,u}=A_{I,n,m,u}e^{j\bar{\psi}_{I,n,m,u}}=\sum_{l_{I,u}=1}^{L_{I,u}} \alpha_{l_{I,u}} e^{j\left( -2 \pi f_n \tau_{l_{I,u}}+\varsigma_{l_{I,u},n,m,u} \right)}$, and the IRS-user (i.e. reflective) channel between element $u$ and user $k$ is $h_{R,n,u,k}=A_{R,n,u,k}e^{j\bar{\psi}_{R,n,u,k}}=\sum_{l_{R,k}=1}^{L_{R,k}} \alpha_{l_{R,k}} e^{j\left( -2 \pi f_n \tau_{l_{R,k}}+\varsigma_{l_{R,k},n,u,k} \right)}$. Assume that $\max_{l_{D,k} \neq l_{D,k}'} \lvert \tau_{l_{D,k}}-\tau_{l_{D,k}'} \rvert \ll 1/B_s$ and $\max_{l_{I,u},l_{R,k} \neq l_{I,u}',l_{R,k}'} \lvert \left( \tau_{l_{I,u}}+\tau_{l_{R,k}} \right)-\left( \tau_{l_{I,u}'}+\tau_{l_{R,k}'} \right) \rvert \ll 1/B_s$ such that $\tilde{x}_n(t)$ and $x_{n,m}(t)$ are narrowband signals, namely $\tilde{x}_n(t-\tau_{l_D}) = \tilde{x}_n(t-\tau_{l_{I}}-\tau_{l_R}) = \tilde{x}_n(t)$ and $x_{n,m}(t-\tau_{l_D}) = x_{n,m}(t-\tau_{l_{I}}-\tau_{l_R}) = x_{n,m}(t)$.;

		Denote the baseband equivalent channels from the AP to user $k$, from the AP to the IRS, and from the IRS to user $k$ as $\boldsymbol{h}_{D,k} \in \mathbb{C}^{1 \times MN}$, $\boldsymbol{H}_I \in \mathbb{C}^{U \times MN}$, and $\boldsymbol{h}_{R,k} \in \mathbb{C}^{1 \times UN}$ respectively. At subband $n$, we collect the frequency response of both direct and reflective links from $M$ and $U$ elements to user $k$ as $\boldsymbol{h}_{D,k,n}=[h_{D,k,n,1},\dots,h_{D,k,n,M}] \in \mathbb{C}^{1 \times M}$ and $\boldsymbol{h}_{R,k,n}=[h_{R,k,n,1},\dots,h_{D,k,n,U}] \in \mathbb{C}^{1 \times U}$. Similarly, the $n$-th incident channel gain matrix from $M$ transmit antennas to $U$ reflectors is expressed as $\boldsymbol{H}_{I,n}=[\boldsymbol{h}_{I,1,n}^T,\dots,\boldsymbol{h}_{I,U,n}^T]^T \in \mathbb{C}^{U \times M}$, with $\boldsymbol{h}_{I,u,n}=[h_{I,u,n,1},\dots,h_{I,u,n,M}] \in \mathbb{C}^{1 \times M}$. Element $u$ of the IRS receives a superposed waveform through the multipath channel, then redistribute it by adjusting the reflection coefficient $r_u = \beta_u e^{j \theta_u}$, where $\beta_u \in [0,1]$ and $\theta_u \in [0,2\pi)$ are the amplitude reflection coefficient and phase shift respectively. Each passive reflector harvests a small portion ($1 - \beta_u$) of the signal to support CSI decoding and impedance matching. Construct a diagonal subband-wise IRS matrix by collecting the reflection coefficients of all reflectors onto the diagonal entries such that $\boldsymbol{R}_0 = \text{diag}(r_1, \dots, r_U) \in \mathbb{C}^{U \times U}$. With the assumption of frequency-flat reflection, the IRS matrix is formed by stacking $\boldsymbol{R}_0$ over $N$ subbands such that $\boldsymbol{R} = [\boldsymbol{R}_0^T,\dots,\boldsymbol{R}_0^T]^T \in \mathbb{C}^{UN \times U}$. The IRS-aided extra links can be modeled as a concatenation of the AP-IRS channel to the IRS reflection matrix and the IRS-user channel. Both the direct link and the IRS-aided link contributes to the composite channel $\boldsymbol{h}_k \in \mathbb{C}^{1 \times MN}$ as
		\begin{equation}	\label{eq:h_k}
			\boldsymbol{h}_k = \left( \boldsymbol{h}_{D,k}+\boldsymbol{h}_{R,k} \boldsymbol{R} \boldsymbol{\boldsymbol{H}_I} \right)
		\end{equation}
		On top of this, we select the $n$-th of every $N$ entries and combine them to form a subchannel as $\boldsymbol{h}_{k,n}=[h_{k,n,m},\dots,h_{k,n,M}] \in \mathbb{C}^{1 \times M}$ to describe the frequency response at subband $n$ for user $k$, where $h_{k,n,m}=h_{D,k,n,m}+\boldsymbol{h}_{R,k,n}\boldsymbol{R}_0\boldsymbol{H}_{I,n}$.
	\end{subsection}

	\begin{subsection}	{Received Signal}
		In the signal received by user $k$, the component from transmit antenna $m$ captures the contribution of information and power waveforms through both direct and reflective links as
		\begin{equation}	\label{eq:y_{m,k}(t)}
			y_{m,k}(t)=\Re\left\{\sum_{n=1}^N{(w_{I,n,m}\tilde{x}_{I,n}(t)+w_{P,n,m})h_{k,n,m}e^{j2{\pi}{f_n}t}}\right\}
		\end{equation}
		Therefore, the RF input signal to the antenna of user $k$ writes as
		\begin{equation}	\label{eq:y_k(t)}
			\begin{split}
				y_k(t)
				&=\Re\left\{\sum_{m=1}^M{y_{m,k}(t)}\right\}	\\
				&=\Re\left\{\sum_{m=1}^M{\sum_{n=1}^N{(w_{I,n,m}\tilde{x}_{I,n}(t)+w_{P,n,m})h_{k,n,m}e^{j2{\pi}{f_n}t}}}\right\}	\\
				&=\Re\left\{\sum_{n=1}^N{\boldsymbol{h}_{k,n}(\boldsymbol{w}_{P,n}+\boldsymbol{w}_{I,n}\tilde{x}_n)e^{j2{\pi}{f_n}t}}\right\}
			\end{split}
		\end{equation}
		% \begin{equation}	\label{eq:y_{m,k}(t)}
		% 	\begin{split}
		% 		y_{m,k}(t)
		% 		&= \Re \left\{ \sum_{n=1}^{N} \sum_{l_{D,k}=1}^{L_{D,k}} \alpha_{l_{D,k}} x_{n,m}(t-\tau_{l_{D,k}}) e^{j \left( -2 \pi f_n (t-\tau_{l_{D,k}})+\varsigma_{l_{D,k},n,m,k}\right) } \right\}	\\
		% 		&\quad+ \Re \left\{ \sum_{n=1}^{N} \sum_{l_{R,k}=1}^{L_{R,k}} \sum_{u=1}^{U} \alpha_{l_{R,k}} r_u \alpha_{l_{I,u}} x_{n,m}(t-\tau_{l_{R,k}}-\tau_{l_{I,u}}) e^{j \left( -2 \pi f_n (t-\tau_{l_{R,k}}-\tau_{l_{I,u}})+\varsigma_{l_{R,k},n,u,k}+\varsigma_{l_{I,u},n,m,u}\right)} \right\}	\\
		% 		&\approx \Re \left\{ \sum_{n=1}^{N} h_{D,n,m,k}(t) x_{n,m}(t) e^{j2 \pi f_n t} \right\}	\\
		% 		&\quad+ \Re \left\{ \sum_{n=1}^{N} \sum_{u=1}^{U} h_{R,n,u,k}(t) r_u h_{I,n,m,u}(t) x_{n,m}(t) e^{j2 \pi f_n t} \right\}	\\
		% 		&= \Re \left\{ \sum_{n=1}^{N} {s_{P,n,m}}{A_{D,n,m,k}} e^{j\left( 2{\pi}{f_n}{t}+\phi_{P,n,m}+\bar{\psi}_{D,n,m,k} \right)} \right\}	\\
		% 		&\quad+ \Re \left\{ \sum_{n=1}^{N} \sum_{u=1}^{U} {s_{P,n,m}}{A_{R,n,u,k}}{\beta_u}{A_{I,n,m,u}} e^{j\left( 2{\pi}{f_n}{t}+\phi_{P,n,m}+\bar{\psi}_{R,n,u,k}+\theta_u+\bar{\psi}_{I,n,m,u} \right)} \right\}	\\
		% 		&\quad+ \Re \left\{ \sum_{n=1}^{N} {\tilde{s}_{I,n,m}}{A_{D,n,m,k}} e^{j\left( 2{\pi}{f_n}{t}+\tilde{\phi}_{I,n,m}+\bar{\psi}_{D,n,m,k} \right)} \right\}	\\
		% 		&\quad+ \Re \left\{ \sum_{n=1}^{N} \sum_{u=1}^{U} {\tilde{s}_{I,n,m}}{A_{R,n,u,k}}{\beta_u}{A_{I,n,m,u}} e^{j\left( 2{\pi}{f_n}{t}+\tilde{\phi}_{I,n,m}+\bar{\psi}_{R,n,u,k}+\theta_u+\bar{\psi}_{I,n,m,u} \right)} \right\}
		% 	\end{split}
		% \end{equation}
		% Therefore, the RF signal input to the antenna at user $k$ is
		% \begin{equation}	\label{eq:y_{k}(t)}
		% 	y_{k}(t) = \sum_{m=1}^{M} y_{m,k}(t)
		% \end{equation}
	\end{subsection}

	\begin{subsection}	{Information Decoder}
		In the
	\end{subsection}
\end{section}

\bibliographystyle{IEEEtran}
\bibliography{library.bib}
\end{document}
