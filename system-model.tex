\documentclass{IEEEtran}

\title{IRS-aided SWIPT}
\author{Yang Zhao}
\date{\today}

\usepackage{amsfonts}
\usepackage{amsmath}
\usepackage{amssymb}
\usepackage{cuted}
\usepackage{hyperref}
\usepackage{mathtools}
\usepackage{siunitx}
\usepackage{stfloats}
% \usepackage{nidanfloat}

\algrenewcommand{\algorithmicrepeat}{\textbf{Repeat}}
\algrenewcommand{\algorithmicuntil}{\textbf{Until}}


\begin{document}

\begin{section}{System Model}
	Consider an IRS-aided multiuser SISO SWIPT system where the IRS not only assists the primal transmission but also retrieves Channel State Information (CSI) and harvests energy for its own operation. The single-antenna transmitter delivers information and power simultaneously, through the $L$-reflector IRS, to $K$ single-antenna users over $N$ orthogonal subbands. Assume a total bandwidth $B$ and evenly-spaced carriers around center frequency $f_0$. Denote the frequency of the $n$-th subband as $f_n$ ($n=1,\dots,N$). Suppose each subband is allocated to one user per time slot. It is assumed that the IRS performs channel estimation in the first subframe and supports information and power transfer in the second subframe \cite{Zheng2019}. Due to the passive characteristics of IRS, we consider a Time-Division Duplexing (TDD) protocol where the CSI can be obtained by exploiting channel reciprocity. Perfect CSI is assumed at the AP and IRS to investigate the analytical upper-bound of the proposed system. A quasi-static frequency-selective model is used for both the AP-user and AP-IRS-user links where the channels are assumed unchanged within each transmission frame. The signals reflected by IRS for two and more times are assumed negligible and thus not considered. Note that although Frequency Selective Surface (FSS) has received much attention for wideband communications, active FSS requires RF-chains thus becomes prohibitive in IRS \cite{Kim2006,Xu2014}. Since passive FSS is not reconfigurable with fixed physical characteristics \cite{Anwar2018}, we assume a frequency-flat IRS with the same reflection coefficients for all subbands. Since a deterministic multisine waveform can boost the energy transfer efficiency \cite{Clerckx2016a} and creates no interference to the information signal \cite{Clerckx2018b}, we use a superposition of multicarrier modulated and unmodulated waveforms, both transmitted on the same frequency bands, to maximize the rate-energy tradeoff. Two practical receiver architectures proposed in \cite{Zhang2013}, namely Time Switching (TS) and Power Splitting (PS), are investigated for the co-located information decoder and energy harvester. In the TS strategy, each transmission subframe is further divided into orthogonal data and energy slots, with duration ratio $(1-\lambda)$ and $\lambda$ respectively. Hence, the achievable rate-energy region can be obtained through a time sharing between wireless power transfer (WPT) with $\lambda=1$ and wireless information transfer (WIT) with $\lambda=0$. The adjustment of $\lambda$ has no impact on the transmit waveform and IRS elements design as they are optimized individually in data and energy slots. In comparison, the PS scheme splits the received signal into data and energy streams with power ratio $(1-\rho)$ and $\rho$ such that the PS ratio is coupled with waveform design. Perfect synchronization is assumed among the three parties in both scenarios.

	\begin{subsection}{Transmit Signal}
		Denote $\tilde{x}_{I,n}(t)$ as the information symbol transmitted over subband $n$, which belongs to one user at time $t$ and follows a capacity-achieving i.i.d. Circular Symmetric Complex Gaussian (CSCG) distribution $\tilde{x}_{I,n}\sim\mathcal{CN}(0,1)$. Let $\alpha_{k,n}$ be the allocation indicator, namely if subband $n$ is given to user $k$ ($k=1,\dots,K$), we have $\alpha_{k,n}=1$ and $\alpha_{k',n}=0 \ \forall k' \ne k$. Let $\boldsymbol{\alpha}_k=[\alpha_{k,1},\dots,\alpha_{k,N}]^T \in \mathbb{C}^{N \times 1}$. The superposed transmit signal at time $t$ is
		\begin{equation}\label{eq:x}
			x(t)=\Re\left\{\sum_{n=1}^N\left({w_{I,n}\tilde{x}_{I,n}(t)}+w_{P,n}\right){e^{j2{\pi}{f_n}{t}}}\right\}
		\end{equation}
		where $w_{I/P,n}=s_{I/P,n}e^{j\psi_{I/P,n}}$ collects the magnitude and phase of the information and power signal at frequency $n$. We further define the waveform vectors $\boldsymbol{w}_{I/P}=[w_{I/P,1},\dots,w_{I/P,N}]^T \in \mathbb{C}^{N{\times}1}$.
	\end{subsection}

	\begin{subsection}{Composite Channel Model}
		Denote the frequency response of the AP-user $k$ direct link as $\boldsymbol{h}_{D,k}=[h_{D,k,1},\dots,h_{D,k,N}]^T \in \mathbb{C}^{N \times 1}$. Let $[\boldsymbol{h}_{I,1},\dots,\boldsymbol{h}_{I,N}] \in \mathbb{C}^{L \times N}$ be the frequency response of AP-IRS incident channel, where $\boldsymbol{h}_{I,n} \in \mathbb{C}^{L \times 1}$ corresponds to the $n$-th incident channel. Similarly, let $[\boldsymbol{h}_{R,k,1},\dots,\boldsymbol{h}_{R,k,N}]^H \in \mathbb{C}^{N \times L}$ be the frequency response of IRS-user $k$ reflective channel, where $\boldsymbol{h}_{R,k,n}^H \in \mathbb{C}^{1 \times L}$ corresponds to the $n$-th reflective channel. At the IRS, element $l$ ($l=1,\dots,L$) redistributes the received signal by adjusting the amplitude reflection coefficient $\beta_l \in [0,1]$ and phase shift $\theta_l \in [0,2\pi)$ \footnote{To investigate the performance upper bound of IRS, we suppose the reflection coefficient is maximized $\beta_l=1 \ \forall l$ while the phase shift is a continuous variable over $[0,2\pi)$.}. On top of this, the IRS matrix is constructed by collecting the reflection coefficients onto the main diagonal entries as $\boldsymbol{\Theta} = \mathrm{diag}\left\{\beta_1 e^{j \theta_1}, \dots, \beta_L e^{j \theta_L}\right\} \in \mathbb{C}^{L \times L}$. The IRS-aided link can be modeled as a concatenation of the AP-IRS channel, IRS reflection, and IRS-user $k$ channel, which for user $k$ over subband $n$ is
		\begin{equation}\label{eq:h_{E,k,n}}
			h_{E,k,n} = \boldsymbol{h}_{R,k,n}^H \boldsymbol{\Theta} \boldsymbol{h}_{I,n} = \boldsymbol{v}_{k,n}^H \boldsymbol{\phi}
		\end{equation}
		where $\boldsymbol{v}_{k,n}^H=\boldsymbol{h}_{R,k,n}^H \mathrm{diag}(\boldsymbol{h}_{I,n}) \in \mathbb{C}^{1 \times L}$ and $\boldsymbol{\phi}=[e^{j{\theta_1}},\dots,e^{j{\theta_L}}] \in \mathbb{C}^{L \times 1}$. Both direct and extra link contributes to the corresponding composite channel as
		\begin{equation}
			h_{k,n} = A_{k,n} e^{j\bar{\psi}_{k,n}}= h_{D,k,n} + \boldsymbol{v}_{k,n}^H \boldsymbol{\phi}
		\end{equation}
		where $A_{k,n}$ and $\bar{\psi}_{k,n}$ are the amplitude and phase of the composite channel of user $k$ at subband $n$. Let $\boldsymbol{V}_k^H=[\boldsymbol{v}_{k,1},\dots,\boldsymbol{v}_{k,N}]^H \in \mathbb{C}^{N \times L}$, the extra link for user $k$ is $\boldsymbol{h}_{E,k}=[h_{E,k,1},\dots,h_{E,k,N}]^T=\boldsymbol{V}_k^H \boldsymbol{\phi} \in \mathbb{C}^{N \times 1}$. Therefore, the composite channel of user $k$ is
		\begin{equation}\label{eq:h_k}
			\boldsymbol{h}_k = \boldsymbol{h}_{D,k} + \boldsymbol{V}_k^H \boldsymbol{\phi}
		\end{equation}
	\end{subsection}

	\begin{subsection}{Receive Signal}
		The RF signal received by user $k$ captures the contribution of information and power waveforms through both direct and IRS-aided links as
		\begin{equation}\label{eq:y_k}
			y_k(t)=\Re\left\{\sum_{n=1}^N{h_{{k,n}}}\left({w_{I,n}\tilde{x}_{I,n}(t)}+w_{P,n}\right){e^{j2{\pi}{f_n}{t}}}\right\}
		\end{equation}
		which can be divided into
		\begin{align}\label{eq:y_{I/P,k}}
			y_{I,k}(t) & = \Re\left\{\sum_{n=1}^N{h_{{k,n}}}{w_{I,n}\tilde{x}_{I,n}(t)}{e^{j2{\pi}{f_n}{t}}}\right\}\\
			y_{P,k}(t) & = \Re\left\{\sum_{n=1}^N{h_{{k,n}}}w_{P,n}{e^{j2{\pi}{f_n}{t}}}\right\}
		\end{align}
	\end{subsection}

	\begin{subsection}{Information Decoder}
		% ? OFDM interference
		A major benefit of the superposed waveform is that the power component $y_{P,k}(t)$ creates no interference to the information component $y_{I,k}(t)$. Hence, the achievable rate of user $k$ is
		\begin{equation}\label{eq:R_k}
			R_k(\boldsymbol{w}_I,\boldsymbol{\phi},\rho,\boldsymbol{\alpha}_k)=\sum_{n=1}^N\alpha_{k,n}{\log_2\left(1+\frac{(1-\rho)\lvert h_{k,n}w_{I,n} \rvert^2}{\sigma_n^2}\right)}
		\end{equation}
		where $\sigma_n^2$ is the variance of the noise at RF band and during RF-to-BB conversion on tone $n$. Rate \ref{eq:R_k} is achievable with either waveform cancellation or translated demodulation \cite{Clerckx2018b}.
	\end{subsection}

	\begin{subsection}{Energy Harvester}
		Consider a nonlinear diode model based on the Taylor expansion of a small signal model \cite{Clerckx2016a,Clerckx2018b}, which highlights the dependency of harvester output DC current on the received waveform of user $k$ as
		\begin{equation}\label{eq:i_k}
			i_k(\boldsymbol{w}_I,\boldsymbol{w}_P,\boldsymbol{\phi},\rho)\approx\sum_{i=0}^{\infty}{k_i'}{\rho^{i/2}}{R_{\text{ant}}^{i/2}}\mathcal{E}\left\{{\mathcal{A}\left\{y_k(t)^i\right\}}\right\}
		\end{equation}
		where $R_{\text{ant}}$ is the impedance of the receive antenna, $k_0'=i_s(e^{-i_kR_{\text{ant}}/nv_t}-1)$, $k_i'=i_se^{-i_kR_{\text{ant}}/nv_t}/i!(nv_t)^i$ for $i=1,\dots,\infty$, $i_s$ is saturation current, $n$ is diode ideality factor, $v_t$ is thermal voltage. For a fixed channel and waveform, $\mathcal{A}\left\{.\right\}$ extracts the DC component of the received signal while $\mathcal{E}\left\{.\right\}$ covers the expectation over $\tilde{x}_{I,n}$.

		With the assumption of evenly spaced frequencies, we have $\mathcal{E}\left\{y_k(t)^i\right\}=0 \ \forall i \ \mathrm{odd}$ such that the related terms has no contribution to DC components. For simplicity, we truncate the infinite series to the $n_0$-th order. Maximizing a truncated \ref{eq:i_k} is equivalent to maximizing a monotonic function \cite{Clerckx2016a}
		\begin{equation}\label{eq:z_k}
			z_k(\boldsymbol{w}_I,\boldsymbol{w}_P,\boldsymbol{\phi},\rho)=\sum_{i\,\text{even},i\ge2}^{n_0}{k_i}{\rho^{i/2}}{R_{\text{ant}}^{i/2}}{\mathcal{E}\left\{\mathcal{A}\left\{y_k(t)^i\right\}\right\}}
		\end{equation}
		where $k_i=i_s/i!(nv_t)^i$. We choose $n_0=4$ to investigate the fundamental impact of diode nonlinearity on waveform design. Note that $\mathcal{E}\left\{\lvert\tilde{x}_{I,n}\rvert^2\right\}=1$ and $\mathcal{E}\left\{\lvert\tilde{x}_{I,n}\rvert^4\right\}=2$, which can be interpreted as a modulation gain on the nonlinear terms of the output DC current.

		For simplicity, we define $\boldsymbol{W}_{I/P}=\boldsymbol{w}_{I/P}\boldsymbol{w}_{I/P}^H$ and $\boldsymbol{H}_k=\boldsymbol{h}_k\boldsymbol{h}_k^H$ as waveform matrices and channel matrix of user $k$. Let $\boldsymbol{W}_{I/P,n}$, $\boldsymbol{H}_{k,n}$ keep the $n$-th ($n=-N+1,\dots,N-1$) diagonal of $\boldsymbol{W}_{I/P}$, $\boldsymbol{H}_k$ and null the remaining entries, respectively. Due to the positive definiteness of $\boldsymbol{W}_{I/P}$ and $\boldsymbol{H}_k$, we have $\boldsymbol{W}_{I/P,-n}=\boldsymbol{W}_{I/P,n}^H$ and $\boldsymbol{H}_{k,-n}=\boldsymbol{H}_{k,n}^H$. Let $\beta_2={k_2}{R_{\text{ant}}}$ and $\beta_4={k_4}{R_{\text{ant}}^2}$. On top of this, nonzero terms in \ref{eq:z_k} are detailed in \ref{eq:z_k_terms_begin} -- \ref{eq:z_k_terms_end} such that the current expression reduces to \ref{eq:z_k_expand} -- \ref{eq:z_k_waveform}.
		\begin{figure*}[b]
			\hrule
			\begin{align}
				\mathcal{E}\left\{\mathcal{A}\left\{y_{I,k}^2(t)\right\}\right\}
				& = \frac{1}{2}\sum_n{(h_{k,n}w_{I,n})(h_{k,n}w_{I,n})^*}\label{eq:z_k_terms_begin}\\
				& = \frac{1}{2}\boldsymbol{h}_k^H\boldsymbol{W}_{I,0}^*\boldsymbol{h}_k = \frac{1}{2}\boldsymbol{w}_I^H\boldsymbol{H}_{k,0}^*\boldsymbol{w}_I\\
				\mathcal{E}\left\{\mathcal{A}\left\{y_{I,k}^4(t)\right\}\right\}
				& = \frac{3}{4}\sum_{\substack{{n_1},{n_2},{n_3},{n_4}\\{n_1}+{n_2}={n_3}+{n_4}}}{(h_{k,{n_1}}w_{I,{n_1}})(h_{k,{n_2}}w_{I,{n_2}})(h_{k,{n_3}}w_{I,{n_3}})^*(h_{k,{n_4}}w_{I,{n_4}})^*}\\
				& = \frac{3}{4}\sum_{n=-N+1}^{N-1}(\boldsymbol{h}_k^H\boldsymbol{W}_{I,n}^*\boldsymbol{h}_k)(\boldsymbol{h}_k^H\boldsymbol{W}_{I,n}^*\boldsymbol{h}_k)^* = \frac{3}{4}\sum_{n=-N+1}^{N-1}(\boldsymbol{w}_I^H\boldsymbol{H}_{k,n}^*\boldsymbol{w}_I)(\boldsymbol{w}_I^H\boldsymbol{H}_{k,n}^*\boldsymbol{w}_I)^*\\
				\mathcal{A}\left\{y_{P,k}^2(t)\right\}
				& = \frac{1}{2}\sum_n{(h_{k,n}w_{P,n})(h_{k,n}w_{P,n})^*}\\
				& = \frac{1}{2}\boldsymbol{h}_k^H\boldsymbol{W}_{P,0}^*\boldsymbol{h}_k = \frac{1}{2}\boldsymbol{w}_P^H\boldsymbol{H}_{k,0}^*\boldsymbol{w}_P\\
				\mathcal{A}\left\{y_{P,k}^4(t)\right\}
				& = \frac{3}{8}\sum_{\substack{{n_1},{n_2},{n_3},{n_4}\\{n_1}+{n_2}={n_3}+{n_4}}}{(h_{k,{n_1}}w_{P,{n_1}})(h_{k,{n_2}}w_{P,{n_2}})(h_{k,{n_3}}w_{P,{n_3}})^*(h_{k,{n_4}}w_{P,{n_4}})^*}\\
				& = \frac{3}{8}\sum_{n=-N+1}^{N-1}(\boldsymbol{h}_k^H\boldsymbol{W}_{P,n}^*\boldsymbol{h}_k)(\boldsymbol{h}_k^H\boldsymbol{W}_{P,n}^*\boldsymbol{h}_k)^* = \frac{3}{8}\sum_{n=-N+1}^{N-1}(\boldsymbol{w}_P^H\boldsymbol{H}_{k,n}^*\boldsymbol{w}_P)(\boldsymbol{w}_P^H\boldsymbol{H}_{k,n}^*\boldsymbol{w}_P)^*\label{eq:z_k_terms_end}
			\end{align}
		\end{figure*}
		\begin{figure*}[b]
			\hrule
			\begin{align}
				z_k(\boldsymbol{w}_I,\boldsymbol{w}_P,\boldsymbol{\phi},\rho)
				& = \beta_2\rho\left(\mathcal{E}\left\{\mathcal{A}\left\{y_{I,k}^2(t)\right\}\right\}+\mathcal{A}\left\{y_{P,k}^2(t)\right\}\right)+\beta_4\rho^2\left(\mathcal{E}\left\{\mathcal{A}\left\{y_{I,k}^4(t)\right\}\right\}+\mathcal{A}\left\{y_{P,k}^4(t)\right\}+6\mathcal{E}\left\{\mathcal{A}\left\{y_{I,k}^2(t)\right\}\right\}\mathcal{A}\left\{y_{P,k}^2(t)\right\}\right)\label{eq:z_k_expand}\\
				& = \frac{1}{2}\beta_2\rho(\boldsymbol{h}_k^H\boldsymbol{W}_{I,0}\boldsymbol{h}_k+\boldsymbol{h}_k^H\boldsymbol{W}_{P,0}\boldsymbol{h}_k)\nonumber\\
				& \quad+ \frac{3}{8}\beta_4\rho^2\sum_{n=-N+1}^{N-1}\left(2(\boldsymbol{h}_k^H\boldsymbol{W}_{I,n}^*\boldsymbol{h}_k)(\boldsymbol{h}_k^H\boldsymbol{W}_{I,n}^*\boldsymbol{h}_k)^* + (\boldsymbol{h}_k^H\boldsymbol{W}_{P,n}^*\boldsymbol{h}_k)(\boldsymbol{h}_k^H\boldsymbol{W}_{P,n}^*\boldsymbol{h}_k)^* \right)\nonumber\\
				& \quad+ \frac{3}{2}\beta_4\rho^2(\boldsymbol{h}_k^H\boldsymbol{W}_{I,0}\boldsymbol{h}_k)(\boldsymbol{h}_k^H\boldsymbol{W}_{P,0}\boldsymbol{h}_k)\label{eq:z_k_channel}\\
				& = \frac{1}{2}\beta_2\rho(\boldsymbol{w}_I^H\boldsymbol{H}_{k,0}\boldsymbol{w}_I+\boldsymbol{w}_P^H\boldsymbol{H}_{k,0}\boldsymbol{w}_P)\nonumber\\
				& \quad+ \frac{3}{8}\beta_4\rho^2\sum_{n=-N+1}^{N-1}\left(2(\boldsymbol{w}_I^H\boldsymbol{H}_{k,n}^*\boldsymbol{w}_I)(\boldsymbol{w}_I^H\boldsymbol{H}_{k,n}^*\boldsymbol{w}_I)^* + (\boldsymbol{w}_P^H\boldsymbol{H}_{k,n}^*\boldsymbol{w}_P)(\boldsymbol{w}_P^H\boldsymbol{H}_{k,n}^*\boldsymbol{w}_P)^* \right)\nonumber\\
				& \quad+ \frac{3}{2}\beta_4\rho^2(\boldsymbol{w}_I^H\boldsymbol{H}_{k,0}\boldsymbol{w}_I)(\boldsymbol{w}_P^H\boldsymbol{H}_{k,0}\boldsymbol{w}_P)\label{eq:z_k_waveform}
			\end{align}
		\end{figure*}
	\end{subsection}

	\begin{subsection}{Weighted Sum Rate-Energy Region}
		Define the achievable weighted sum rate-energy (WSR-E) region as
		\begin{equation}
			\begin{split}
				C_{R-I}(P)
				&\triangleq \biggl\{(R,I):R\le\sum_{k=1}^K{u_{I,k}R_k},I\le\sum_{k=1}^K u_{P,k}z_k,\\
				&\quad \frac{1}{2}({\boldsymbol{w}_I^H}{\boldsymbol{w}_I}+{\boldsymbol{w}_P^H}{\boldsymbol{w}_P}) \le P\biggr\}
			\end{split}
		\end{equation}
		where $P$ is the transmit power budget and $u_{I,k},u_{P,k}$ are the information and power weight of user $k$.
	\end{subsection}
\end{section}

\begin{section}{Single-User Optimization}
	Consider a single-user waveform and IRS optimization problem where $\boldsymbol{\alpha}=\boldsymbol{1}^{N \times 1}$. We characterize the rate-energy region through a current maximization problem subject to transmit power, rate, and IRS constraints
	\begin{maxi!}
			{\boldsymbol{w}_I,\boldsymbol{w}_P,\boldsymbol{\phi},\rho}{z(\boldsymbol{w}_I,\boldsymbol{w}_P,\boldsymbol{\phi},\rho)}{\label{op:su}}{}
			\addConstraint{\frac{1}{2}({\boldsymbol{w}_I^H}{\boldsymbol{w}_I}+{\boldsymbol{w}_P^H}{\boldsymbol{w}_P})\le{P}}
			\addConstraint{\sum_{n}{\log_2\left(1+\frac{(1-\rho)\lvert(h_{D,n}+\boldsymbol{v}_n^H\boldsymbol{\phi})w_{I,n}\rvert^2}{\sigma_n^2}\right)} \ge \bar{R}}
			\addConstraint{\lvert{\phi_l}\rvert=1, \quad l=1,\dots,L}
		\end{maxi!}
	Problem \ref{op:su} is intricate due to the non-convex objective function with coupled variables. In the next section, we propose two waveform and phase shift design algorithms for frequency-flat and frequency-selective IRS, respectively.

	\begin{subsection}{Frequency-Selective IRS}
		Each element in frequency-selective IRS is expected to provide subband-dependent reflection coefficients. Hence, $\boldsymbol{\phi}_n$ replaces $\boldsymbol{\phi}$ in \ref{op:su} and the IRS has a total degree of freedom (DoF) of $NL$. Note that $\lvert{(h_{D,n}+\boldsymbol{v}_n^H\boldsymbol{\phi}_n)w_{I,n}}\rvert \le \lvert{h_{D,n}w_{I,n}}\rvert+\lvert{\boldsymbol{v}_n^H\boldsymbol{\phi}_n w_{I,n}}\rvert$ where the equality holds when the direct and IRS-aided links are aligned. Therefore, we choose the phase shift of element $l$ at subband $n$ as
		\begin{equation}\label{eq:theta}
			\theta_{n,l}^\star = \angle{h}_{D,n} - \angle{h_{R,n,l}}-\angle{h_{I,n,l}}
		\end{equation}
		That is to say, the optimal phase shift is obtained in closed form in the single-user scenario, and the phase of the composite channel equals that of the direct channel. Moreover, it can be observed from \ref{eq:R_k} and \ref{eq:z_k_expand} that the optimal phases of information and power waveform are both match to the composite channel as
		\begin{equation}\label{eq:psi}
			\psi_{I,n}^{\star}=\psi_{P,n}^{\star}=\bar{\psi}_{n}
		\end{equation}
		By such a phase selection, we have
		\begin{equation}
			\lvert{(h_{D,n}+\boldsymbol{v}_n^H\boldsymbol{\phi}_n)w_{I,n}}\rvert = \lvert h_{D,n} \rvert \lvert w_{I,n} \rvert + \vert \boldsymbol{v}_n^H\boldsymbol{\phi}_n \rvert \lvert w_{I,n} \rvert
		\end{equation}
		Therefore, the original problem \ref{op:su} is reduced to an waveform magnitude optimization problem
		\begin{maxi!}
				{\boldsymbol{s}_I,\boldsymbol{s}_P,\rho}{z(\boldsymbol{s}_I,\boldsymbol{s}_P,\rho)}{\label{op:su_gp}}{}
				\addConstraint{\frac{1}{2}({\boldsymbol{s}_I^H}{\boldsymbol{s}_I}+{\boldsymbol{s}_P^H}{\boldsymbol{s}_P})\le{P}}
				\addConstraint{\sum_{n}{\log_2\left(1+\frac{(1-\rho){A_n^2}{s_{I,n}^2}}{\sigma_n^2}\right)} \ge \bar{R}}
			\end{maxi!}
		\begin{figure*}[b]
			\hrule
			\begin{equation}\label{eq:z_gp}
				\begin{split}
					z(\boldsymbol{s}_I,\boldsymbol{s}_P,\rho)
					&=\frac{1}{2}{\beta_2}{\rho}\sum_n{A_n^2(s_{I,n}^2+s_{P,n}^2)}\\
					&\quad+\frac{3}{8}{\beta_4}{\rho^2}\sum_{\substack{{n_1},{n_2},{n_3},{n_4}\\{n_1}+{n_2}={n_3}+{n_4}}}{A_{n_1}A_{n_2}A_{n_3}A_{n_4}(2s_{I,n_1}s_{I,n_2}s_{I,n_3}s_{I,n_4}+s_{P,n_1}s_{P,n_2}s_{P,n_3}s_{P,n_4})}\\
					&\quad+\frac{3}{2}{\beta_4}{\rho^2}\sum_n{{A_n^4}{s_{I,n}^2}{s_{P,n}^2}}
				\end{split}
			\end{equation}
		\end{figure*}
		with $z$ given by \ref{eq:z_gp}. We introduce an auxiliary variable $t''$ and transform problem \ref{op:su_gp} into
		\begin{mini!}
				{\boldsymbol{s}_I,\boldsymbol{s}_P,\rho,t''}{\frac{1}{t''}}{\label{op:su_gp_1}}{}
				\addConstraint{\frac{1}{2}({\boldsymbol{s}_I^H}{\boldsymbol{s}_I}+{\boldsymbol{s}_P^H}{\boldsymbol{s}_P}) \le P}
				\addConstraint{\frac{t''}{z(\boldsymbol{s}_I,\boldsymbol{s}_P,\rho)} \le 1}
				\addConstraint{\frac{2^{\bar{R}}}{\prod_n \left(1+\frac{(1-\rho){A_n^2}{s_{I,n}^2}}{\sigma_n^2}\right)} \le 1}
			\end{mini!}
		Problem \ref{op:su_gp_1} is a Reversed Geometric Program which can be transformed to standard Geometric Program (GP). The idea is to decompose the information and power posynomials as sum of monomials, then derive their upper bounds using Arithmetic Mean-Geometric Mean (AM-GM) inequality \cite{Clerckx2018b,Chiang2005}. Let $z(\boldsymbol{s}_I,\boldsymbol{s}_P,\rho)=\sum_{m=1}^{M}{g_{P,m}(\boldsymbol{s}_I,\boldsymbol{s}_P,\rho)}$, problem \ref{op:su_gp_1} is equivalent to
		\begin{mini}
			{\boldsymbol{s}_I,\boldsymbol{s}_P,\rho,\bar{\rho},t''}{\frac{1}{t''}}{\label{op:su_gp_2}}{}
			\addConstraint{\frac{1}{2}({\boldsymbol{s}_I^H}{\boldsymbol{s}_I}+{\boldsymbol{s}_P^H}{\boldsymbol{s}_P})\le{P}}
			\addConstraint{{t''}\prod_m{\left(\frac{g_{P,m}(\boldsymbol{s}_I,\boldsymbol{s}_P,\rho)}{\gamma_{P,m}}\right)^{-\gamma_{P,m}}}\le{1}}
			\addConstraint{2^{\bar{R}}\prod_{n}{\left(\frac{1}{\gamma_{I,n,1}}\right)^{-\gamma_{I,n,1}} \left(\frac{\bar{\rho}{A_n^2}{s_{I,n}^2}}{{\sigma_n^2}{\gamma_{I,n,2}}}\right)^{-\gamma_{I,n,2}}}\le{1}}
			\addConstraint{\rho + \bar{\rho} \le 1}
		\end{mini}
		where $\gamma_{I,n,1},\gamma_{I,n,2} \ge 0$, $\gamma_{I,n,1}+\gamma_{I,n,2}=1$, $\gamma_{P,m} \ge 0 \ \forall m$, and $\sum_{m=1}^{M}{\gamma_m}=1$. The tightness of the AM-GM inequality depends on $\{\gamma_{I,n},\gamma_P\}$ that require successive update. At iteration $i$, we choose \cite{Clerckx2018b}
		\begin{align}
			\gamma_{I,n,1}^{(i)} & = \left. 1 \middle/ \left(1+\frac{\bar{\rho}^{(i-1)}{A_n^2}(s_{I,n}^{(i-1)})^2}{\sigma_n^2}\right) \right.\label{eq:gamma_I_1}\\
			\gamma_{I,n,2}^{(i)} & = 1 - \gamma_{I,n,1}^{(i)}\label{eq:gamma_I_2}\\
			\gamma_{P,m}^{(i)} & =\frac{g_{P,m}(\boldsymbol{s}_I^{(i-1)},\boldsymbol{s}_P^{(i-1)},\rho^{(i-1)})}{z(\boldsymbol{s}_I^{(i-1)},\boldsymbol{s}_P^{(i-1)},\rho^{(i-1)})}\label{eq:gamma_P}
		\end{align}
		and then solve problem \ref{op:su_gp_2}. The algorithm is summarized in Algorithm \ref{al:fs_waveform}.
		\begin{algorithm}
			\caption{FS-IRS: Waveform Amplitude Optimization}
			\label{al:fs_waveform}
			\begin{algorithmic}[1]
				\State \textbf{Input} $P, \bar{R}, \theta_l \ \forall l$
				\State \textbf{Initialize} $i \leftarrow 0$, $\boldsymbol{s}_{I/P}^{(0)}$, $\rho^{(0)}, \boldsymbol{h}$ by \ref{eq:h_k}, \ref{eq:theta}
				\Repeat
				\State $i \leftarrow i + 1$
				\State Update GM exponents $\{\gamma_{I,n}^{(i)},\gamma_{P}^{(i)}\}$ by \ref{eq:gamma_I_1} -- \ref{eq:gamma_P}
				\State Obtain waveform amplitude $\boldsymbol{s}_{I/P}^{(i)}$ and splitting ratio $\rho^{(i)}, \bar{\rho}^{(i)}$ by solving problem \ref{op:su_gp_2}
				\State Compute output DC current $z^{(i)}$ by \ref{eq:z_gp}
				\Until $\lvert z^{(i)} - z^{(i-1)} \rvert \le \epsilon$
				\State \textbf{Output} $\boldsymbol{s}_{I/P}^{\star}, \rho^{\star}, z^{\star}$
			\end{algorithmic}
		\end{algorithm}
	\end{subsection}

	\begin{subsection}{Frequency-Flat IRS}
		In contrast, frequency-flat IRS reflects all subbands equally with a DoF of $L$. We observe that
		\begin{equation}
			\begin{split}
				\lvert{h_{D,n}+\boldsymbol{v}_n^H\boldsymbol{\phi}}\rvert^2
				&=\lvert{h_{D,n}}\rvert^2+h_{D,n}^*\boldsymbol{v}_n^H\boldsymbol{\phi}+\boldsymbol{\phi}^H\boldsymbol{v}_n{h_{D,n}}+\boldsymbol{\phi}^H\boldsymbol{v}\boldsymbol{v}^H\boldsymbol{\phi}\\
				&=\bar{\boldsymbol{\phi}}^H\boldsymbol{R}_n\bar{\boldsymbol{\phi}}=\mathrm{Tr}(\boldsymbol{R}_n\bar{\boldsymbol{\phi}}\bar{\boldsymbol{\phi}}^H)=\mathrm{Tr}(\boldsymbol{R}_n\boldsymbol{\Phi})
			\end{split}
		\end{equation}
		where $t$ is an auxiliary variable with unit modulus and
		\begin{equation}
			\boldsymbol{R}_n=
			\begin{bmatrix}
				\boldsymbol{v}_n\boldsymbol{v}_n^H & \boldsymbol{v}_n{h_{D,n}} \\
				h_{D,n}^*{\boldsymbol{v}_n^H}      & h_{D,n}^*{h_{D,n}}
			\end{bmatrix},
			\quad \bar{\boldsymbol{\phi}}=\
			\begin{bmatrix}
				\boldsymbol{\phi} \\
				t
			\end{bmatrix},
			\quad \boldsymbol{\Phi}=\bar{\boldsymbol{\phi}}\bar{\boldsymbol{\phi}}^H
		\end{equation}
		with $\boldsymbol{R}_n,\boldsymbol{\Phi} \in \mathbb{C}^{(L+1) \times (L+1)}$ and $\bar{\boldsymbol{\phi}} \in \mathbb{C}^{(L+1) \times 1}$. On top of this, the outer product of composite channel rewrites as
		\begin{equation}
			\boldsymbol{h}\boldsymbol{h}^H=\boldsymbol{M}^H\boldsymbol{\Phi}\boldsymbol{M}
		\end{equation}
		where $\boldsymbol{M}=[\boldsymbol{V}^H,\boldsymbol{h}_D]^H \in \mathbb{C}^{(L+1) \times N}$. To reduce the design complexity, we propose an suboptimal alternating optimization algorithm that iteratively updates the phase shifts and the waveforms with the other being fixed.

		\begin{subsubsection}{Phase shift optimization}\label{se:fs_irs}
			The phase optimization subproblem is formed as follows. With a given waveform $\boldsymbol{w}_I,\boldsymbol{w}_P,\rho$, introduce auxiliary variables
			\begin{equation}\label{eq:t}
				\begin{split}
					t_{I/P,n}
					&=\boldsymbol{h}^H\boldsymbol{W}_{I/P,n}^*\boldsymbol{h}\\
					&=\mathrm{Tr}(\boldsymbol{h}\boldsymbol{h}^H\boldsymbol{W}_{I/P,n}^*)\\
					&=\mathrm{Tr}(\boldsymbol{M}^H\boldsymbol{\Phi}\boldsymbol{M}\boldsymbol{W}_{I/P,n}^*)\\
					&=\mathrm{Tr}(\boldsymbol{M}\boldsymbol{W}_{I/P,n}^*\boldsymbol{M}^H\boldsymbol{\Phi})\\
					&=\mathrm{Tr}(\boldsymbol{C}_{I/P,n}\boldsymbol{\Phi})
				\end{split}
			\end{equation}
			where we define $\boldsymbol{C}_{I/P,n}=\boldsymbol{M}\boldsymbol{W}_{I/P,n}^*\boldsymbol{M}^H \in \mathbb{C}^{(L+1)\times(L+1)}$. Therefore, \ref{eq:z_k_channel} rewrites as
			\begin{equation}\label{eq:z_irs}
				\begin{split}
					z(\boldsymbol{\Phi})
					&=\frac{1}{2}{\beta_2}{\rho}(t_{I,0}+t_{P,0})\\
					&\quad+\frac{3}{8}{\beta_4}{\rho^2}\sum_{n=-N+1}^{N-1}{2t_{I,n}t_{I,n}^*+t_{P,n}t_{P,n}^*}\\
					&\quad+\frac{3}{2}{\beta_4}{\rho^2}t_{I,0}t_{P,0}
				\end{split}
			\end{equation}
			We use first-order Taylor expansion to approximate the second-order terms in \ref{eq:z_irs}. Based on the variables optimized at iteration $i - 1$, the local approximation at iteration $i$ suggests \cite{Adali2010}
			\begin{equation}
				\begin{split}
					t_{I/P,n}^{(i)} (t_{I/P,n}^{(i)})^*
					& \approx 2 \Re\left\{t_{I/P,n}^{(i)} (t_{I/P,n}^{(i-1)})^*\right\} - t_{I/P,n}^{(i-1)} (t_{I/P,n}^{(i-1)})^* \\
					& = 2 \Re \left\{\mathrm{Tr}\left((t_{I/P,n}^{(i-1)})^*\boldsymbol{C}_{I/P,n}\boldsymbol{\Phi}^{(i)}\right)\right\}\\
					& \quad - t_{I/P,n}^{(i-1)} (t_{I/P,n}^{(i-1)})^*\\
					& = \mathrm{Tr}\left((t_{I/P,n}^{(i-1)})^*\boldsymbol{C}_{I/P,n}\boldsymbol{\Phi}^{(i)}\right)\\
					& \quad + \mathrm{Tr}\left(t_{I/P,n}^{(i-1)}\boldsymbol{C}_{I/P,n}^H\boldsymbol{\Phi}^{(i)}\right)\\
					& \quad- t_{I/P,n}^{(i-1)} (t_{I/P,n}^{(i-1)})^*
				\end{split}
			\end{equation}
			and
			\begin{equation}
				\begin{split}
					t_{I,0}^{(i)} t_{P,0}^{(i)}
					& \approx t_{I,0}^{(i)} t_{P,0}^{(i-1)} + t_{I,0}^{(i-1)} t_{P,0}^{(i)} - t_{I,0}^{(i-1)} t_{P,0}^{(i-1)}\\
					& = \mathrm{Tr}(t_{P,0}^{(i-1)}\boldsymbol{C}_{I,0}\boldsymbol{\Phi}^{(i)}) + \mathrm{Tr}(t_{I,0}^{(i-1)}\boldsymbol{C}_{P,0}\boldsymbol{\Phi}^{(i)})\\
					& \quad - t_{I,0}^{(i-1)} t_{P,0}^{(i-1)}
				\end{split}
			\end{equation}
			Therefore, the approximated current function at iteration $i$ is
			\begin{equation}\label{eq:z_irs_approx}
				\begin{split}
					\tilde{z}(\boldsymbol{\Phi}^{(i)})
					& = \mathrm{Tr}(\boldsymbol{A}^{(i)}\boldsymbol{\Phi}^{(i)}) \\
					& \quad - \frac{3}{8} \beta_4 \rho^2 \sum_{n=-N+1}^{N-1} 2t_{I,n}^{(i-1)} (t_{I,n}^{(i-1)})^* + t_{P,n}^{(i-1)} (t_{P,n}^{(i-1)})^* \\
					& \quad - \frac{3}{2} \beta_4 \rho^2 t_{I,0}^{(i-1)} t_{P,0}^{(i-1)}
				\end{split}
			\end{equation}
			where the corresponding Hermitian matrix $\boldsymbol{A}^{(i)}$ is
			\begin{equation}\label{eq:A}
				\begin{split}
					\boldsymbol{A}^{(i)}
					& = \frac{1}{2} \beta_2 \rho (\boldsymbol{C}_{I,0}+\boldsymbol{C}_{P,0})\\
					& \quad+ \frac{3}{8} \beta_4 \rho^2 \sum_{n=-N+1}^{N-1} 2\left((t_{I,n}^{(i-1)})^*\boldsymbol{C}_{I,n} + t_{I,n}^{(i-1)}\boldsymbol{C}_{I,n}^H\right)\\
					& \quad \quad + \left((t_{P,n}^{(i-1)})^*\boldsymbol{C}_{P,n} + t_{P,n}^{(i-1)}\boldsymbol{C}_{P,n}^H\right)\\
					& \quad+ \frac{3}{2} \beta_4 \rho^2 (t_{P,0}^{(i-1)}\boldsymbol{C}_{I,0} + t_{I,0}^{(i-1)}\boldsymbol{C}_{P,0})
				\end{split}
			\end{equation}
			Hence, problem \ref{op:su} is transformed to
			\begin{maxi!}
				{\boldsymbol{\boldsymbol{\Phi}}}{\tilde{z}(\boldsymbol{\Phi})}{\label{op:su_irs}}{\label{eq:su_irs_target}}
				\addConstraint{\sum_{n}{\log_2\left(1+\frac{(1-\rho)\lvert{w}_{I,n}\rvert^2\mathrm{Tr}(\boldsymbol{R}_n\boldsymbol{\Phi})}{\sigma_n^2}\right)} \ge \bar{R}}
				\addConstraint{\boldsymbol{\Phi}_{l,l}=1, \quad l=1,\dots,L+1}
				\addConstraint{\boldsymbol{\Phi}\succeq{0}}
				\addConstraint{\mathrm{rank}(\boldsymbol{\Phi})=1\label{co:irs_rank}}
			\end{maxi!}
			We then relax the rank constraint \ref{co:irs_rank} and solve the optimal IRS matrix $\boldsymbol{\Phi}^{\star}$ iteratively by interior-point method. If $\mathrm{rank}(\boldsymbol{\Phi}^{\star})=1$, the optimal phase shift vector $\bar{\boldsymbol{\phi}}^\star$ is attained by eigenvalue decomposition (EVD). Otherwise, a best feasible candidate $\bar{\boldsymbol{\phi}}^\star$ can be extracted through Gaussian randomization method \cite{Huang2010}. First, perform EVD on $\boldsymbol{\Phi}^{\star}$ as $\boldsymbol{\Phi}^{\star}=\boldsymbol{U}_{\boldsymbol{\Phi}^{\star}}\boldsymbol{\Sigma}_{\boldsymbol{\Phi}^{\star}}\boldsymbol{U}_{\boldsymbol{\Phi}^{\star}}^H$. Then, we generate $Q$ CSCG random vectors $\boldsymbol{r}_q \sim \mathcal{CN}(\boldsymbol{0},\boldsymbol{I}_{L+1}),\ q=1,\dots,Q$ and construct the corresponding candidates $\bar{\boldsymbol{\phi}}_q=\boldsymbol{U}_{\boldsymbol{\Phi}^{\star}}\boldsymbol{\Sigma}_{\boldsymbol{\Phi}^{\star}}^{\frac{1}{2}}\boldsymbol{r}_q$. Next, the optimal solution $\bar{\boldsymbol{\phi}}^\star$ is approximated by the one achieving maximum objective value \ref{eq:su_irs_target}. Finally, we can retrieve the phase shift by $\theta_l=\arg(\phi_l^\star/\phi_{L+1}^\star), \ l=1,\dots,L$. The algorithm for the phase optimization subproblem is summarized in Algorithm \ref{al:ff_irs}.
			\begin{algorithm}
				\caption{FF-IRS: Phase Shift Optimization}
				\label{al:ff_irs}
				\begin{algorithmic}[1]
					\State \textbf{Input} $\bar{R},\boldsymbol{w}_I,\boldsymbol{w}_P,\rho$
					\State \textbf{Initialize} $i \leftarrow 0,\boldsymbol{\Phi}^{(0)},\boldsymbol{R}_n,\boldsymbol{C}_{I/P,n},t_{I/P,n}^{(0)}\ \forall n$ by \ref{eq:t}
					\Repeat
					\State $i \leftarrow i + 1$
					\State Update SDR matrix $\boldsymbol{A}^{(i)}$ by \ref{eq:A}
					\State Obtain IRS matrix $\boldsymbol{\Phi}^{(i)}$ by solving problem \ref{op:su_irs}
					\State Update auxiliary $t_{I/P,n}^{(i)} \forall n$ by \ref{eq:t} for SCA
					\Until $\lvert \tilde{z}^{(i)}-\tilde{z}^{(i-1)} \rvert \le \epsilon$
					\State Perform EVD $\boldsymbol{\Phi}^{\star}=\boldsymbol{U}_{\boldsymbol{\Phi}^{\star}}\boldsymbol{\Sigma}_{\boldsymbol{\Phi}^{\star}}\boldsymbol{U}_{\boldsymbol{\Phi}^{\star}}^H$
					\State Generate random vectors $\boldsymbol{r}_q \sim \mathcal{CN}(\boldsymbol{0},\boldsymbol{I}_{L+1}) \ \forall q$
					\State Construct candidate IRS vectors $\bar{\boldsymbol{\phi}}_q=\boldsymbol{U}_{\boldsymbol{\Phi}^{\star}}\boldsymbol{\Sigma}_{\boldsymbol{\Phi}^{\star}}^{\frac{1}{2}}\boldsymbol{r}_q$ and corresponding matrices $\boldsymbol{\Phi}_q=\bar{\boldsymbol{\phi}}_q\bar{\boldsymbol{\phi}}_q^H  \ \forall q$
					\State Select the best solution $\boldsymbol{\Phi}^\star$ for problem \ref{op:su_irs} and corresponding $\boldsymbol{\phi}^\star$
					\State Compute IRS phase shift by $\theta_l=\arg(\phi_l^\star/\phi_{L+1}^\star), \ l=1,\dots,L$
					\State \textbf{Output} $\theta_l, \ l=1,\dots,L$
				\end{algorithmic}
			\end{algorithm}
		\end{subsubsection}

		\begin{subsubsection}{Waveform optimization}
			Consider the waveform optimization subproblem. Once $\boldsymbol{\Phi}$ is obtained, introduce auxiliary variables
			\begin{equation}\label{eq:t'}
				t_{I/P,n}' = \rho \boldsymbol{w}_{I/P}^H \boldsymbol{H}_n^* \boldsymbol{w}_{I/P} = \mathrm{Tr}(\rho \boldsymbol{H}_n^*\boldsymbol{W}_{I/P})
			\end{equation}
			Therefore, \ref{eq:z_k_waveform} rewrites as
			\begin{equation}\label{eq:z_waveform}
				\begin{split}
					z(\boldsymbol{W}_I,\boldsymbol{W}_P,\rho)
					&=\frac{1}{2} \beta_2 (t'_{I,0}+t'_{P,0})\\
					&\quad+\frac{3}{8} \beta_4 \sum_{n=-N+1}^{N-1}{2t'_{I,n}t_{I,n}^{\prime \ *}+t'_{P,n}t_{P,n}^{\prime \ *}}\\
					&\quad+\frac{3}{2} \beta_4 t'_{I,0}t'_{P,0}
				\end{split}
			\end{equation}
			At iteration $i$, the second-order terms are approximated by first-order Taylor series as
			\begin{equation}
				\begin{split}
					t_{I/P,n}^{\prime \ (i)} (t_{I/P,n}^{\prime \ (i)})^*
					& \approx 2 \Re\left\{t_{I/P,n}^{\prime \ (i)} (t_{I/P,n}^{\prime (i-1)})^*\right\} - t_{I/P,n}^{\prime (i-1)} (t_{I/P,n}^{\prime (i-1)})^* \\
					& = 2 \Re \left\{\mathrm{Tr}\left((t_{I/P,n}^{\prime (i-1)})^*\rho^{(i)}\boldsymbol{H}_{n}^*\boldsymbol{W}_{I/P}^{(i)}\right)\right\}\\
					& \quad - t_{I/P,n}^{\prime (i-1)} (t_{I/P,n}^{\prime (i-1)})^*\\
					& = \mathrm{Tr}\left((t_{I/P,n}^{\prime (i-1)})^*\rho^{(i)}\boldsymbol{H}_{n}^*\boldsymbol{W}_{I/P}^{(i)}\right)\\
					& \quad + \mathrm{Tr}\left(t_{I/P,n}^{\prime (i-1)}\rho^{(i)}\boldsymbol{H}_{n}^T\boldsymbol{W}_{I/P}^{(i)}\right)\\
					& \quad - t_{I/P,n}^{\prime (i-1)} (t_{I/P,n}^{\prime (i-1)})^*
				\end{split}
			\end{equation}
			and
			\begin{equation}
				\begin{split}
					t_{I,0}^{\prime \ (i)} t_{P,0}^{\prime \ (i)}
					& \approx t_{I,0}^{\prime \ (i)} t_{P,0}^{\prime (i-1)} + t_{I,0}^{\prime (i-1)} t_{P,0}^{\prime \ (i)} - t_{I,0}^{\prime (i-1)} t_{P,0}^{\prime (i-1)}\\
					& = \mathrm{Tr}(t_{P,0}^{\prime (i-1)}\rho^{(i)}\boldsymbol{H}_{0}^*\boldsymbol{W}_{I,0}^{(i)}) + \mathrm{Tr}(t_{I,0}^{\prime (i-1)}\rho^{(i)}\boldsymbol{H}_{0}^*\boldsymbol{W}_{P,0}^{(i)})\\
					& \quad - t_{I,0}^{\prime (i-1)} t_{P,0}^{\prime (i-1)}
				\end{split}
			\end{equation}
			Therefore, the approximated current function at iteration $i$ is
			\begin{equation}\label{eq:z_waveform_approx}
				\begin{split}
					\tilde{z}(\boldsymbol{W}_I^{(i)},\boldsymbol{W}_P^{(i)},\rho^{(i)})
					& = \rho^{(i)}\left(\mathrm{Tr}(\boldsymbol{A}_I^{(i)}\boldsymbol{W}_I^{(i)}) + \mathrm{Tr}(\boldsymbol{A}_P^{(i)}\boldsymbol{W}_P^{(i)})\right)\\
					& \quad - \frac{3}{8} \beta_4 \sum_{n=-N+1}^{N-1} 2t_{I,n}^{\prime (i-1)} (t_{I,n}^{\prime (i-1)})^* + t_{P,n}^{\prime (i-1)} (t_{P,n}^{\prime (i-1)})^*\\
					& \quad - \frac{3}{2} \beta_4 t_{I,0}^{\prime (i-1)} t_{P,0}^{\prime (i-1)}
				\end{split}
			\end{equation}
			where the Hermitian matrices $\boldsymbol{A}_I^{(i)}$ and $\boldsymbol{A}_P^{(i)}$ are
			\begin{align}
				\boldsymbol{A}_I^{(i)}
				& = \frac{1}{2} \beta_2 \boldsymbol{H}_0^*\nonumber\\
				& \quad + \frac{3}{4} \beta_4 \sum_{n=-N+1}^{N-1} (t_{I,n}^{\prime (i-1)})^*\boldsymbol{H}_{n}^* + t_{I,n}^{\prime (i-1)}\boldsymbol{H}_{n}^T\nonumber\\
				& \quad + \frac{3}{2} \beta_4 t_{P,0}^{\prime (i-1)}\boldsymbol{H}_{0}^*\label{eq:A_I}\\
				\boldsymbol{A}_P^{(i)}
				& = \frac{1}{2} \beta_2 \boldsymbol{H}_0^*\nonumber\\
				& \quad + \frac{3}{8} \beta_4 \sum_{n=-N+1}^{N-1} (t_{P,n}^{\prime (i-1)})^*\boldsymbol{H}_{n}^* + t_{P,n}^{\prime (i-1)}\boldsymbol{H}_{n}^T\nonumber\\
				& \quad + \frac{3}{2} \beta_4 t_{I,0}^{\prime (i-1)}\boldsymbol{H}_{0}^*\label{eq:A_P}
			\end{align}
			Hence, problem \ref{op:su} is transformed to
			\begin{maxi!}
				{\boldsymbol{W}_I,\boldsymbol{W}_P,\rho}{\tilde{z}(\boldsymbol{W}_I,\boldsymbol{W}_P,\rho)}{\label{op:su_waveform}}{\label{eq:su_waveform_target}}
				\addConstraint{\sum_{n}{\log_2\left(1+\frac{(1-\rho)W_{I,n,n}\mathrm{Tr}(\boldsymbol{R}_n\boldsymbol{\Phi})}{\sigma_n^2}\right)} \ge \bar{R}}
				\addConstraint{\frac{1}{2}\left(\mathrm{Tr}(\boldsymbol{W}_I)+\mathrm{Tr}(\boldsymbol{W}_P)\right) \le P}
				\addConstraint{\mathrm{rank}(\boldsymbol{W}_{I/P})=1\label{co:waveform_rank}}
			\end{maxi!}
			We then perform SDR and solve the optimal waveform matrix $\boldsymbol{W}_{I/P}^{\star}$ iteratively by interior-point method. $\boldsymbol{w}_I^{\star}$ and $\boldsymbol{w}_P^{\star}$ can be approximated via randomization method as in section \ref{se:fs_irs}. The algorithm is summarized in Algorithm \ref{al:fs_waveform}.

			\begin{algorithm}
				\caption{Waveform Optimization}
				\label{alg:waveform}
				\begin{algorithmic}[1]
					\State \textbf{Input} $\bar{R},P,\theta_l \ \forall l$
					\State \textbf{Initialize} $i \leftarrow 0$, $\boldsymbol{W}_{I/P}^{(0)}$, $\rho^{(0)}$, $t_{I/P,n}^{\prime (0)} \ \forall n$
					\Repeat
					\State $i \leftarrow i + 1$
					\State Update SDR matrices $\boldsymbol{A}_{I/P}^{(i)}$ by \ref{eq:A_I} and \ref{eq:A_P}
					\State Obtain waveform matrices $\boldsymbol{W}_{I/P}^{(i)}$ and $\rho^{(i)}$ by solving problem \ref{op:su_waveform}
					\State Update auxiliary $t_{I/P,n}^{\prime (k)} \forall n$ by \ref{eq:t'} for SCA
					\Until $\lvert \tilde{z}^{(i)}-\tilde{z}^{(i-1)} \rvert \le \epsilon$
					\State Perform EVD $\boldsymbol{W}_{I/P}^{\star}=\boldsymbol{U}_{\boldsymbol{W}_{I/P}^{\star}}\boldsymbol{\Sigma}_{\boldsymbol{W}_{I/P}^{\star}}\boldsymbol{U}_{\boldsymbol{W}_{I/P}^{\star}}^H$
					\State Generate random vectors $\boldsymbol{r}_q \sim \mathcal{CN}(\boldsymbol{0},\boldsymbol{I}_{N}) \ \forall q$
					\State Construct candidate waveform vectors $\boldsymbol{w}_{I/P,r}=\boldsymbol{U}_{\boldsymbol{W}_{I/P}^{\star}}\boldsymbol{\Sigma}_{\boldsymbol{W}_{I/P}^{\star}}^{\frac{1}{2}}\boldsymbol{r}_q$ and corresponding matrices $\boldsymbol{W}_{I/P,q}=\boldsymbol{w}_{I/P,q}\boldsymbol{w}_{I/P,q}^H  \ \forall q$
					\State Select the best solution $\boldsymbol{W}_{I}^\star,\boldsymbol{W}_{P}^\star$ for problem \ref{op:su_waveform} and corresponding $\boldsymbol{w}_{I}^\star, \boldsymbol{w}_{P}^\star$
					\State \textbf{Output} $\boldsymbol{w}_I^\star, \boldsymbol{w}_P^\star, \rho^\star$
				\end{algorithmic}
			\end{algorithm}
		\end{subsubsection}
	\end{subsection}
\end{section}

\bibliographystyle{IEEEtran}
\bibliography{library.bib}
\end{document}
